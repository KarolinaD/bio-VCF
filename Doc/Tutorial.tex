% This is the main LaTeX file which is used to produce the Biopython
% Tutorial documentation.
%
% If you just want to read the documentation, you can pick up ready-to-go
% copies in both pdf and html format from:
%
% http://biopython.org/DIST/docs/tutorial/Tutorial.html
% http://biopython.org/DIST/docs/tutorial/Tutorial.pdf
%
% If you want to typeset the documentation, you'll need a standard TeX/LaTeX
% distribution (I use teTeX, which works great for me on Unix platforms).
% Additionally, you need HeVeA (or at least hevea.sty), which can be
% found at:
%
% http://pauillac.inria.fr/~maranget/hevea/index.html
%
% You will also need the pictures included in the document, some of
% which are UMLish diagrams created by Dia
% (http://www.lysator.liu.se/~alla/dia/dia.html).
% These diagrams are available from Biopython git in the original dia
% format, which you can easily save as .png format using Dia itself.
% They are also checked in as the png files, so if you make
% modifications to the original dia files, the png files should also be
% changed.
%
% Once you're all set, you should be able to generate pdf by running:
%
% pdflatex Tutorial.tex  (to generate the first draft)
% pdflatex Tutorial.tex  (to get the cross references right)
% pdflatex Tutorial.tex  (to get the table of contents right)
%
% To generate the html, you'll need HeVeA installed. You should be
% able to just run:
%
% hevea -fix Tutorial.tex
%
% However, on older versions of hevea you may first need to remove the
% Tutorial.aux file generated by LaTeX, then run hevea twice to get
% the references right.
%
% If you want to typeset this and have problems, please report them
% at biopython-dev@biopython.org, and we'll try to get things resolved. We
% always love to have people interested in the documentation!

\documentclass{report}
\usepackage{url}
\usepackage{fullpage}
\usepackage{enumitem}
\usepackage{hevea}
\usepackage{graphicx}
\usepackage{listings}

% make everything have section numbers
\setcounter{secnumdepth}{4}

% Make links between references
\usepackage{hyperref}
\newif\ifpdf
\ifx\pdfoutput\undefined
  \pdffalse
\else
  \pdfoutput=1
  \pdftrue
\fi
\ifpdf
  \hypersetup{colorlinks=true, hyperindex=true, citecolor=red, urlcolor=blue}
\fi

\begin{document}

\begin{htmlonly}
\title{Biopython Tutorial and Cookbook}
\end{htmlonly}
\begin{latexonly}
\title{
%Hack to get the logo on the PDF front page:
\includegraphics[width=\textwidth]{images/biopython.jpg}\\
%Hack to get some white space using a blank line:
~\\
Biopython Tutorial and Cookbook}
\end{latexonly}

\author{Jeff Chang, Brad Chapman, Iddo Friedberg, Thomas Hamelryck, \\
Michiel de Hoon, Peter Cock, Tiago Antao, Eric Talevich, Bartek Wilczy\'{n}ski}
\date{Last Update -- 25 August 2016 (Biopython 1.69.dev0)}

%Hack to get the logo at the start of the HTML front page:
%(hopefully this isn't going to be too wide for most people)
\begin{rawhtml}
<P ALIGN="center">
<IMG ALIGN="center" SRC="images/biopython.jpg" TITLE="Biopython Logo" ALT="[Biopython Logo]" width="1024" height="288" />
</p>
\end{rawhtml}

\maketitle
\tableofcontents

%\chapter{Introduction}
%\label{chapter:introduction}
\include{Tutorial/chapter_introduction}

%\chapter{Quick Start -- What can you do with Biopython?}
%\label{chapter:quick-start}
\include{Tutorial/chapter_quick_start}

%\chapter{Sequence objects}
%\label{chapter:Bio.Seq}
\include{Tutorial/chapter_seq_objects}

%\chapter{Sequence annotation objects}
%\label{chapter:SeqRecord}
\chapter{Sequence annotation objects}
\label{chapter:SeqRecord}

Chapter~\ref{chapter:Bio.Seq} introduced the sequence classes.  Immediately ``above'' the \verb|Seq| class is the Sequence Record or \verb|SeqRecord| class, defined in the \verb|Bio.SeqRecord| module. This class allows higher level features such as identifiers and features (as \verb|SeqFeature| objects) to be associated with the sequence, and is used throughout the sequence input/output interface \verb|Bio.SeqIO| described fully in Chapter~\ref{chapter:Bio.SeqIO}.

If you are only going to be working with simple data like FASTA files, you can probably skip this chapter
for now. If on the other hand you are going to be using richly annotated sequence data, say from GenBank
or EMBL files, this information is quite important.

While this chapter should cover most things to do with the \verb|SeqRecord| and \verb|SeqFeature| objects in this chapter, you may also want to read the \verb|SeqRecord| wiki page (\url{http://biopython.org/wiki/SeqRecord}), and the built in documentation (also online -- \href{http://biopython.org/DIST/docs/api/Bio.SeqRecord.SeqRecord-class.html}{SeqRecord} and \href{http://biopython.org/DIST/docs/api/Bio.SeqFeature.SeqFeature-class.html}{SeqFeature}):

\begin{verbatim}
>>> from Bio.SeqRecord import SeqRecord
>>> help(SeqRecord)
...
\end{verbatim}

\section{The SeqRecord object}
\label{sec:SeqRecord}

The \verb|SeqRecord| (Sequence Record) class is defined in the \verb|Bio.SeqRecord| module. This class allows higher level features such as identifiers and features to be associated with a sequence (see Chapter~\ref{chapter:Bio.Seq}), and is the basic data type for the \verb|Bio.SeqIO| sequence input/output interface (see Chapter~\ref{chapter:Bio.SeqIO}).

The \verb|SeqRecord| class itself is quite simple, and offers the following information as attributes:

\begin{description}
  \item[.seq] -- The sequence itself, typically a \verb|Seq| object.

  \item[.id] -- The primary ID used to identify the sequence -- a string. In most cases this is something like an accession number.

  \item[.name] -- A ``common'' name/id for the sequence -- a string. In some cases this will be the same as the accession number, but it could also be a clone name. I think of this as being analogous to the LOCUS id in a GenBank record.

  \item[.description] -- A human readable description or expressive name for the sequence -- a string.

  \item[.letter\_annotations] -- Holds per-letter-annotations using a (restricted) dictionary of additional information about the letters in the sequence. The keys are the name of the information, and the information is contained in the value as a Python sequence (i.e. a list, tuple or string) with the same length as the sequence itself.  This is often used for quality scores (e.g. Section~\ref{sec:FASTQ-filtering-example}) or secondary structure information (e.g. from Stockholm/PFAM alignment files).

  \item[.annotations] -- A dictionary of additional information about the sequence. The keys are the name of the information, and the information is contained in the value. This allows the addition of more ``unstructured'' information to the sequence.

  \item[.features] -- A list of \verb|SeqFeature| objects with more structured information about the features on a sequence (e.g. position of genes on a genome, or domains on a protein sequence). The structure of sequence features is described below in Section~\ref{sec:seq_features}.

  \item[.dbxrefs] - A list of database cross-references as strings.
\end{description}

\section{Creating a SeqRecord}

Using a \verb|SeqRecord| object is not very complicated, since all of the
information is presented as attributes of the class. Usually you won't create
a \verb|SeqRecord| ``by hand'', but instead use \verb|Bio.SeqIO| to read in a
sequence file for you (see Chapter~\ref{chapter:Bio.SeqIO} and the examples
below).  However, creating \verb|SeqRecord| can be quite simple.

\subsection{SeqRecord objects from scratch}

To create a \verb|SeqRecord| at a minimum you just need a \verb|Seq| object:

%doctest
\begin{verbatim}
>>> from Bio.Seq import Seq
>>> simple_seq = Seq("GATC")
>>> from Bio.SeqRecord import SeqRecord
>>> simple_seq_r = SeqRecord(simple_seq)
\end{verbatim}

Additionally, you can also pass the id, name and description to the initialization function, but if not they will be set as strings indicating they are unknown, and can be modified subsequently:

%cont-doctest
\begin{verbatim}
>>> simple_seq_r.id
'<unknown id>'
>>> simple_seq_r.id = "AC12345"
>>> simple_seq_r.description = "Made up sequence I wish I could write a paper about"
>>> print(simple_seq_r.description)
Made up sequence I wish I could write a paper about
>>> simple_seq_r.seq
Seq('GATC', Alphabet())
\end{verbatim}

Including an identifier is very important if you want to output your \verb|SeqRecord| to a file.  You would normally include this when creating the object:

%doctest
\begin{verbatim}
>>> from Bio.Seq import Seq
>>> simple_seq = Seq("GATC")
>>> from Bio.SeqRecord import SeqRecord
>>> simple_seq_r = SeqRecord(simple_seq, id="AC12345")
\end{verbatim}

As mentioned above, the \verb|SeqRecord| has an dictionary attribute \verb|annotations|. This is used
for any miscellaneous annotations that doesn't fit under one of the other more specific attributes.
Adding annotations is easy, and just involves dealing directly with the annotation dictionary:

%cont-doctest
\begin{verbatim}
>>> simple_seq_r.annotations["evidence"] = "None. I just made it up."
>>> print(simple_seq_r.annotations)
{'evidence': 'None. I just made it up.'}
>>> print(simple_seq_r.annotations["evidence"])
None. I just made it up.
\end{verbatim}

Working with per-letter-annotations is similar, \verb|letter_annotations| is a
dictionary like attribute which will let you assign any Python sequence (i.e.
a string, list or tuple) which has the same length as the sequence:

%cont-doctest
\begin{verbatim}
>>> simple_seq_r.letter_annotations["phred_quality"] = [40, 40, 38, 30]
>>> print(simple_seq_r.letter_annotations)
{'phred_quality': [40, 40, 38, 30]}
>>> print(simple_seq_r.letter_annotations["phred_quality"])
[40, 40, 38, 30]
\end{verbatim}

The \verb|dbxrefs| and \verb|features| attributes are just Python lists, and
should be used to store strings and \verb|SeqFeature| objects (discussed later
in this chapter) respectively.

%TODO - Update this to show passing in the annotations etc to __init__ ?

\subsection{SeqRecord objects from FASTA files}

This example uses a fairly large FASTA file containing the whole sequence for \textit{Yersinia pestis biovar Microtus} str. 91001 plasmid pPCP1, originally downloaded from the NCBI.  This file is included with the Biopython unit tests under the GenBank folder, or online \href{http://biopython.org/SRC/biopython/Tests/GenBank/NC_005816.fna}{\texttt{NC\_005816.fna}} from our website.

The file starts like this - and you can check there is only one record present (i.e. only one line starting with a greater than symbol):

\begin{verbatim}
>gi|45478711|ref|NC_005816.1| Yersinia pestis biovar Microtus ... pPCP1, complete sequence
TGTAACGAACGGTGCAATAGTGATCCACACCCAACGCCTGAAATCAGATCCAGGGGGTAATCTGCTCTCC
...
\end{verbatim}

Back in Chapter~\ref{chapter:quick-start} you will have seen the function \verb|Bio.SeqIO.parse(...)|
used to loop over all the records in a file as \verb|SeqRecord| objects. The \verb|Bio.SeqIO| module
has a sister function for use on files which contain just one record which we'll use here (see Chapter~\ref{chapter:Bio.SeqIO} for details):

%TODO - line wrapping for doctest?
\begin{verbatim}
>>> from Bio import SeqIO
>>> record = SeqIO.read("NC_005816.fna", "fasta")
>>> record
SeqRecord(seq=Seq('TGTAACGAACGGTGCAATAGTGATCCACACCCAACGCCTGAAATCAGATCCAGG...CTG',
SingleLetterAlphabet()), id='gi|45478711|ref|NC_005816.1|', name='gi|45478711|ref|NC_005816.1|',
description='gi|45478711|ref|NC_005816.1| Yersinia pestis biovar Microtus ... sequence',
dbxrefs=[])
\end{verbatim}

Now, let's have a look at the key attributes of this \verb|SeqRecord|
individually -- starting with the \verb|seq| attribute which gives you a
\verb|Seq| object:

\begin{verbatim}
>>> record.seq
Seq('TGTAACGAACGGTGCAATAGTGATCCACACCCAACGCCTGAAATCAGATCCAGG...CTG', SingleLetterAlphabet())
\end{verbatim}

\noindent Here \verb|Bio.SeqIO| has defaulted to a generic alphabet, rather
than guessing that this is DNA. If you know in advance what kind of sequence
your FASTA file contains, you can tell \verb|Bio.SeqIO| which alphabet to use
(see Chapter~\ref{chapter:Bio.SeqIO}).

Next, the identifiers and description:

\begin{verbatim}
>>> record.id
'gi|45478711|ref|NC_005816.1|'
>>> record.name
'gi|45478711|ref|NC_005816.1|'
>>> record.description
'gi|45478711|ref|NC_005816.1| Yersinia pestis biovar Microtus ... pPCP1, complete sequence'
\end{verbatim}

As you can see above, the first word of the FASTA record's title line (after
removing the greater than symbol) is used for both the \verb|id| and
\verb|name| attributes. The whole title line (after removing the greater than
symbol) is used for the record description. This is deliberate, partly for
backwards compatibility reasons, but it also makes sense if you have a FASTA
file like this:

\begin{verbatim}
>Yersinia pestis biovar Microtus str. 91001 plasmid pPCP1
TGTAACGAACGGTGCAATAGTGATCCACACCCAACGCCTGAAATCAGATCCAGGGGGTAATCTGCTCTCC
...
\end{verbatim}

Note that none of the other annotation attributes get populated when reading a
FASTA file:

\begin{verbatim}
>>> record.dbxrefs
[]
>>> record.annotations
{}
>>> record.letter_annotations
{}
>>> record.features
[]
\end{verbatim}

In this case our example FASTA file was from the NCBI, and they have a fairly well defined set of conventions for formatting their FASTA lines. This means it would be possible to parse this information and extract the GI number and accession for example. However, FASTA files from other sources vary, so this isn't possible in general.

\subsection{SeqRecord objects from GenBank files}

As in the previous example, we're going to look at the whole sequence for \textit{Yersinia pestis biovar Microtus} str. 91001 plasmid pPCP1, originally downloaded from the NCBI, but this time as a GenBank file.
Again, this file is included with the Biopython unit tests under the GenBank folder, or online \href{http://biopython.org/SRC/biopython/Tests/GenBank/NC_005816.gb}{\texttt{NC\_005816.gb}} from our website.

This file contains a single record (i.e. only one LOCUS line) and starts:
\begin{verbatim}
LOCUS       NC_005816               9609 bp    DNA     circular BCT 21-JUL-2008
DEFINITION  Yersinia pestis biovar Microtus str. 91001 plasmid pPCP1, complete
            sequence.
ACCESSION   NC_005816
VERSION     NC_005816.1  GI:45478711
PROJECT     GenomeProject:10638
...
\end{verbatim}

Again, we'll use \verb|Bio.SeqIO| to read this file in, and the code is almost identical to that for used above for the FASTA file (see Chapter~\ref{chapter:Bio.SeqIO} for details):

\begin{verbatim}
>>> from Bio import SeqIO
>>> record = SeqIO.read("NC_005816.gb", "genbank")
>>> record
SeqRecord(seq=Seq('TGTAACGAACGGTGCAATAGTGATCCACACCCAACGCCTGAAATCAGATCCAGG...CTG',
IUPACAmbiguousDNA()), id='NC_005816.1', name='NC_005816',
description='Yersinia pestis biovar Microtus str. 91001 plasmid pPCP1, complete sequence.',
dbxrefs=['Project:10638'])
\end{verbatim}

You should be able to spot some differences already! But taking the attributes individually,
the sequence string is the same as before, but this time \verb|Bio.SeqIO| has been able to automatically assign a more specific alphabet (see Chapter~\ref{chapter:Bio.SeqIO} for details):

\begin{verbatim}
>>> record.seq
Seq('TGTAACGAACGGTGCAATAGTGATCCACACCCAACGCCTGAAATCAGATCCAGG...CTG', IUPACAmbiguousDNA())
\end{verbatim}

The \verb|name| comes from the LOCUS line, while the \verb|id| includes the version suffix.
The description comes from the DEFINITION line:

\begin{verbatim}
>>> record.id
'NC_005816.1'
>>> record.name
'NC_005816'
>>> record.description
'Yersinia pestis biovar Microtus str. 91001 plasmid pPCP1, complete sequence.'
\end{verbatim}

GenBank files don't have any per-letter annotations:

\begin{verbatim}
>>> record.letter_annotations
{}
\end{verbatim}

Most of the annotations information gets recorded in the \verb|annotations| dictionary, for example:

\begin{verbatim}
>>> len(record.annotations)
11
>>> record.annotations["source"]
'Yersinia pestis biovar Microtus str. 91001'
\end{verbatim}

The \verb|dbxrefs| list gets populated from any PROJECT or DBLINK lines:

\begin{verbatim}
>>> record.dbxrefs
['Project:10638']
\end{verbatim}

Finally, and perhaps most interestingly, all the entries in the features table (e.g. the genes or CDS features) get recorded as \verb|SeqFeature| objects in the \verb|features| list.

\begin{verbatim}
>>> len(record.features)
29
\end{verbatim}

\noindent We'll talk about \verb|SeqFeature| objects next, in
Section~\ref{sec:seq_features}.

\section{Feature, location and position objects}
\label{sec:seq_features}

\subsection{SeqFeature objects}

Sequence features are an essential part of describing a sequence. Once you get beyond the sequence itself, you need some way to organize and easily get at the more ``abstract'' information that is known about the sequence. While it is probably impossible to develop a general sequence feature class that will cover everything, the Biopython \verb|SeqFeature| class attempts to encapsulate as much of the information about the sequence as possible. The design is heavily based on the GenBank/EMBL feature tables, so if you understand how they look, you'll probably have an easier time grasping the structure of the Biopython classes.

The key idea about each \verb|SeqFeature| object is to describe a region on a parent sequence, typically a \verb|SeqRecord| object. That region is described with a location object, typically a range between two positions (see Section~\ref{sec:locations} below).

The \verb|SeqFeature| class has a number of attributes, so first we'll list them and their general features, and then later in the chapter work through examples to show how this applies to a real life example. The attributes of a SeqFeature are:

\begin{description}
  \item[.type] -- This is a textual description of the type of feature (for instance, this will be something like `CDS' or `gene').

  \item[.location] -- The location of the \verb|SeqFeature| on the sequence
  that you are dealing with, see Section~\ref{sec:locations} below. The
  \verb|SeqFeature| delegates much of its functionality to the location
  object, and includes a number of shortcut attributes for properties
  of the location:

  \begin{description}
    \item[.ref] -- shorthand for \verb|.location.ref| -- any (different)
    reference sequence the location is referring to. Usually just None.

    \item[.ref\_db] -- shorthand for \verb|.location.ref_db| -- specifies
    the database any identifier in \verb|.ref| refers to. Usually just None.

    \item[.strand] -- shorthand for \verb|.location.strand| -- the strand on
    the sequence that the feature is located on. For double stranded nucleotide
    sequence this may either be $1$ for the top strand, $-1$ for the bottom
    strand, $0$ if the strand is important but is unknown, or \texttt{None}
    if it doesn't matter. This is None for proteins, or single stranded sequences.
  \end{description}

  \item[.qualifiers] -- This is a Python dictionary of additional information about the feature. The key is some kind of terse one-word description of what the information contained in the value is about, and the value is the actual information. For example, a common key for a qualifier might be ``evidence'' and the value might be ``computational (non-experimental).'' This is just a way to let the person who is looking at the feature know that it has not be experimentally (i.~e.~in a wet lab) confirmed. Note that other the value will be a list of strings (even when there is only one string). This is a reflection of the feature tables in GenBank/EMBL files.

  \item[.sub\_features] -- This used to be used to represent features with complicated locations like `joins' in GenBank/EMBL files. This has been deprecated with the introduction of the \verb|CompoundLocation| object, and should now be ignored.

\end{description}

\subsection{Positions and locations}
\label{sec:locations}

The key idea about each \verb|SeqFeature| object is to describe a
region on a parent sequence, for which we use a location object,
typically describing a range between two positions. Two try to
clarify the terminology we're using:

\begin{description}
  \item[position] -- This refers to a single position on a sequence,
  which may be fuzzy or not. For instance, 5, 20, \verb|<100| and
  \verb|>200| are all positions.

  \item[location] -- A location is region of sequence bounded by
  some positions. For instance 5..20 (i.~e.~5 to 20) is a location.
\end{description}

I just mention this because sometimes I get confused between the two.

\subsubsection{FeatureLocation object}

Unless you work with eukaryotic genes, most \verb|SeqFeature| locations are
extremely simple - you just need start and end coordinates and a strand.
That's essentially all the basic \verb|FeatureLocation| object does.

%TODO -- add example here

In practise of course, things can be more complicated. First of all
we have to handle compound locations made up of several regions.
Secondly, the positions themselves may be fuzzy (inexact).

\subsubsection{CompoundLocation object}

Biopython 1.62 introduced the \verb|CompoundLocation| as part of
a restructuring of how complex locations made up of multiple regions
are represented.
The main usage is for handling `join' locations in EMBL/GenBank files.

%TODO -- add example here

\subsubsection{Fuzzy Positions}

So far we've only used simple positions. One complication in dealing
with feature locations comes in the positions themselves.
In biology many times things aren't entirely certain
(as much as us wet lab biologists try to make them certain!). For
instance, you might do a dinucleotide priming experiment and discover
that the start of mRNA transcript starts at one of two sites. This
is very useful information, but the complication comes in how to
represent this as a position. To help us deal with this, we have
the concept of fuzzy positions. Basically there are several types
of fuzzy positions, so we have five classes do deal with them:

\begin{description}
  \item[ExactPosition] -- As its name suggests, this class represents a position which is specified as exact along the sequence. This is represented as just a number, and you can get the position by looking at the \verb|position| attribute of the object.

  \item[BeforePosition] -- This class represents a fuzzy position
  that occurs prior to some specified site. In GenBank/EMBL notation,
  this is represented as something like \verb|`<13'|, signifying that
  the real position is located somewhere less than 13. To get
  the specified upper boundary, look at the \verb|position|
  attribute of the object.

  \item[AfterPosition] -- Contrary to \verb|BeforePosition|, this
  class represents a position that occurs after some specified site.
  This is represented in GenBank as \verb|`>13'|, and like
  \verb|BeforePosition|, you get the boundary number by looking
  at the \verb|position| attribute of the object.

  \item[WithinPosition] -- Occasionally used for GenBank/EMBL locations,
  this class models a position which occurs somewhere between two
  specified nucleotides. In GenBank/EMBL notation, this would be
  represented as `(1.5)', to represent that the position is somewhere
  within the range 1 to 5. To get the information in this class you
  have to look at two attributes. The \verb|position| attribute
  specifies the lower boundary of the range we are looking at, so in
  our example case this would be one. The \verb|extension| attribute
  specifies the range to the higher boundary, so in this case it
  would be 4. So \verb|object.position| is the lower boundary and
  \verb|object.position + object.extension| is the upper boundary.

  \item[OneOfPosition] -- Occasionally used for GenBank/EMBL locations,
  this class deals with a position where several possible values exist,
  for instance you could use this if the start codon was unclear and
  there where two candidates for the start of the gene. Alternatively,
  that might be handled explicitly as two related gene features.

  \item[UnknownPosition] -- This class deals with a position of unknown
  location. This is not used in GenBank/EMBL, but corresponds to the `?'
  feature coordinate used in UniProt.

\end{description}

Here's an example where we create a location with fuzzy end points:

%doctest
\begin{verbatim}
>>> from Bio import SeqFeature
>>> start_pos = SeqFeature.AfterPosition(5)
>>> end_pos = SeqFeature.BetweenPosition(9, left=8, right=9)
>>> my_location = SeqFeature.FeatureLocation(start_pos, end_pos)
\end{verbatim}

Note that the details of some of the fuzzy-locations changed in Biopython 1.59,
in particular for BetweenPosition and WithinPosition you must now make it explicit
which integer position should be used for slicing etc. For a start position this
is generally the lower (left) value, while for an end position this would generally
be the higher (right) value.

If you print out a \verb|FeatureLocation| object, you can get a nice representation of the information:

%cont-doctest
\begin{verbatim}
>>> print(my_location)
[>5:(8^9)]
\end{verbatim}

We can access the fuzzy start and end positions using the start and end attributes of the location:

%cont-doctest
\begin{verbatim}
>>> my_location.start
AfterPosition(5)
>>> print(my_location.start)
>5
>>> my_location.end
BetweenPosition(9, left=8, right=9)
>>> print(my_location.end)
(8^9)
\end{verbatim}

If you don't want to deal with fuzzy positions and just want numbers,
they are actually subclasses of integers so should work like integers:

%cont-doctest
\begin{verbatim}
>>> int(my_location.start)
5
>>> int(my_location.end)
9
\end{verbatim}

For compatibility with older versions of Biopython you can ask for the
\verb|nofuzzy_start| and \verb|nofuzzy_end| attributes of the location
which are plain integers:

%cont-doctest
\begin{verbatim}
>>> my_location.nofuzzy_start
5
>>> my_location.nofuzzy_end
9
\end{verbatim}

Notice that this just gives you back the position attributes of the fuzzy locations.

Similarly, to make it easy to create a position without worrying about fuzzy positions, you can just pass in numbers to the \verb|FeaturePosition| constructors, and you'll get back out \verb|ExactPosition| objects:

%cont-doctest
\begin{verbatim}
>>> exact_location = SeqFeature.FeatureLocation(5, 9)
>>> print(exact_location)
[5:9]
>>> exact_location.start
ExactPosition(5)
>>> int(exact_location.start)
5
>>> exact_location.nofuzzy_start
5
\end{verbatim}

That is most of the nitty gritty about dealing with fuzzy positions in Biopython.
It has been designed so that dealing with fuzziness is not that much more
complicated than dealing with exact positions, and hopefully you find that true!

\subsubsection{Location testing}

You can use the Python keyword \verb|in| with a \verb|SeqFeature| or location
object to see if the base/residue for a parent coordinate is within the
feature/location or not.

For example, suppose you have a SNP of interest and you want to know which
features this SNP is within, and lets suppose this SNP is at index 4350
(Python counting!). Here is a simple brute force solution where we just
check all the features one by one in a loop:

%doctest ../Tests/GenBank
\begin{verbatim}
>>> from Bio import SeqIO
>>> my_snp = 4350
>>> record = SeqIO.read("NC_005816.gb", "genbank")
>>> for feature in record.features:
...     if my_snp in feature:
...         print("%s %s" % (feature.type, feature.qualifiers.get('db_xref')))
...
source ['taxon:229193']
gene ['GeneID:2767712']
CDS ['GI:45478716', 'GeneID:2767712']
\end{verbatim}

Note that gene and CDS features from GenBank or EMBL files defined with joins
are the union of the exons -- they do not cover any introns.

%TODO - Add join example

\subsection{Sequence described by a feature or location}

A \verb|SeqFeature| or location object doesn't directly contain a sequence, instead the location (see Section~\ref{sec:locations}) describes how to get this from the parent sequence. For example consider a (short) gene sequence with location 5:18 on the reverse strand, which in GenBank/EMBL notation using 1-based counting would be \texttt{complement(6..18)}, like this:

%doctest
\begin{verbatim}
>>> from Bio.Seq import Seq
>>> from Bio.SeqFeature import SeqFeature, FeatureLocation
>>> example_parent = Seq("ACCGAGACGGCAAAGGCTAGCATAGGTATGAGACTTCCTTCCTGCCAGTGCTGAGGAACTGGGAGCCTAC")
>>> example_feature = SeqFeature(FeatureLocation(5, 18), type="gene", strand=-1)
\end{verbatim}

You could take the parent sequence, slice it to extract 5:18, and then take the reverse complement.
If you are using Biopython 1.59 or later, the feature location's start and end are integer like so this works:

%cont-doctest
\begin{verbatim}
>>> feature_seq = example_parent[example_feature.location.start:example_feature.location.end].reverse_complement()
>>> print(feature_seq)
AGCCTTTGCCGTC
\end{verbatim}

This is a simple example so this isn't too bad -- however once you have to deal with compound features (joins) this is rather messy. Instead, the \verb|SeqFeature| object has an \verb|extract| method to take care of all this:

%cont-doctest
\begin{verbatim}
>>> feature_seq = example_feature.extract(example_parent)
>>> print(feature_seq)
AGCCTTTGCCGTC
\end{verbatim}

The length of a \verb|SeqFeature| or location matches
that of the region of sequence it describes.

%cont-doctest
\begin{verbatim}
>>> print(example_feature.extract(example_parent))
AGCCTTTGCCGTC
>>> print(len(example_feature.extract(example_parent)))
13
>>> print(len(example_feature))
13
>>> print(len(example_feature.location))
13
\end{verbatim}

For simple \verb|FeatureLocation| objects the length is just
the difference between the start and end positions. However,
for a \verb|CompoundLocation| the length is the sum of the
constituent regions.

\section{Comparison}

The \verb|SeqRecord| objects can be very complex, but here's a simple example:

%doctest
\begin{verbatim}
>>> from Bio.Seq import Seq
>>> from Bio.SeqRecord import SeqRecord
>>> record1 = SeqRecord(Seq("ACGT"), id="test")
>>> record2 = SeqRecord(Seq("ACGT"), id="test")
\end{verbatim}

What happens when you try to compare these ``identical'' records?

%this is not a doctest:
\begin{verbatim}
>>> record1 == record2
...
\end{verbatim}

Perhaps surprisingly older versions of Biopython would use Python's default object
comparison for the \verb|SeqRecord|, meaning \verb|record1 == record2| would
only return \verb|True| if these variables pointed at the same object in memory.
In this example, \verb|record1 == record2| would have returned \verb|False|
here!

%this is not a doctest:
\begin{verbatim}
>>> record1 == record2  # on old versions of Biopython!
False
\end{verbatim}

As of Biopython 1.67, \verb|SeqRecord| comparison like \verb|record1 == record2|
will instead raise an explicit error to avoid people being caught out by this:

%cont-doctest
\begin{verbatim}
>>> record1 == record2
Traceback (most recent call last):
...
NotImplementedError: SeqRecord comparison is deliberately not implemented. Explicitly compare the attributes of interest.
\end{verbatim}

Instead you should check the attributes you are interested in, for example the
identifier and the sequence:

%cont-doctest
\begin{verbatim}
>>> record1.id == record2.id
True
>>> record1.seq == record2.seq
True
\end{verbatim}

Beware that comparing complex objects quickly gets complicated (see also
Section~\ref{sec:seq-comparison}).

\section{References}

Another common annotation related to a sequence is a reference to a journal or other published work dealing with the sequence. We have a fairly simple way of representing a Reference in Biopython -- we have a \verb|Bio.SeqFeature.Reference| class that stores the relevant information about a reference as attributes of an object.

The attributes include things that you would expect to see in a reference like \verb|journal|, \verb|title| and \verb|authors|. Additionally, it also can hold the \verb|medline_id| and \verb|pubmed_id| and a \verb|comment| about the reference. These are all accessed simply as attributes of the object.

A reference also has a \verb|location| object so that it can specify a particular location on the sequence that the reference refers to. For instance, you might have a journal that is dealing with a particular gene located on a BAC, and want to specify that it only refers to this position exactly. The \verb|location| is a potentially fuzzy location, as described in section~\ref{sec:locations}.

Any reference objects are stored as a list in the \verb|SeqRecord| object's \verb|annotations| dictionary under the key ``references''.
That's all there is too it. References are meant to be easy to deal with, and hopefully general enough to cover lots of usage cases.

\section{The format method}
\label{sec:SeqRecord-format}

The \verb|format()| method of the \verb|SeqRecord| class gives a string
containing your record formatted using one of the output file formats
supported by \verb|Bio.SeqIO|, such as FASTA:

\begin{verbatim}
from Bio.Seq import Seq
from Bio.SeqRecord import SeqRecord
from Bio.Alphabet import generic_protein

record = SeqRecord(Seq("MMYQQGCFAGGTVLRLAKDLAENNRGARVLVVCSEITAVTFRGPSETHLDSMVGQALFGD" \
                      +"GAGAVIVGSDPDLSVERPLYELVWTGATLLPDSEGAIDGHLREVGLTFHLLKDVPGLISK" \
                      +"NIEKSLKEAFTPLGISDWNSTFWIAHPGGPAILDQVEAKLGLKEEKMRATREVLSEYGNM" \
                      +"SSAC", generic_protein),
                   id="gi|14150838|gb|AAK54648.1|AF376133_1",
                   description="chalcone synthase [Cucumis sativus]")

print(record.format("fasta"))
\end{verbatim}
\noindent which should give:
\begin{verbatim}
>gi|14150838|gb|AAK54648.1|AF376133_1 chalcone synthase [Cucumis sativus]
MMYQQGCFAGGTVLRLAKDLAENNRGARVLVVCSEITAVTFRGPSETHLDSMVGQALFGD
GAGAVIVGSDPDLSVERPLYELVWTGATLLPDSEGAIDGHLREVGLTFHLLKDVPGLISK
NIEKSLKEAFTPLGISDWNSTFWIAHPGGPAILDQVEAKLGLKEEKMRATREVLSEYGNM
SSAC
\end{verbatim}

This \verb|format| method takes a single mandatory argument, a lower case string which is
supported by \verb|Bio.SeqIO| as an output format (see Chapter~\ref{chapter:Bio.SeqIO}).
However, some of the file formats \verb|Bio.SeqIO| can write to \emph{require} more than
one record (typically the case for multiple sequence alignment formats), and thus won't
work via this \verb|format()| method.  See also Section~\ref{sec:Bio.SeqIO-and-StringIO}.

\section{Slicing a SeqRecord}
\label{sec:SeqRecord-slicing}

You can slice a \verb|SeqRecord|, to give you a new \verb|SeqRecord| covering just
part of the sequence. What is important
here is that any per-letter annotations are also sliced, and any features which fall
completely within the new sequence are preserved (with their locations adjusted).

For example, taking the same GenBank file used earlier:

%doctest ../Tests/GenBank
\begin{verbatim}
>>> from Bio import SeqIO
>>> record = SeqIO.read("NC_005816.gb", "genbank")
\end{verbatim}
%TODO - support line wrapping in doctest
\begin{verbatim}
>>> record
SeqRecord(seq=Seq('TGTAACGAACGGTGCAATAGTGATCCACACCCAACGCCTGAAATCAGATCCAGG...CTG',
IUPACAmbiguousDNA()), id='NC_005816.1', name='NC_005816',
description='Yersinia pestis biovar Microtus str. 91001 plasmid pPCP1, complete sequence',
dbxrefs=['Project:58037'])
\end{verbatim}
%cont-doctest
\begin{verbatim}
>>> len(record)
9609
>>> len(record.features)
41
\end{verbatim}


For this example we're going to focus in on the \verb|pim| gene, \verb|YP_pPCP05|.
If you have a look at the GenBank file directly you'll find this gene/CDS has
location string \texttt{4343..4780}, or in Python counting \texttt{4342:4780}.
From looking at the file you can work out that these are the twelfth and
thirteenth entries in the file, so in Python zero-based counting they are
entries $11$ and $12$ in the \texttt{features} list:

%cont-doctest
\begin{verbatim}
>>> print(record.features[20])
type: gene
location: [4342:4780](+)
qualifiers:
    Key: db_xref, Value: ['GeneID:2767712']
    Key: gene, Value: ['pim']
    Key: locus_tag, Value: ['YP_pPCP05']
<BLANKLINE>
\end{verbatim}
%This one is truncated so can't use for doctest
\begin{verbatim}
>>> print(record.features[21])
type: CDS
location: [4342:4780](+)
qualifiers:
    Key: codon_start, Value: ['1']
    Key: db_xref, Value: ['GI:45478716', 'GeneID:2767712']
    Key: gene, Value: ['pim']
    Key: locus_tag, Value: ['YP_pPCP05']
    Key: note, Value: ['similar to many previously sequenced pesticin immunity ...']
    Key: product, Value: ['pesticin immunity protein']
    Key: protein_id, Value: ['NP_995571.1']
    Key: transl_table, Value: ['11']
    Key: translation, Value: ['MGGGMISKLFCLALIFLSSSGLAEKNTYTAKDILQNLELNTFGNSLSH...']
\end{verbatim}

Let's slice this parent record from 4300 to 4800 (enough to include the \verb|pim|
gene/CDS), and see how many features we get:

%cont-doctest
\begin{verbatim}
>>> sub_record = record[4300:4800]
\end{verbatim}
%TODO - Line wrapping for doctest?
\begin{verbatim}
>>> sub_record
SeqRecord(seq=Seq('ATAAATAGATTATTCCAAATAATTTATTTATGTAAGAACAGGATGGGAGGGGGA...TTA',
IUPACAmbiguousDNA()), id='NC_005816.1', name='NC_005816',
description='Yersinia pestis biovar Microtus str. 91001 plasmid pPCP1, complete sequence.',
dbxrefs=[])
\end{verbatim}
%cont-doctest
\begin{verbatim}
>>> len(sub_record)
500
>>> len(sub_record.features)
2
\end{verbatim}

Our sub-record just has two features, the gene and CDS entries for \verb|YP_pPCP05|:

%cont-doctest
\begin{verbatim}
>>> print(sub_record.features[0])
type: gene
location: [42:480](+)
qualifiers:
    Key: db_xref, Value: ['GeneID:2767712']
    Key: gene, Value: ['pim']
    Key: locus_tag, Value: ['YP_pPCP05']
<BLANKLINE>
\end{verbatim}
%As above, output is truncated so cannot test this as a doctest:
\begin{verbatim}
>>> print(sub_record.features[1])
type: CDS
location: [42:480](+)
qualifiers:
    Key: codon_start, Value: ['1']
    Key: db_xref, Value: ['GI:45478716', 'GeneID:2767712']
    Key: gene, Value: ['pim']
    Key: locus_tag, Value: ['YP_pPCP05']
    Key: note, Value: ['similar to many previously sequenced pesticin immunity ...']
    Key: product, Value: ['pesticin immunity protein']
    Key: protein_id, Value: ['NP_995571.1']
    Key: transl_table, Value: ['11']
    Key: translation, Value: ['MGGGMISKLFCLALIFLSSSGLAEKNTYTAKDILQNLELNTFGNSLSH...']
\end{verbatim}

\noindent Notice that their locations have been adjusted to reflect the new parent sequence!

While Biopython has done something sensible and hopefully intuitive with the features
(and any per-letter annotation), for the other annotation it is impossible to know if
this still applies to the sub-sequence or not. To avoid guessing, the \texttt{annotations}
and \texttt{dbxrefs} are omitted from the sub-record, and it is up to you to transfer
any relevant information as appropriate.

%cont-doctest
\begin{verbatim}
>>> sub_record.annotations
{}
>>> sub_record.dbxrefs
[]
\end{verbatim}

The same point could be made about the record \texttt{id}, \texttt{name}
and \texttt{description}, but for practicality these are preserved:

%cont-doctest
\begin{verbatim}
>>> sub_record.id
'NC_005816.1'
>>> sub_record.name
'NC_005816'
>>> sub_record.description
'Yersinia pestis biovar Microtus str. 91001 plasmid pPCP1, complete sequence'
\end{verbatim}

\noindent This illustrates the problem nicely though, our new sub-record is
\emph{not} the complete sequence of the plasmid, so the description is wrong!
Let's fix this and then view the sub-record as a reduced GenBank file using
the \texttt{format} method described above in Section~\ref{sec:SeqRecord-format}:

\begin{verbatim}
>>> sub_record.description = "Yersinia pestis biovar Microtus str. 91001 plasmid pPCP1, partial."
>>> print(sub_record.format("genbank"))
...
\end{verbatim}

See Sections~\ref{sec:FASTQ-slicing-off-primer}
and~\ref{sec:FASTQ-slicing-off-adaptor} for some FASTQ examples where the
per-letter annotations (the read quality scores) are also sliced.

\section{Adding SeqRecord objects}
\label{sec:SeqRecord-addition}

You can add \verb|SeqRecord| objects together, giving a new \verb|SeqRecord|.
What is important here is that any common
per-letter annotations are also added, all the features are preserved (with their
locations adjusted), and any other common annotation is also kept (like the id, name
and description).

For an example with per-letter annotation, we'll use the first record in a
FASTQ file. Chapter~\ref{chapter:Bio.SeqIO} will explain the \verb|SeqIO| functions:

%doctest ../Tests/Quality
\begin{verbatim}
>>> from Bio import SeqIO
>>> record = next(SeqIO.parse("example.fastq", "fastq"))
>>> len(record)
25
>>> print(record.seq)
CCCTTCTTGTCTTCAGCGTTTCTCC
\end{verbatim}
%TODO - doctest wrapping
\begin{verbatim}
>>> print(record.letter_annotations["phred_quality"])
[26, 26, 18, 26, 26, 26, 26, 26, 26, 26, 26, 26, 26, 26, 26, 22, 26, 26, 26, 26,
26, 26, 26, 23, 23]
\end{verbatim}

\noindent Let's suppose this was Roche 454 data, and that from other information
you think the \texttt{TTT} should be only \texttt{TT}. We can make a new edited
record by first slicing the \verb|SeqRecord| before and after the ``extra''
third \texttt{T}:

%cont-doctest
\begin{verbatim}
>>> left = record[:20]
>>> print(left.seq)
CCCTTCTTGTCTTCAGCGTT
>>> print(left.letter_annotations["phred_quality"])
[26, 26, 18, 26, 26, 26, 26, 26, 26, 26, 26, 26, 26, 26, 26, 22, 26, 26, 26, 26]
>>> right = record[21:]
>>> print(right.seq)
CTCC
>>> print(right.letter_annotations["phred_quality"])
[26, 26, 23, 23]
\end{verbatim}

\noindent Now add the two parts together:

%cont-doctest
\begin{verbatim}
>>> edited = left + right
>>> len(edited)
24
>>> print(edited.seq)
CCCTTCTTGTCTTCAGCGTTCTCC
\end{verbatim}
\begin{verbatim}
>>> print(edited.letter_annotations["phred_quality"])
[26, 26, 18, 26, 26, 26, 26, 26, 26, 26, 26, 26, 26, 26, 26, 22, 26, 26, 26, 26,
26, 26, 23, 23]
\end{verbatim}

\noindent Easy and intuitive? We hope so! You can make this shorter with just:

%cont-doctest
\begin{verbatim}
>>> edited = record[:20] + record[21:]
\end{verbatim}

Now, for an example with features, we'll use a GenBank file.
Suppose you have a circular genome:

%doctest ../Tests/GenBank
\begin{verbatim}
>>> from Bio import SeqIO
>>> record = SeqIO.read("NC_005816.gb", "genbank")
\end{verbatim}
%TODO - doctest wrapping
\begin{verbatim}
>>> record
SeqRecord(seq=Seq('TGTAACGAACGGTGCAATAGTGATCCACACCCAACGCCTGAAATCAGATCCAGG...CTG',
IUPACAmbiguousDNA()), id='NC_005816.1', name='NC_005816',
description='Yersinia pestis biovar Microtus str. 91001 plasmid pPCP1, complete sequence.',
dbxrefs=['Project:10638'])
\end{verbatim}
%cont-doctest
\begin{verbatim}
>>> len(record)
9609
>>> len(record.features)
41
>>> record.dbxrefs
['Project:58037']
\end{verbatim}
%TODO - doctest wrapping
\begin{verbatim}
>>> record.annotations.keys()
['comment', 'sequence_version', 'source', 'taxonomy', 'keywords', 'references',
'accessions', 'data_file_division', 'date', 'organism', 'gi']
\end{verbatim}

You can shift the origin like this:

%cont-doctest
\begin{verbatim}
>>> shifted = record[2000:] + record[:2000]
\end{verbatim}
%TODO - doctest wrapping
\begin{verbatim}
>>> shifted
SeqRecord(seq=Seq('GATACGCAGTCATATTTTTTACACAATTCTCTAATCCCGACAAGGTCGTAGGTC...GGA',
IUPACAmbiguousDNA()), id='NC_005816.1', name='NC_005816',
description='Yersinia pestis biovar Microtus str. 91001 plasmid pPCP1, complete sequence.',
dbxrefs=[])
\end{verbatim}
%cont-doctest
\begin{verbatim}
>>> len(shifted)
9609
\end{verbatim}

Note that this isn't perfect in that some annotation like the database cross references
and one of the features (the source feature) have been lost:

%cont-doctest
\begin{verbatim}
>>> len(shifted.features)
40
>>> shifted.dbxrefs
[]
>>> shifted.annotations.keys()
[]
\end{verbatim}

This is because the \verb|SeqRecord| slicing step is cautious in what annotation
it preserves (erroneously propagating annotation can cause major problems). If
you want to keep the database cross references or the annotations dictionary,
this must be done explicitly:

\begin{verbatim}
>>> shifted.dbxrefs = record.dbxrefs[:]
>>> shifted.annotations = record.annotations.copy()
>>> shifted.dbxrefs
['Project:10638']
>>> shifted.annotations.keys()
['comment', 'sequence_version', 'source', 'taxonomy', 'keywords', 'references',
'accessions', 'data_file_division', 'date', 'organism', 'gi']
\end{verbatim}

Also note that in an example like this, you should probably change the record
identifiers since the NCBI references refer to the \emph{original} unmodified
sequence.

\section{Reverse-complementing SeqRecord objects}
\label{sec:SeqRecord-reverse-complement}

One of the new features in Biopython 1.57 was the \verb|SeqRecord| object's
\verb|reverse_complement| method. This tries to balance easy of use with worries
about what to do with the annotation in the reverse complemented record.

For the sequence, this uses the Seq object's reverse complement method. Any
features are transferred with the location and strand recalculated. Likewise
any per-letter-annotation is also copied but reversed (which makes sense for
typical examples like quality scores). However, transfer of most annotation
is problematical.

For instance, if the record ID was an accession, that accession should not really
apply to the reverse complemented sequence, and transferring the identifier by
default could easily cause subtle data corruption in downstream analysis.
Therefore by default, the \verb|SeqRecord|'s id, name, description, annotations
and database cross references are all \emph{not} transferred by default.

The \verb|SeqRecord| object's \verb|reverse_complement| method takes a number
of optional arguments corresponding to properties of the record. Setting these
arguments to \verb|True| means copy the old values, while \verb|False| means
drop the old values and use the default value. You can alternatively provide
the new desired value instead.

Consider this example record:

%doctest ../Tests/GenBank
\begin{verbatim}
>>> from Bio import SeqIO
>>> record = SeqIO.read("NC_005816.gb", "genbank")
>>> print("%s %i %i %i %i" % (record.id, len(record), len(record.features), len(record.dbxrefs), len(record.annotations)))
NC_005816.1 9609 41 1 13
\end{verbatim}

Here we take the reverse complement and specify a new identifier -- but notice
how most of the annotation is dropped (but not the features):

%cont-doctest
\begin{verbatim}
>>> rc = record.reverse_complement(id="TESTING")
>>> print("%s %i %i %i %i" % (rc.id, len(rc), len(rc.features), len(rc.dbxrefs), len(rc.annotations)))
TESTING 9609 41 0 0
\end{verbatim}



%\chapter{Sequence Input/Output}
%\label{chapter:Bio.SeqIO}
\chapter{Sequence Input/Output}
\label{chapter:Bio.SeqIO}

In this chapter we'll discuss in more detail the \verb|Bio.SeqIO| module, which was briefly introduced in Chapter~\ref{chapter:quick-start} and also used in Chapter~\ref{chapter:SeqRecord}. This aims to provide a simple interface for working with assorted sequence file formats in a uniform way.
See also the \verb|Bio.SeqIO| wiki page (\url{http://biopython.org/wiki/SeqIO}), and the built in documentation (also \href{http://biopython.org/DIST/docs/api/Bio.SeqIO-module.html}{online}):

\begin{verbatim}
>>> from Bio import SeqIO
>>> help(SeqIO)
...
\end{verbatim}

The ``catch'' is that you have to work with \verb|SeqRecord| objects (see Chapter~\ref{chapter:SeqRecord}), which contain a \verb|Seq| object (see Chapter~\ref{chapter:Bio.Seq}) plus annotation like an identifier and description.

\section{Parsing or Reading Sequences}
\label{sec:Bio.SeqIO-input}

The workhorse function \verb|Bio.SeqIO.parse()| is used to read in sequence data as SeqRecord objects.  This function expects two arguments:

\begin{enumerate}
\item The first argument is a {\it handle} to read the data from, or a filename. A handle is typically a file opened for reading, but could be the output from a command line program, or data downloaded from the internet (see Section~\ref{sec:SeqIO_Online}).  See Section~\ref{sec:appendix-handles} for more about handles.
\item The second argument is a lower case string specifying sequence format -- we don't try and guess the file format for you!  See \url{http://biopython.org/wiki/SeqIO} for a full listing of supported formats.
\end{enumerate}

\noindent There is an optional argument \verb|alphabet| to specify the alphabet to be used.  This is useful for file formats like FASTA where otherwise \verb|Bio.SeqIO| will default to a generic alphabet.

The \verb|Bio.SeqIO.parse()| function returns an {\it iterator} which gives \verb|SeqRecord| objects.  Iterators are typically used in a for loop as shown below.

Sometimes you'll find yourself dealing with files which contain only a single record.  For this situation use the function \verb|Bio.SeqIO.read()| which takes the same arguments.  Provided there is one and only one record in the file, this is returned as a \verb|SeqRecord| object.  Otherwise an exception is raised.

\subsection{Reading Sequence Files}

In general \verb|Bio.SeqIO.parse()| is used to read in sequence files as \verb|SeqRecord| objects, and is typically used with a for loop like this:

\begin{verbatim}
from Bio import SeqIO
for seq_record in SeqIO.parse("ls_orchid.fasta", "fasta"):
    print(seq_record.id)
    print(repr(seq_record.seq))
    print(len(seq_record))
\end{verbatim}

The above example is repeated from the introduction in Section~\ref{sec:sequence-parsing}, and will load the orchid DNA sequences in the FASTA format file \href{https://raw.githubusercontent.com/biopython/biopython/master/Doc/examples/ls_orchid.fasta}{ls\_orchid.fasta}.  If instead you wanted to load a GenBank format file like \href{https://raw.githubusercontent.com/biopython/biopython/master/Doc/examples/ls_orchid.gbk}{ls\_orchid.gbk} then all you need to do is change the filename and the format string:

\begin{verbatim}
from Bio import SeqIO
for seq_record in SeqIO.parse("ls_orchid.gbk", "genbank"):
    print(seq_record.id)
    print(repr(seq_record.seq))
    print(len(seq_record))
\end{verbatim}

Similarly, if you wanted to read in a file in another file format, then assuming \verb|Bio.SeqIO.parse()| supports it you would just need to change the format string as appropriate, for example ``swiss'' for SwissProt files or ``embl'' for EMBL text files. There is a full listing on the wiki page (\url{http://biopython.org/wiki/SeqIO}) and in the built in documentation (also \href{http://biopython.org/DIST/docs/api/Bio.SeqIO-module.html}{online}).

Another very common way to use a Python iterator is within a list comprehension (or
a generator expression).  For example, if all you wanted to extract from the file was
a list of the record identifiers we can easily do this with the following list comprehension:

\begin{verbatim}
>>> from Bio import SeqIO
>>> identifiers = [seq_record.id for seq_record in SeqIO.parse("ls_orchid.gbk", "genbank")]
>>> identifiers
['Z78533.1', 'Z78532.1', 'Z78531.1', 'Z78530.1', 'Z78529.1', 'Z78527.1', ..., 'Z78439.1']
\end{verbatim}

\noindent There are more examples using \verb|SeqIO.parse()| in a list
comprehension like this in Section~\ref{seq:sequence-parsing-plus-pylab}
(e.g. for plotting sequence lengths or GC\%).

\subsection{Iterating over the records in a sequence file}

In the above examples, we have usually used a for loop to iterate over all the records one by one.  You can use the for loop with all sorts of Python objects (including lists, tuples and strings) which support the iteration interface.

The object returned by \verb|Bio.SeqIO| is actually an iterator which returns \verb|SeqRecord| objects.  You get to see each record in turn, but once and only once.  The plus point is that an iterator can save you memory when dealing with large files.

Instead of using a for loop, can also use the \verb|next()| function on an iterator to step through the entries, like this:

\begin{verbatim}
from Bio import SeqIO
record_iterator = SeqIO.parse("ls_orchid.fasta", "fasta")

first_record = next(record_iterator)
print(first_record.id)
print(first_record.description)

second_record = next(record_iterator)
print(second_record.id)
print(second_record.description)
\end{verbatim}

Note that if you try to use \verb|next()| and there are no more results, you'll get the special \verb|StopIteration| exception.

One special case to consider is when your sequence files have multiple records, but you only want the first one.  In this situation the following code is very concise:

\begin{verbatim}
from Bio import SeqIO
first_record = next(SeqIO.parse("ls_orchid.gbk", "genbank"))
\end{verbatim}

A word of warning here -- using the \verb|next()| function like this will silently ignore any additional records in the file.
If your files have {\it one and only one} record, like some of the online examples later in this chapter, or a GenBank file for a single chromosome, then use the new \verb|Bio.SeqIO.read()| function instead.
This will check there are no extra unexpected records present.

\subsection{Getting a list of the records in a sequence file}

In the previous section we talked about the fact that \verb|Bio.SeqIO.parse()| gives you a \verb|SeqRecord| iterator, and that you get the records one by one.  Very often you need to be able to access the records in any order. The Python \verb|list| data type is perfect for this, and we can turn the record iterator into a list of \verb|SeqRecord| objects using the built-in Python function \verb|list()| like so:

\begin{verbatim}
from Bio import SeqIO
records = list(SeqIO.parse("ls_orchid.gbk", "genbank"))

print("Found %i records" % len(records))

print("The last record")
last_record = records[-1] #using Python's list tricks
print(last_record.id)
print(repr(last_record.seq))
print(len(last_record))

print("The first record")
first_record = records[0] #remember, Python counts from zero
print(first_record.id)
print(repr(first_record.seq))
print(len(first_record))
\end{verbatim}

\noindent Giving:

\begin{verbatim}
Found 94 records
The last record
Z78439.1
Seq('CATTGTTGAGATCACATAATAATTGATCGAGTTAATCTGGAGGATCTGTTTACT...GCC', IUPACAmbiguousDNA())
592
The first record
Z78533.1
Seq('CGTAACAAGGTTTCCGTAGGTGAACCTGCGGAAGGATCATTGATGAGACCGTGG...CGC', IUPACAmbiguousDNA())
740
\end{verbatim}

You can of course still use a for loop with a list of \verb|SeqRecord| objects.  Using a list is much more flexible than an iterator (for example, you can determine the number of records from the length of the list), but does need more memory because it will hold all the records in memory at once.

\subsection{Extracting data}

The \verb|SeqRecord| object and its annotation structures are described more fully in
Chapter~\ref{chapter:SeqRecord}.  As an example of how annotations are stored, we'll look at the output from parsing the first record in the GenBank file \href{https://raw.githubusercontent.com/biopython/biopython/master/Doc/examples/ls_orchid.gbk}{ls\_orchid.gbk}.

\begin{verbatim}
from Bio import SeqIO
record_iterator = SeqIO.parse("ls_orchid.gbk", "genbank")
first_record = next(record_iterator)
print(first_record)
\end{verbatim}

\noindent That should give something like this:

\begin{verbatim}
ID: Z78533.1
Name: Z78533
Description: C.irapeanum 5.8S rRNA gene and ITS1 and ITS2 DNA.
Number of features: 5
/sequence_version=1
/source=Cypripedium irapeanum
/taxonomy=['Eukaryota', 'Viridiplantae', 'Streptophyta', ..., 'Cypripedium']
/keywords=['5.8S ribosomal RNA', '5.8S rRNA gene', ..., 'ITS1', 'ITS2']
/references=[...]
/accessions=['Z78533']
/data_file_division=PLN
/date=30-NOV-2006
/organism=Cypripedium irapeanum
/gi=2765658
Seq('CGTAACAAGGTTTCCGTAGGTGAACCTGCGGAAGGATCATTGATGAGACCGTGG...CGC', IUPACAmbiguousDNA())
\end{verbatim}

This gives a human readable summary of most of the annotation data for the \verb|SeqRecord|.
For this example we're going to use the \verb|.annotations| attribute which is just a Python dictionary.
The contents of this annotations dictionary were shown when we printed the record above.
You can also print them out directly:
\begin{verbatim}
print(first_record.annotations)
\end{verbatim}
\noindent Like any Python dictionary, you can easily get a list of the keys:
\begin{verbatim}
print(first_record.annotations.keys())
\end{verbatim}
\noindent or values:
\begin{verbatim}
print(first_record.annotations.values())
\end{verbatim}

In general, the annotation values are strings, or lists of strings.  One special case is any references in the file get stored as reference objects.

Suppose you wanted to extract a list of the species from the \href{https://raw.githubusercontent.com/biopython/biopython/master/Doc/examples/ls_orchid.gbk}{ls\_orchid.gbk} GenBank file.  The information we want, \emph{Cypripedium irapeanum}, is held in the annotations dictionary under `source' and `organism', which we can access like this:

\begin{verbatim}
>>> print(first_record.annotations["source"])
Cypripedium irapeanum
\end{verbatim}

\noindent or:

\begin{verbatim}
>>> print(first_record.annotations["organism"])
Cypripedium irapeanum
\end{verbatim}

In general, `organism' is used for the scientific name (in Latin, e.g. \textit{Arabidopsis thaliana}),
while `source' will often be the common name (e.g. thale cress).  In this example, as is often the case,
the two fields are identical.

Now let's go through all the records, building up a list of the species each orchid sequence is from:

\begin{verbatim}
from Bio import SeqIO
all_species = []
for seq_record in SeqIO.parse("ls_orchid.gbk", "genbank"):
    all_species.append(seq_record.annotations["organism"])
print(all_species)
\end{verbatim}

Another way of writing this code is to use a list comprehension:

\begin{verbatim}
from Bio import SeqIO
all_species = [seq_record.annotations["organism"] for seq_record in \
               SeqIO.parse("ls_orchid.gbk", "genbank")]
print(all_species)
\end{verbatim}

\noindent In either case, the result is:

% Try and keep this example output line short enough to fit on one page of PDF output:
\begin{verbatim}
['Cypripedium irapeanum', 'Cypripedium californicum', ..., 'Paphiopedilum barbatum']
\end{verbatim}

Great.  That was pretty easy because GenBank files are annotated in a standardised way.

Now, let's suppose you wanted to extract a list of the species from a FASTA file, rather than the GenBank file.  The bad news is you will have to write some code to extract the data you want from the record's description line - if the information is in the file in the first place!  Our example FASTA format file \href{https://raw.githubusercontent.com/biopython/biopython/master/Doc/examples/ls_orchid.fasta}{ls\_orchid.fasta} starts like this:

\begin{verbatim}
>gi|2765658|emb|Z78533.1|CIZ78533 C.irapeanum 5.8S rRNA gene and ITS1 and ITS2 DNA
CGTAACAAGGTTTCCGTAGGTGAACCTGCGGAAGGATCATTGATGAGACCGTGGAATAAACGATCGAGTG
AATCCGGAGGACCGGTGTACTCAGCTCACCGGGGGCATTGCTCCCGTGGTGACCCTGATTTGTTGTTGGG
...
\end{verbatim}

You can check by hand, but for every record the species name is in the description line as the second word.  This means if we break up each record's \verb|.description| at the spaces, then the species is there as field number one (field zero is the record identifier).  That means we can do this:

\begin{verbatim}
from Bio import SeqIO
all_species = []
for seq_record in SeqIO.parse("ls_orchid.fasta", "fasta"):
    all_species.append(seq_record.description.split()[1])
print(all_species)
\end{verbatim}

\noindent This gives:

\begin{verbatim}
['C.irapeanum', 'C.californicum', 'C.fasciculatum', 'C.margaritaceum', ..., 'P.barbatum']
\end{verbatim}

The concise alternative using list comprehensions would be:

\begin{verbatim}
from Bio import SeqIO
all_species == [seq_record.description.split()[1] for seq_record in \
                SeqIO.parse("ls_orchid.fasta", "fasta")]
print(all_species)
\end{verbatim}

In general, extracting information from the FASTA description line is not very nice.
If you can get your sequences in a well annotated file format like GenBank or EMBL,
then this sort of annotation information is much easier to deal with.

\section{Parsing sequences from compressed files}
\label{sec:SeqIO_compressed}
In the previous section, we looked at parsing sequence data from a file.
Instead of using a filename, you can give \verb|Bio.SeqIO| a handle
(see Section~\ref{sec:appendix-handles}), and in this section
we'll use handles to parse sequence from compressed files.

As you'll have seen above, we can use \verb|Bio.SeqIO.read()| or
\verb|Bio.SeqIO.parse()| with a filename - for instance this quick
example calculates the total length of the sequences in a multiple
record GenBank file using a generator expression:

%doctest examples
\begin{verbatim}
>>> from Bio import SeqIO
>>> print(sum(len(r) for r in SeqIO.parse("ls_orchid.gbk", "gb")))
67518
\end{verbatim}

\noindent
Here we use a file handle instead, using the \verb|with| statement
to close the handle automatically:

%doctest examples
\begin{verbatim}
>>> from Bio import SeqIO
>>> with open("ls_orchid.gbk") as handle:
...     print(sum(len(r) for r in SeqIO.parse(handle, "gb")))
67518
\end{verbatim}

\noindent
Or, the old fashioned way where you manually close the handle:

%doctest examples
\begin{verbatim}
>>> from Bio import SeqIO
>>> handle = open("ls_orchid.gbk")
>>> print(sum(len(r) for r in SeqIO.parse(handle, "gb")))
67518
>>> handle.close()
\end{verbatim}

Now, suppose we have a gzip compressed file instead? These are very
commonly used on Linux. We can use Python's \verb|gzip| module to open
the compressed file for reading - which gives us a handle object:

%doctest examples
\begin{verbatim}
>>> import gzip
>>> from Bio import SeqIO
>>> with gzip.open("ls_orchid.gbk.gz", "rt") as handle:
...     print(sum(len(r) for r in SeqIO.parse(handle, "gb")))
...
67518
\end{verbatim}

Similarly if we had a bzip2 compressed file (sadly the function name isn't
quite as consistent under Python 2):

%TODO: Can we make the doctest code Python version specific?
%doctest examples
\begin{verbatim}
>>> import bz2
>>> from Bio import SeqIO
>>> if hasattr(bz2, "open"):
...     handle = bz2.open("ls_orchid.gbk.bz2", "rt")  # Python 3
... else:
...     handle = bz2.BZ2File("ls_orchid.gbk.bz2", "r")  # Python 2
...
>>> with handle:
...     print(sum(len(r) for r in SeqIO.parse(handle, "gb")))
...
67518
\end{verbatim}

There is a gzip (GNU Zip) variant called BGZF (Blocked GNU Zip Format),
which can be treated like an ordinary gzip file for reading, but has
advantages for random access later which we'll talk about later in
Section~\ref{sec:SeqIO-index-bgzf}.

\section{Parsing sequences from the net}
\label{sec:SeqIO_Online}
In the previous sections, we looked at parsing sequence data from a file
(using a filename or handle), and from compressed files (using a handle).
Here we'll use \verb|Bio.SeqIO| with another type of handle, a network
connection, to download and parse sequences from the internet.

Note that just because you \emph{can} download sequence data and parse it into
a \verb|SeqRecord| object in one go doesn't mean this is a good idea.
In general, you should probably download sequences \emph{once} and save them to
a file for reuse.

\subsection{Parsing GenBank records from the net}
\label{sec:SeqIO_GenBank_Online}
Section~\ref{sec:efetch} talks about the Entrez EFetch interface in more detail,
but for now let's just connect to the NCBI and get a few \textit{Opuntia} (prickly-pear)
sequences from GenBank using their GI numbers.

First of all, let's fetch just one record.  If you don't care about the
annotations and features downloading a FASTA file is a good choice as these
are compact.  Now remember, when you expect the handle to contain one and
only one record, use the \verb|Bio.SeqIO.read()| function:

\begin{verbatim}
from Bio import Entrez
from Bio import SeqIO
Entrez.email = "A.N.Other@example.com"
handle = Entrez.efetch(db="nucleotide", rettype="fasta", retmode="text", id="6273291")
seq_record = SeqIO.read(handle, "fasta")
handle.close()
print("%s with %i features" % (seq_record.id, len(seq_record.features)))
\end{verbatim}

\noindent Expected output:

\begin{verbatim}
gi|6273291|gb|AF191665.1|AF191665 with 0 features
\end{verbatim}

The NCBI will also let you ask for the file in other formats, in particular as
a GenBank file. Until Easter 2009, the Entrez EFetch API let you use ``genbank''
as the return type, however the NCBI now insist on using the official
return types of ``gb'' (or ``gp'' for proteins) as described on
\href{http://www.ncbi.nlm.nih.gov/entrez/query/static/efetchseq_help.html}
{EFetch for Sequence and other Molecular Biology Databases}.
As a result, in Biopython 1.50 onwards, we support ``gb'' as an
alias for ``genbank'' in \verb|Bio.SeqIO|.

\begin{verbatim}
from Bio import Entrez
from Bio import SeqIO
Entrez.email = "A.N.Other@example.com"
handle = Entrez.efetch(db="nucleotide", rettype="gb", retmode="text", id="6273291")
seq_record = SeqIO.read(handle, "gb") #using "gb" as an alias for "genbank"
handle.close()
print("%s with %i features" % (seq_record.id, len(seq_record.features)))
\end{verbatim}

\noindent The expected output of this example is:

\begin{verbatim}
AF191665.1 with 3 features
\end{verbatim}

\noindent Notice this time we have three features.

Now let's fetch several records.  This time the handle contains multiple records,
so we must use the \verb|Bio.SeqIO.parse()| function:

\begin{verbatim}
from Bio import Entrez
from Bio import SeqIO
Entrez.email = "A.N.Other@example.com"
handle = Entrez.efetch(db="nucleotide", rettype="gb", retmode="text",
                       id="6273291,6273290,6273289")
for seq_record in SeqIO.parse(handle, "gb"):
    print("%s %s..." % (seq_record.id, seq_record.description[:50]))
    print("Sequence length %i, %i features, from: %s"
          % (len(seq_record), len(seq_record.features), seq_record.annotations["source"]))
handle.close()
\end{verbatim}

\noindent That should give the following output:

\begin{verbatim}
AF191665.1 Opuntia marenae rpl16 gene; chloroplast gene for c...
Sequence length 902, 3 features, from: chloroplast Opuntia marenae
AF191664.1 Opuntia clavata rpl16 gene; chloroplast gene for c...
Sequence length 899, 3 features, from: chloroplast Grusonia clavata
AF191663.1 Opuntia bradtiana rpl16 gene; chloroplast gene for...
Sequence length 899, 3 features, from: chloroplast Opuntia bradtianaa
\end{verbatim}

See Chapter~\ref{chapter:entrez} for more about the \verb|Bio.Entrez| module, and make sure to read about the NCBI guidelines for using Entrez (Section~\ref{sec:entrez-guidelines}).

\subsection{Parsing SwissProt sequences from the net}
\label{sec:SeqIO_ExPASy_and_SwissProt}
Now let's use a handle to download a SwissProt file from ExPASy,
something covered in more depth in Chapter~\ref{chapter:swiss_prot}.
As mentioned above, when you expect the handle to contain one and only one record,
use the \verb|Bio.SeqIO.read()| function:

\begin{verbatim}
from Bio import ExPASy
from Bio import SeqIO
handle = ExPASy.get_sprot_raw("O23729")
seq_record = SeqIO.read(handle, "swiss")
handle.close()
print(seq_record.id)
print(seq_record.name)
print(seq_record.description)
print(repr(seq_record.seq))
print("Length %i" % len(seq_record))
print(seq_record.annotations["keywords"])
\end{verbatim}

\noindent Assuming your network connection is OK, you should get back:

\begin{verbatim}
O23729
CHS3_BROFI
RecName: Full=Chalcone synthase 3; EC=2.3.1.74; AltName: Full=Naringenin-chalcone synthase 3;
Seq('MAPAMEEIRQAQRAEGPAAVLAIGTSTPPNALYQADYPDYYFRITKSEHLTELK...GAE', ProteinAlphabet())
Length 394
['Acyltransferase', 'Flavonoid biosynthesis', 'Transferase']
\end{verbatim}

\section{Sequence files as Dictionaries}

We're now going to introduce three related functions in the \verb|Bio.SeqIO|
module which allow dictionary like random access to a multi-sequence file.
There is a trade off here between flexibility and memory usage. In summary:
\begin{itemize}
\item \verb|Bio.SeqIO.to_dict()| is the most flexible but also the most
memory demanding option (see Section~\ref{SeqIO:to_dict}). This is basically
a helper function to build a normal Python \verb|dictionary| with each entry
held as a \verb|SeqRecord| object in memory, allowing you to modify the
records.
\item \verb|Bio.SeqIO.index()| is a useful middle ground, acting like a
read only dictionary and parsing sequences into \verb|SeqRecord| objects
on demand (see Section~\ref{sec:SeqIO-index}).
\item \verb|Bio.SeqIO.index_db()| also acts like a read only dictionary
but stores the identifiers and file offsets in a file on disk (as an
SQLite3 database), meaning it has very low memory requirements (see
Section~\ref{sec:SeqIO-index-db}), but will be a little bit slower.
\end{itemize}
See the discussion for an broad overview
(Section~\ref{sec:SeqIO-indexing-discussion}).

\subsection{Sequence files as Dictionaries -- In memory}
\label{SeqIO:to_dict}

The next thing that we'll do with our ubiquitous orchid files is to show how
to index them and access them like a database using the Python \verb|dictionary|
data type (like a hash in Perl). This is very useful for moderately large files
where you only need to access certain elements of the file, and makes for a nice
quick 'n dirty database. For dealing with larger files where memory becomes a
problem, see Section~\ref{sec:SeqIO-index} below.

You can use the function \verb|Bio.SeqIO.to_dict()| to make a SeqRecord dictionary
(in memory).  By default this will use each record's identifier (i.e. the \verb|.id|
attribute) as the key.  Let's try this using our GenBank file:

%doctest examples
\begin{verbatim}
>>> from Bio import SeqIO
>>> orchid_dict = SeqIO.to_dict(SeqIO.parse("ls_orchid.gbk", "genbank"))
\end{verbatim}

There is just one required argument for \verb|Bio.SeqIO.to_dict()|, a list or
generator giving \verb|SeqRecord| objects. Here we have just used the output
from the \verb|SeqIO.parse| function. As the name suggests, this returns a
Python dictionary.

Since this variable \verb|orchid_dict| is an ordinary Python dictionary,
we can look at all of the keys we have available:

%cont-doctest
\begin{verbatim}
>>> len(orchid_dict)
94
\end{verbatim}
%Can't use following for doctest due to abbreviation
\begin{verbatim}
>>> list(orchid_dict.keys())
['Z78484.1', 'Z78464.1', 'Z78455.1', 'Z78442.1', 'Z78532.1', 'Z78453.1', ..., 'Z78471.1']
\end{verbatim}

You can leave out the ``list(...)`` bit if you are still using Python 2.
Under Python 3 the dictionary methods like ``.keys()`` and ``.values()``
are iterators rather than lists.

If you really want to, you can even look at all the records at once:
\begin{verbatim}
>>> list(orchid_dict.values()) #lots of output!
...
\end{verbatim}

We can access a single \verb|SeqRecord| object via the keys and manipulate the object as normal:

%cont-doctest
\begin{verbatim}
>>> seq_record = orchid_dict["Z78475.1"]
>>> print(seq_record.description)
P.supardii 5.8S rRNA gene and ITS1 and ITS2 DNA
>>> print(repr(seq_record.seq))
Seq('CGTAACAAGGTTTCCGTAGGTGAACCTGCGGAAGGATCATTGTTGAGATCACAT...GGT', IUPACAmbiguousDNA())
\end{verbatim}

So, it is very easy to create an in memory ``database'' of our GenBank records.  Next we'll try this for the FASTA file instead.

Note that those of you with prior Python experience should all be able to construct a dictionary like this ``by hand''.  However, typical dictionary construction methods will not deal with the case of repeated keys very nicely.  Using the \verb|Bio.SeqIO.to_dict()| will explicitly check for duplicate keys, and raise an exception if any are found.

\subsubsection{Specifying the dictionary keys}
\label{seq:seqio-todict-functionkey}

Using the same code as above, but for the FASTA file instead:

\begin{verbatim}
from Bio import SeqIO
orchid_dict = SeqIO.to_dict(SeqIO.parse("ls_orchid.fasta", "fasta"))
print(orchid_dict.keys())
\end{verbatim}

\noindent This time the keys are:

\begin{verbatim}
['gi|2765596|emb|Z78471.1|PDZ78471', 'gi|2765646|emb|Z78521.1|CCZ78521', ...
 ..., 'gi|2765613|emb|Z78488.1|PTZ78488', 'gi|2765583|emb|Z78458.1|PHZ78458']
\end{verbatim}

You should recognise these strings from when we parsed the FASTA file earlier in Section~\ref{sec:fasta-parsing}.  Suppose you would rather have something else as the keys - like the accession numbers.  This brings us nicely to \verb|SeqIO.to_dict()|'s optional argument \verb|key_function|, which lets you define what to use as the dictionary key for your records.

First you must write your own function to return the key you want (as a string) when given a \verb|SeqRecord| object.  In general, the details of function will depend on the sort of input records you are dealing with.  But for our orchids, we can just split up the record's identifier using the ``pipe'' character (the vertical line) and return the fourth entry (field three):

\begin{verbatim}
def get_accession(record):
    """"Given a SeqRecord, return the accession number as a string.

    e.g. "gi|2765613|emb|Z78488.1|PTZ78488" -> "Z78488.1"
    """
    parts = record.id.split("|")
    assert len(parts) == 5 and parts[0] == "gi" and parts[2] == "emb"
    return parts[3]
\end{verbatim}

\noindent Then we can give this function to the \verb|SeqIO.to_dict()| function to use in building the dictionary:

\begin{verbatim}
from Bio import SeqIO
orchid_dict = SeqIO.to_dict(SeqIO.parse("ls_orchid.fasta", "fasta"), key_function=get_accession)
print(orchid_dict.keys())
\end{verbatim}

\noindent Finally, as desired, the new dictionary keys:

\begin{verbatim}
>>> print(orchid_dict.keys())
['Z78484.1', 'Z78464.1', 'Z78455.1', 'Z78442.1', 'Z78532.1', 'Z78453.1', ..., 'Z78471.1']
\end{verbatim}

\noindent Not too complicated, I hope!

\subsubsection{Indexing a dictionary using the SEGUID checksum}

To give another example of working with dictionaries of \verb|SeqRecord| objects, we'll use the SEGUID checksum function.  This is a relatively recent checksum, and collisions should be very rare (i.e. two different sequences with the same checksum), an improvement on the CRC64 checksum.

Once again, working with the orchids GenBank file:

\begin{verbatim}
from Bio import SeqIO
from Bio.SeqUtils.CheckSum import seguid
for record in SeqIO.parse("ls_orchid.gbk", "genbank"):
    print(record.id, seguid(record.seq))
\end{verbatim}

\noindent This should give:

\begin{verbatim}
Z78533.1 JUEoWn6DPhgZ9nAyowsgtoD9TTo
Z78532.1 MN/s0q9zDoCVEEc+k/IFwCNF2pY
...
Z78439.1 H+JfaShya/4yyAj7IbMqgNkxdxQ
\end{verbatim}

Now, recall the \verb|Bio.SeqIO.to_dict()| function's \verb|key_function| argument expects a function which turns a \verb|SeqRecord| into a string.  We can't use the \verb|seguid()| function directly because it expects to be given a \verb|Seq| object (or a string).  However, we can use Python's \verb|lambda| feature to create a ``one off'' function to give to \verb|Bio.SeqIO.to_dict()| instead:

%doctest examples
\begin{verbatim}
>>> from Bio import SeqIO
>>> from Bio.SeqUtils.CheckSum import seguid
>>> seguid_dict = SeqIO.to_dict(SeqIO.parse("ls_orchid.gbk", "genbank"),
...                             lambda rec : seguid(rec.seq))
>>> record = seguid_dict["MN/s0q9zDoCVEEc+k/IFwCNF2pY"]
>>> print(record.id)
Z78532.1
>>> print(record.description)
C.californicum 5.8S rRNA gene and ITS1 and ITS2 DNA
\end{verbatim}

\noindent That should have retrieved the record {\tt Z78532.1}, the second entry in the file.

\subsection{Sequence files as Dictionaries -- Indexed files}
%\subsection{Indexing really large files}
\label{sec:SeqIO-index}

As the previous couple of examples tried to illustrate, using
\verb|Bio.SeqIO.to_dict()| is very flexible. However, because it holds
everything in memory, the size of file you can work with is limited by your
computer's RAM. In general, this will only work on small to medium files.

For larger files you should consider
\verb|Bio.SeqIO.index()|, which works a little differently. Although
it still returns a dictionary like object, this does \emph{not} keep
\emph{everything} in memory. Instead, it just records where each record
is within the file -- when you ask for a particular record, it then parses
it on demand.

As an example, let's use the same GenBank file as before:

%doctest examples
\begin{verbatim}
>>> from Bio import SeqIO
>>> orchid_dict = SeqIO.index("ls_orchid.gbk", "genbank")
>>> len(orchid_dict)
94
\end{verbatim}
%Following is abbr.
\begin{verbatim}
>>> orchid_dict.keys()
['Z78484.1', 'Z78464.1', 'Z78455.1', 'Z78442.1', 'Z78532.1', 'Z78453.1', ..., 'Z78471.1']
\end{verbatim}
%cont-doctest
\begin{verbatim}
>>> seq_record = orchid_dict["Z78475.1"]
>>> print(seq_record.description)
P.supardii 5.8S rRNA gene and ITS1 and ITS2 DNA
>>> seq_record.seq
Seq('CGTAACAAGGTTTCCGTAGGTGAACCTGCGGAAGGATCATTGTTGAGATCACAT...GGT', IUPACAmbiguousDNA())
>>> orchid_dict.close()
\end{verbatim}

\noindent Note that \verb|Bio.SeqIO.index()| won't take a handle,
but only a filename. There are good reasons for this, but it is a little
technical. The second argument is the file format (a lower case string as
used in the other \verb|Bio.SeqIO| functions). You can use many other
simple file formats, including FASTA and FASTQ files (see the example in
Section~\ref{sec:fastq-indexing}). However, alignment
formats like PHYLIP or Clustal are not supported. Finally as an optional
argument you can supply an alphabet, or a key function.

Here is the same example using the FASTA file - all we change is the
filename and the format name:

\begin{verbatim}
>>> from Bio import SeqIO
>>> orchid_dict = SeqIO.index("ls_orchid.fasta", "fasta")
>>> len(orchid_dict)
94
>>> orchid_dict.keys()
['gi|2765596|emb|Z78471.1|PDZ78471', 'gi|2765646|emb|Z78521.1|CCZ78521', ...
 ..., 'gi|2765613|emb|Z78488.1|PTZ78488', 'gi|2765583|emb|Z78458.1|PHZ78458']
\end{verbatim}

\subsubsection{Specifying the dictionary keys}
\label{seq:seqio-index-functionkey}

Suppose you want to use the same keys as before? Much like with the
\verb|Bio.SeqIO.to_dict()| example in Section~\ref{seq:seqio-todict-functionkey},
you'll need to write a tiny function to map from the FASTA identifier
(as a string) to the key you want:

\begin{verbatim}
def get_acc(identifier):
    """"Given a SeqRecord identifier string, return the accession number as a string.

    e.g. "gi|2765613|emb|Z78488.1|PTZ78488" -> "Z78488.1"
    """
    parts = identifier.split("|")
    assert len(parts) == 5 and parts[0] == "gi" and parts[2] == "emb"
    return parts[3]
\end{verbatim}

\noindent Then we can give this function to the \verb|Bio.SeqIO.index()|
function to use in building the dictionary:

\begin{verbatim}
>>> from Bio import SeqIO
>>> orchid_dict = SeqIO.index("ls_orchid.fasta", "fasta", key_function=get_acc)
>>> print(orchid_dict.keys())
['Z78484.1', 'Z78464.1', 'Z78455.1', 'Z78442.1', 'Z78532.1', 'Z78453.1', ..., 'Z78471.1']
\end{verbatim}

\noindent Easy when you know how?

\subsubsection{Getting the raw data for a record}
\label{sec:seqio-index-getraw}

The dictionary-like object from \verb|Bio.SeqIO.index()| gives you each
entry as a \verb|SeqRecord| object. However, it is sometimes useful to
be able to get the original raw data straight from the file. For this
use the \verb|get_raw()| method which takes a
single argument (the record identifier) and returns a bytes string
(extracted from the file without modification).

A motivating example is extracting a subset of a records from a large
file where either \verb|Bio.SeqIO.write()| does not (yet) support the
output file format (e.g. the plain text SwissProt file format) or
where you need to preserve the text exactly (e.g. GenBank or EMBL
output from Biopython does not yet preserve every last bit of
annotation).

Let's suppose you have download the whole of UniProt in the plain
text SwissPort file format from their FTP site
(\url{ftp://ftp.uniprot.org/pub/databases/uniprot/current_release/knowledgebase/complete/uniprot_sprot.dat.gz})
and uncompressed it as the file \verb|uniprot_sprot.dat|, and you
want to extract just a few records from it:

\begin{verbatim}
>>> from Bio import SeqIO
>>> uniprot = SeqIO.index("uniprot_sprot.dat", "swiss")
>>> handle = open("selected.dat", "wb")
>>> for acc in ["P33487", "P19801", "P13689", "Q8JZQ5", "Q9TRC7"]:
...     handle.write(uniprot.get_raw(acc))
>>> handle.close()
\end{verbatim}

Note with Python 3 onwards, we have to open the file for writing in
binary mode because the \verb|get_raw()| method returns bytes strings.

There is a longer example in Section~\ref{sec:SeqIO-sort} using the
\verb|SeqIO.index()| function to sort a large sequence file (without
loading everything into memory at once).

\subsection{Sequence files as Dictionaries -- Database indexed files}
\label{sec:SeqIO-index-db}

Biopython 1.57 introduced an alternative, \verb|Bio.SeqIO.index_db()|, which
can work on even extremely large files since it stores the record information
as a file on disk (using an SQLite3 database) rather than in memory. Also,
you can index multiple files together (providing all the record identifiers
are unique).

The \verb|Bio.SeqIO.index()| function takes three required arguments:
\begin{itemize}
\item Index filename, we suggest using something ending \texttt{.idx}.
This index file is actually an SQLite3 database.
\item List of sequence filenames to index (or a single filename)
\item File format (lower case string as used in the rest of the
\verb|SeqIO| module).
\end{itemize}

As an example, consider the GenBank flat file releases from the NCBI FTP site,
\url{ftp://ftp.ncbi.nih.gov/genbank/}, which are gzip compressed GenBank files.

As of GenBank release $210$, there are $38$ files making up the viral sequences,
\texttt{gbvrl1.seq}, \ldots, \texttt{gbvrl38.seq}, taking about 8GB on disk once
decompressed, and containing in total nearly two million records.

If you were interested in the viruses, you could download all the virus files
from the command line very easily with the \texttt{rsync} command, and then
decompress them with \texttt{gunzip}:

\begin{verbatim}
# For illustration only, see reduced example below
$ rsync -avP "ftp.ncbi.nih.gov::genbank/gbvrl*.seq.gz" .
$ gunzip gbvrl*.seq.gz
\end{verbatim}

Unless you care about viruses, that's a lot of data to download just for this
example - so let's download \emph{just} the first four chunks (about 25MB each
compressed), and decompress them (taking in all about 1GB of space):

\begin{verbatim}
# Reduced example, download only the first four chunks
$ curl -O ftp://ftp.ncbi.nih.gov/genbank/gbvrl1.seq.gz
$ curl -O ftp://ftp.ncbi.nih.gov/genbank/gbvrl2.seq.gz
$ curl -O ftp://ftp.ncbi.nih.gov/genbank/gbvrl3.seq.gz
$ curl -O ftp://ftp.ncbi.nih.gov/genbank/gbvrl4.seq.gz
$ gunzip gbvrl*.seq.gz
\end{verbatim}

Now, in Python, index these GenBank files as follows:

\begin{verbatim}
>>> import glob
>>> from Bio import SeqIO
>>> files = glob.glob("gbvrl*.seq")
>>> print("%i files to index" % len(files))
4
>>> gb_vrl = SeqIO.index_db("gbvrl.idx", files, "genbank")
>>> print("%i sequences indexed" % len(gb_vrl))
272960 sequences indexed
\end{verbatim}

Indexing the full set of virus GenBank files took about ten minutes on my machine,
just the first four files took about a minute or so.

However, once done, repeating this will reload the index file \verb|gbvrl.idx|
in a fraction of a second.

You can use the index as a read only Python dictionary - without having to worry
about which file the sequence comes from, e.g.

\begin{verbatim}
>>> print(gb_vrl[``AB811634.1''].description)
Equine encephalosis virus NS3 gene, complete cds, isolate: Kimron1.
\end{verbatim}

\subsubsection{Getting the raw data for a record}

Just as with the \verb|Bio.SeqIO.index()| function discussed above in
Section~\ref{sec:seqio-index-getraw}, the dictionary like object also lets you
get at the raw bytes of each record:

% TODO - Under Python 3 you'd get the bytes string representation with
% leading b single-quote, escaped newlines, and closing single-quote:
%
% >>> print(gb_vrl.get_raw(``GQ333173.1''))
% b'LOCUS       GQ333173                 459 bp    DNA     linear   VRL 21-OCT-2009\nDEFINITION...'
\begin{verbatim}
>>> print(gb_vrl.get_raw(``AB811634.1''))
LOCUS       AB811634                 723 bp    RNA     linear   VRL 17-JUN-2015
DEFINITION  Equine encephalosis virus NS3 gene, complete cds, isolate: Kimron1.
ACCESSION   AB811634
...
//
\end{verbatim}

\subsection{Indexing compressed files}
\label{sec:SeqIO-index-bgzf}

Very often when you are indexing a sequence file it can be quite large -- so
you may want to compress it on disk. Unfortunately efficient random access
is difficult with the more common file formats like gzip and bzip2. In this
setting, BGZF (Blocked GNU Zip Format) can be very helpful. This is a variant
of gzip (and can be decompressed using standard gzip tools) popularised by
the BAM file format, \href{http://samtools.sourceforge.net/}{samtools}, and
\href{http://samtools.sourceforge.net/tabix.shtml}{tabix}.

To create a BGZF compressed file you can use the command line tool \verb|bgzip|
which comes with samtools. In our examples we use a filename extension
\verb|*.bgz|, so they can be distinguished from normal gzipped files (named
\verb|*.gz|). You can also use the \verb|Bio.bgzf| module to read and write
BGZF files from within Python.

The \verb|Bio.SeqIO.index()| and \verb|Bio.SeqIO.index_db()| can both be
used with BGZF compressed files. For example, if you started with an
uncompressed GenBank file:

%doctest examples
\begin{verbatim}
>>> from Bio import SeqIO
>>> orchid_dict = SeqIO.index("ls_orchid.gbk", "genbank")
>>> len(orchid_dict)
94
>>> orchid_dict.close()
\end{verbatim}

You could compress this (while keeping the original file) at the command
line using the following command -- but don't worry, the compressed file
is already included with the other example files:

\begin{verbatim}
$ bgzip -c ls_orchid.gbk > ls_orchid.gbk.bgz
\end{verbatim}

You can use the compressed file in exactly the same way:

%doctest examples
\begin{verbatim}
>>> from Bio import SeqIO
>>> orchid_dict = SeqIO.index("ls_orchid.gbk.bgz", "genbank")
>>> len(orchid_dict)
94
>>> orchid_dict.close()
\end{verbatim}

\noindent
or:

%Don't use doctest as would have to clean up the *.idx file
\begin{verbatim}
>>> from Bio import SeqIO
>>> orchid_dict = SeqIO.index_db("ls_orchid.gbk.bgz.idx", "ls_orchid.gbk.bgz", "genbank")
>>> len(orchid_dict)
94
>>> orchid_dict.close()
\end{verbatim}

The \verb|SeqIO| indexing automatically detects the BGZF compression. Note
that you can't use the same index file for the uncompressed and compressed files.

\subsection{Discussion}
\label{sec:SeqIO-indexing-discussion}

So, which of these methods should you use and why? It depends on what you are
trying to do (and how much data you are dealing with). However, in general
picking \verb|Bio.SeqIO.index()| is a good starting point. If you are dealing
with millions of records, multiple files, or repeated analyses, then look at
\verb|Bio.SeqIO.index_db()|.

Reasons to choose \verb|Bio.SeqIO.to_dict()| over either
\verb|Bio.SeqIO.index()| or \verb|Bio.SeqIO.index_db()| boil down to a need
for flexibility despite its high memory needs. The advantage of storing the
\verb|SeqRecord| objects in memory is they can be changed, added to, or
removed at will. In addition to the downside of high memory consumption,
indexing can also take longer because all the records must be fully parsed.

Both \verb|Bio.SeqIO.index()| and \verb|Bio.SeqIO.index_db()| only parse
records on demand. When indexing, they scan the file once looking for the
start of each record and do as little work as possible to extract the
identifier.

Reasons to choose \verb|Bio.SeqIO.index()| over \verb|Bio.SeqIO.index_db()|
include:
\begin{itemize}
\item Faster to build the index (more noticeable in simple file formats)
\item Slightly faster access as SeqRecord objects (but the difference is only
really noticeable for simple to parse file formats).
\item Can use any immutable Python object as the dictionary keys (e.g. a
tuple of strings, or a frozen set) not just strings.
\item Don't need to worry about the index database being out of date if the
sequence file being indexed has changed.
\end{itemize}

Reasons to choose \verb|Bio.SeqIO.index_db()| over \verb|Bio.SeqIO.index()|
include:
\begin{itemize}
\item Not memory limited -- this is already important with files from second
generation sequencing where 10s of millions of sequences are common, and
using \verb|Bio.SeqIO.index()| can require more than 4GB of RAM and therefore
a 64bit version of Python.
\item Because the index is kept on disk, it can be reused. Although building
the index database file takes longer, if you have a script which will be
rerun on the same datafiles in future, this could save time in the long run.
\item Indexing multiple files together
\item The \verb|get_raw()| method can be much faster, since for most file
formats the length of each record is stored as well as its offset.
\end{itemize}

\section{Writing Sequence Files}

We've talked about using \verb|Bio.SeqIO.parse()| for sequence input (reading files), and now we'll look at \verb|Bio.SeqIO.write()| which is for sequence output (writing files).  This is a function taking three arguments: some \verb|SeqRecord| objects, a handle or filename to write to, and a sequence format.

Here is an example, where we start by creating a few \verb|SeqRecord| objects the hard way (by hand, rather than by loading them from a file):

\begin{verbatim}
from Bio.Seq import Seq
from Bio.SeqRecord import SeqRecord
from Bio.Alphabet import generic_protein

rec1 = SeqRecord(Seq("MMYQQGCFAGGTVLRLAKDLAENNRGARVLVVCSEITAVTFRGPSETHLDSMVGQALFGD" \
                    +"GAGAVIVGSDPDLSVERPLYELVWTGATLLPDSEGAIDGHLREVGLTFHLLKDVPGLISK" \
                    +"NIEKSLKEAFTPLGISDWNSTFWIAHPGGPAILDQVEAKLGLKEEKMRATREVLSEYGNM" \
                    +"SSAC", generic_protein),
                 id="gi|14150838|gb|AAK54648.1|AF376133_1",
                 description="chalcone synthase [Cucumis sativus]")

rec2 = SeqRecord(Seq("YPDYYFRITNREHKAELKEKFQRMCDKSMIKKRYMYLTEEILKENPSMCEYMAPSLDARQ" \
                    +"DMVVVEIPKLGKEAAVKAIKEWGQ", generic_protein),
                 id="gi|13919613|gb|AAK33142.1|",
                 description="chalcone synthase [Fragaria vesca subsp. bracteata]")

rec3 = SeqRecord(Seq("MVTVEEFRRAQCAEGPATVMAIGTATPSNCVDQSTYPDYYFRITNSEHKVELKEKFKRMC" \
                    +"EKSMIKKRYMHLTEEILKENPNICAYMAPSLDARQDIVVVEVPKLGKEAAQKAIKEWGQP" \
                    +"KSKITHLVFCTTSGVDMPGCDYQLTKLLGLRPSVKRFMMYQQGCFAGGTVLRMAKDLAEN" \
                    +"NKGARVLVVCSEITAVTFRGPNDTHLDSLVGQALFGDGAAAVIIGSDPIPEVERPLFELV" \
                    +"SAAQTLLPDSEGAIDGHLREVGLTFHLLKDVPGLISKNIEKSLVEAFQPLGISDWNSLFW" \
                    +"IAHPGGPAILDQVELKLGLKQEKLKATRKVLSNYGNMSSACVLFILDEMRKASAKEGLGT" \
                    +"TGEGLEWGVLFGFGPGLTVETVVLHSVAT", generic_protein),
                 id="gi|13925890|gb|AAK49457.1|",
                 description="chalcone synthase [Nicotiana tabacum]")

my_records = [rec1, rec2, rec3]
\end{verbatim}

\noindent Now we have a list of \verb|SeqRecord| objects, we'll write them to a FASTA format file:

\begin{verbatim}
from Bio import SeqIO
SeqIO.write(my_records, "my_example.faa", "fasta")
\end{verbatim}

\noindent And if you open this file in your favourite text editor it should look like this:

\begin{verbatim}
>gi|14150838|gb|AAK54648.1|AF376133_1 chalcone synthase [Cucumis sativus]
MMYQQGCFAGGTVLRLAKDLAENNRGARVLVVCSEITAVTFRGPSETHLDSMVGQALFGD
GAGAVIVGSDPDLSVERPLYELVWTGATLLPDSEGAIDGHLREVGLTFHLLKDVPGLISK
NIEKSLKEAFTPLGISDWNSTFWIAHPGGPAILDQVEAKLGLKEEKMRATREVLSEYGNM
SSAC
>gi|13919613|gb|AAK33142.1| chalcone synthase [Fragaria vesca subsp. bracteata]
YPDYYFRITNREHKAELKEKFQRMCDKSMIKKRYMYLTEEILKENPSMCEYMAPSLDARQ
DMVVVEIPKLGKEAAVKAIKEWGQ
>gi|13925890|gb|AAK49457.1| chalcone synthase [Nicotiana tabacum]
MVTVEEFRRAQCAEGPATVMAIGTATPSNCVDQSTYPDYYFRITNSEHKVELKEKFKRMC
EKSMIKKRYMHLTEEILKENPNICAYMAPSLDARQDIVVVEVPKLGKEAAQKAIKEWGQP
KSKITHLVFCTTSGVDMPGCDYQLTKLLGLRPSVKRFMMYQQGCFAGGTVLRMAKDLAEN
NKGARVLVVCSEITAVTFRGPNDTHLDSLVGQALFGDGAAAVIIGSDPIPEVERPLFELV
SAAQTLLPDSEGAIDGHLREVGLTFHLLKDVPGLISKNIEKSLVEAFQPLGISDWNSLFW
IAHPGGPAILDQVELKLGLKQEKLKATRKVLSNYGNMSSACVLFILDEMRKASAKEGLGT
TGEGLEWGVLFGFGPGLTVETVVLHSVAT
\end{verbatim}

Suppose you wanted to know how many records the \verb|Bio.SeqIO.write()| function wrote to the handle?
If your records were in a list you could just use \verb|len(my_records)|, however you can't do that when your records come from a generator/iterator.  The \verb|Bio.SeqIO.write()| function returns the number of \verb|SeqRecord| objects written to the file.

\emph{Note} - If you tell the \verb|Bio.SeqIO.write()| function to write to a file that already exists, the old file will be overwritten without any warning.

\subsection{Round trips}

Some people like their parsers to be ``round-tripable'', meaning if you read in
a file and write it back out again it is unchanged. This requires that the parser
must extract enough information to reproduce the original file \emph{exactly}.
\verb|Bio.SeqIO| does \emph{not} aim to do this.

As a trivial example, any line wrapping of the sequence data in FASTA files is
allowed. An identical \verb|SeqRecord| would be given from parsing the following
two examples which differ only in their line breaks:

\begin{verbatim}
>YAL068C-7235.2170 Putative promoter sequence
TACGAGAATAATTTCTCATCATCCAGCTTTAACACAAAATTCGCACAGTTTTCGTTAAGA
GAACTTAACATTTTCTTATGACGTAAATGAAGTTTATATATAAATTTCCTTTTTATTGGA

>YAL068C-7235.2170 Putative promoter sequence
TACGAGAATAATTTCTCATCATCCAGCTTTAACACAAAATTCGCA
CAGTTTTCGTTAAGAGAACTTAACATTTTCTTATGACGTAAATGA
AGTTTATATATAAATTTCCTTTTTATTGGA
\end{verbatim}

To make a round-tripable FASTA parser you would need to keep track of where the
sequence line breaks occurred, and this extra information is usually pointless.
Instead Biopython uses a default line wrapping of $60$ characters on output.
The same problem with white space applies in many other file formats too.
Another issue in some cases is that Biopython does not (yet) preserve every
last bit of annotation (e.g. GenBank and EMBL).

Occasionally preserving the original layout (with any quirks it may have) is
important. See Section~\ref{sec:seqio-index-getraw} about the \verb|get_raw()|
method of the \verb|Bio.SeqIO.index()| dictionary-like object for one potential
solution.

\subsection{Converting between sequence file formats}
\label{sec:SeqIO-conversion}

In previous example we used a list of \verb|SeqRecord| objects as input to the \verb|Bio.SeqIO.write()| function, but it will also accept a \verb|SeqRecord| iterator like we get from \verb|Bio.SeqIO.parse()| -- this lets us do file conversion by combining these two functions.

For this example we'll read in the GenBank format file \href{https://raw.githubusercontent.com/biopython/biopython/master/Doc/examples/ls_orchid.gbk}{ls\_orchid.gbk} and write it out in FASTA format:

\begin{verbatim}
from Bio import SeqIO
records = SeqIO.parse("ls_orchid.gbk", "genbank")
count = SeqIO.write(records, "my_example.fasta", "fasta")
print("Converted %i records" % count)
\end{verbatim}

Still, that is a little bit complicated. So, because file conversion is such a
common task, there is a helper function letting you replace that with just:

\begin{verbatim}
from Bio import SeqIO
count = SeqIO.convert("ls_orchid.gbk", "genbank", "my_example.fasta", "fasta")
print("Converted %i records" % count)
\end{verbatim}

The \verb|Bio.SeqIO.convert()| function will take handles \emph{or} filenames.
Watch out though -- if the output file already exists, it will overwrite it!
To find out more, see the built in help:

\begin{verbatim}
>>> from Bio import SeqIO
>>> help(SeqIO.convert)
...
\end{verbatim}

In principle, just by changing the filenames and the format names, this code
could be used to convert between any file formats available in Biopython.
However, writing some formats requires information (e.g. quality scores) which
other files formats don't contain. For example, while you can turn a FASTQ
file into a FASTA file, you can't do the reverse. See also
Sections~\ref{sec:SeqIO-fastq-conversion} and~\ref{sec:SeqIO-fasta-qual-conversion}
in the cookbook chapter which looks at inter-converting between different FASTQ formats.

Finally, as an added incentive for using the \verb|Bio.SeqIO.convert()| function
(on top of the fact your code will be shorter), doing it this way may also be
faster! The reason for this is the convert function can take advantage of
several file format specific optimisations and tricks.

\subsection{Converting a file of sequences to their reverse complements}
\label{sec:SeqIO-reverse-complement}

Suppose you had a file of nucleotide sequences, and you wanted to turn it into a file containing their reverse complement sequences.  This time a little bit of work is required to transform the \verb|SeqRecord| objects we get from our input file into something suitable for saving to our output file.

To start with, we'll use \verb|Bio.SeqIO.parse()| to load some nucleotide
sequences from a file, then print out their reverse complements using
the \verb|Seq| object's built in \verb|.reverse_complement()| method (see Section~\ref{sec:seq-reverse-complement}):

\begin{verbatim}
>>> from Bio import SeqIO
>>> for record in SeqIO.parse("ls_orchid.gbk", "genbank"):
...     print(record.id)
...     print(record.seq.reverse_complement())
\end{verbatim}

Now, if we want to save these reverse complements to a file, we'll need to make \verb|SeqRecord| objects.
We can use  the \verb|SeqRecord| object's built in \verb|.reverse_complement()| method (see Section~\ref{sec:SeqRecord-reverse-complement}) but we must decide how to name our new records.

This is an excellent place to demonstrate the power of list comprehensions which make a list in memory:
%doctest examples
\begin{verbatim}
>>> from Bio import SeqIO
>>> records = [rec.reverse_complement(id="rc_"+rec.id, description = "reverse complement") \
...            for rec in SeqIO.parse("ls_orchid.fasta", "fasta")]
>>> len(records)
94
\end{verbatim}

\noindent Now list comprehensions have a nice trick up their sleeves, you can add a conditional statement:

%cont-doctest examples
\begin{verbatim}
>>> records = [rec.reverse_complement(id="rc_"+rec.id, description = "reverse complement") \
...            for rec in SeqIO.parse("ls_orchid.fasta", "fasta") if len(rec)<700]
>>> len(records)
18
\end{verbatim}

That would create an in memory list of reverse complement records where the sequence length was under 700 base pairs. However, we can do exactly the same with a generator expression - but with the advantage that this does not create a list of all the records in memory at once:

%cont-doctest examples
\begin{verbatim}
>>> records = (rec.reverse_complement(id="rc_"+rec.id, description = "reverse complement") \
...           for rec in SeqIO.parse("ls_orchid.fasta", "fasta") if len(rec)<700)
\end{verbatim}

As a complete example:

%not a doctest as would have to remove the output file
\begin{verbatim}
>>> from Bio import SeqIO
>>> records = (rec.reverse_complement(id="rc_"+rec.id, description = "reverse complement") \
...            for rec in SeqIO.parse("ls_orchid.fasta", "fasta") if len(rec)<700)
>>> SeqIO.write(records, "rev_comp.fasta", "fasta")
18
\end{verbatim}

There is a related example in Section~\ref{sec:SeqIO-translate}, translating each
record in a FASTA file from nucleotides to amino acids.

\subsection{Getting your SeqRecord objects as formatted strings}
\label{sec:Bio.SeqIO-and-StringIO}
Suppose that you don't really want to write your records to a file or handle -- instead you want a string containing the records in a particular file format.  The \verb|Bio.SeqIO| interface is based on handles, but Python has a useful built in module which provides a string based handle.

For an example of how you might use this, let's load in a bunch of \verb|SeqRecord| objects from our orchids GenBank file, and create a string containing the records in FASTA format:

\begin{verbatim}
from Bio import SeqIO
from StringIO import StringIO
records = SeqIO.parse("ls_orchid.gbk", "genbank")
out_handle = StringIO()
SeqIO.write(records, out_handle, "fasta")
fasta_data = out_handle.getvalue()
print(fasta_data)
\end{verbatim}

This isn't entirely straightforward the first time you see it!  On the bright side, for the special case where you would like a string containing a \emph{single} record in a particular file format, use the the \verb|SeqRecord| class' \verb|format()| method (see Section~\ref{sec:SeqRecord-format}).

Note that although we don't encourage it, you \emph{can} use the \verb|format()| method to write to a file, for example something like this:
\begin{verbatim}
from Bio import SeqIO
out_handle = open("ls_orchid_long.tab", "w")
for record in SeqIO.parse("ls_orchid.gbk", "genbank"):
    if len(record) > 100:
        out_handle.write(record.format("tab"))
out_handle.close()
\end{verbatim}
\noindent While this style of code will work for a simple sequential file format like FASTA or the simple tab separated format used here, it will \emph{not} work for more complex or interlaced file formats.  This is why we still recommend using \verb|Bio.SeqIO.write()|, as in the following example:
\begin{verbatim}
from Bio import SeqIO
records = (rec for rec in SeqIO.parse("ls_orchid.gbk", "genbank") if len(rec) > 100)
SeqIO.write(records, "ls_orchid.tab", "tab")
\end{verbatim}
\noindent Making a single call to \verb|SeqIO.write(...)| is also much quicker than
multiple calls to the \verb|SeqRecord.format(...)| method.



%\chapter{Multiple Sequence Alignment objects}
%\label{chapter:Bio.AlignIO}
\include{Tutorial/chapter_align}

%\chapter{BLAST}
%\label{chapter:blast}
\include{Tutorial/chapter_blast}

%\chapter{BLAST and other sequence search tools (\textit{experimental code})}
%\label{chapter:searchio}
\chapter{BLAST and other sequence search tools (\textit{experimental code})}
\label{chapter:searchio}

\emph{WARNING}: This chapter of the Tutorial describes an \emph{experimental}
module in Biopython. It is being included in Biopython and documented
here in  the tutorial in a pre-final state to allow a period of feedback
and refinement before we declare it stable. Until then the details will
probably change, and any scripts using the current \verb|Bio.SearchIO|
would need to be updated. Please keep this in mind! For stable code
working with NCBI BLAST, please continue to use Bio.Blast described
in the preceding Chapter~\ref{chapter:blast}.

Biological sequence identification is an integral part of bioinformatics.
Several tools are available for this, each with their own algorithms and
approaches, such as BLAST (arguably the most popular), FASTA, HMMER, and many
more. In general, these tools usually use your sequence to search a database of
potential matches. With the growing number of known sequences (hence the
growing number of potential matches), interpreting the results becomes
increasingly hard as there could be hundreds or even thousands of potential
matches. Naturally, manual interpretation of these searches' results is out of
the question. Moreover, you often need to work with several sequence search
tools, each with its own statistics, conventions, and output format. Imagine how
daunting it would be when you need to work with multiple sequences using
multiple search tools.

We know this too well ourselves, which is why we created the \verb|Bio.SearchIO|
submodule in Biopython. \verb|Bio.SearchIO| allows you to extract information
from your search results in a convenient way, while also dealing with the
different standards and conventions used by different search tools.
The name \verb|SearchIO| is a homage to BioPerl's module of the same name.

In this chapter, we'll go through the main features of \verb|Bio.SearchIO| to
show what it can do for you. We'll use two popular search tools along the way:
BLAST and BLAT. They are used merely for illustrative purposes, and you should
be able to adapt the workflow to any other search tools supported by
\verb|Bio.SearchIO| in a breeze. You're very welcome to follow along with the
search output files we'll be using. The BLAST output file can be downloaded
\href{http://biopython.org/SRC/biopython/Doc/examples/my_blast.xml}{here},
and the BLAT output file
\href{http://biopython.org/SRC/biopython/Doc/examples/my_blat.psl}{here}
or are included with the Biopython source code under the \verb|Doc/examples/|
folder. Both output files were generated using this sequence:

\begin{verbatim}
>mystery_seq
CCCTCTACAGGGAAGCGCTTTCTGTTGTCTGAAAGAAAAGAAAGTGCTTCCTTTTAGAGGG
\end{verbatim}

The BLAST result is an XML file generated using \verb|blastn| against the NCBI
\verb|refseq_rna| database. For BLAT, the sequence database was the February 2009
\verb|hg19| human genome draft and the output format is PSL.

We'll start from an introduction to the \verb|Bio.SearchIO| object model. The
model is the representation of your search results, thus it is core to
\verb|Bio.SearchIO| itself. After that, we'll check out the main functions in
\verb|Bio.SearchIO| that you may often use.

Now that we're all set, let's go to the first step: introducing the core
object model.

\section{The SearchIO object model}
\label{sec:searchio-model}

Despite the wildly differing output styles among many sequence search tools,
it turns out that their underlying concept is similar:

\begin{itemize}
\item The output file may contain results from one or more search queries.
\item In each search query, you will see one or more hits from the given
search database.
\item In each database hit, you will see one or more regions containing the
actual sequence alignment between your query sequence and the database
sequence.
\item Some programs like BLAT or Exonerate may further split these regions into
several alignment fragments (or blocks in BLAT and possibly exons in
exonerate). This is not something you always see, as programs like BLAST and
HMMER do not do this.
\end{itemize}

Realizing this generality, we decided use it as base for creating the
\verb|Bio.SearchIO| object model. The object model consists of a nested
hierarchy of Python objects, each one representing one concept outlined above.
These objects are:

\begin{itemize}
\item \verb|QueryResult|, to represent a single search query.
\item \verb|Hit|, to represent a single database hit. \verb|Hit| objects are
    contained within \verb|QueryResult| and in each \verb|QueryResult| there is
    zero or more \verb|Hit| objects.
\item \verb|HSP| (short for high-scoring pair), to represent region(s) of
    significant alignments between query and hit sequences. \verb|HSP| objects
    are contained within \verb|Hit| objects and each \verb|Hit| has one or more
    \verb|HSP| objects.
\item \verb|HSPFragment|, to represent a single contiguous alignment between
    query and hit sequences. \verb|HSPFragment| objects are contained within
    \verb|HSP| objects. Most sequence search tools like BLAST and HMMER unify
    \verb|HSP| and \verb|HSPFragment| objects as each \verb|HSP| will only have
    a single \verb|HSPFragment|. However there are tools like BLAT and Exonerate
    that produce \verb|HSP| containing multiple \verb|HSPFragment|. Don't worry
    if this seems a tad confusing now, we'll elaborate more on these two objects
    later on.
\end{itemize}

These four objects are the ones you will interact with when you use
\verb|Bio.SearchIO|. They are created using one of the main \verb|Bio.SearchIO|
methods: \verb|read|, \verb|parse|, \verb|index|, or \verb|index_db|. The
details of these methods are provided in later sections. For this section, we'll
only be using read and parse. These functions behave similarly to their
\verb|Bio.SeqIO| and \verb|Bio.AlignIO| counterparts:

\begin{itemize}
\item \verb|read| is used for search output files with a single query and
    returns a \verb|QueryResult| object
\item \verb|parse| is used for search output files with multiple queries and
    returns a generator that yields \verb|QueryResult| objects
\end{itemize}

With that settled, let's start probing each \verb|Bio.SearchIO| object,
beginning with \verb|QueryResult|.

\subsection{QueryResult}
\label{sec:searchio-qresult}

The QueryResult object represents a single search query and contains zero or
more Hit objects. Let's see what it looks like using the BLAST file we have:

%doctest examples
\begin{verbatim}
>>> from Bio import SearchIO
>>> blast_qresult = SearchIO.read('my_blast.xml', 'blast-xml')
>>> print(blast_qresult)
Program: blastn (2.2.27+)
  Query: 42291 (61)
         mystery_seq
 Target: refseq_rna
   Hits: ----  -----  ----------------------------------------------------------
            #  # HSP  ID + description
         ----  -----  ----------------------------------------------------------
            0      1  gi|262205317|ref|NR_030195.1|  Homo sapiens microRNA 52...
            1      1  gi|301171311|ref|NR_035856.1|  Pan troglodytes microRNA...
            2      1  gi|270133242|ref|NR_032573.1|  Macaca mulatta microRNA ...
            3      2  gi|301171322|ref|NR_035857.1|  Pan troglodytes microRNA...
            4      1  gi|301171267|ref|NR_035851.1|  Pan troglodytes microRNA...
            5      2  gi|262205330|ref|NR_030198.1|  Homo sapiens microRNA 52...
            6      1  gi|262205302|ref|NR_030191.1|  Homo sapiens microRNA 51...
            7      1  gi|301171259|ref|NR_035850.1|  Pan troglodytes microRNA...
            8      1  gi|262205451|ref|NR_030222.1|  Homo sapiens microRNA 51...
            9      2  gi|301171447|ref|NR_035871.1|  Pan troglodytes microRNA...
           10      1  gi|301171276|ref|NR_035852.1|  Pan troglodytes microRNA...
           11      1  gi|262205290|ref|NR_030188.1|  Homo sapiens microRNA 51...
           12      1  gi|301171354|ref|NR_035860.1|  Pan troglodytes microRNA...
           13      1  gi|262205281|ref|NR_030186.1|  Homo sapiens microRNA 52...
           14      2  gi|262205298|ref|NR_030190.1|  Homo sapiens microRNA 52...
           15      1  gi|301171394|ref|NR_035865.1|  Pan troglodytes microRNA...
           16      1  gi|262205429|ref|NR_030218.1|  Homo sapiens microRNA 51...
           17      1  gi|262205423|ref|NR_030217.1|  Homo sapiens microRNA 52...
           18      1  gi|301171401|ref|NR_035866.1|  Pan troglodytes microRNA...
           19      1  gi|270133247|ref|NR_032574.1|  Macaca mulatta microRNA ...
           20      1  gi|262205309|ref|NR_030193.1|  Homo sapiens microRNA 52...
           21      2  gi|270132717|ref|NR_032716.1|  Macaca mulatta microRNA ...
           22      2  gi|301171437|ref|NR_035870.1|  Pan troglodytes microRNA...
           23      2  gi|270133306|ref|NR_032587.1|  Macaca mulatta microRNA ...
           24      2  gi|301171428|ref|NR_035869.1|  Pan troglodytes microRNA...
           25      1  gi|301171211|ref|NR_035845.1|  Pan troglodytes microRNA...
           26      2  gi|301171153|ref|NR_035838.1|  Pan troglodytes microRNA...
           27      2  gi|301171146|ref|NR_035837.1|  Pan troglodytes microRNA...
           28      2  gi|270133254|ref|NR_032575.1|  Macaca mulatta microRNA ...
           29      2  gi|262205445|ref|NR_030221.1|  Homo sapiens microRNA 51...
           ~~~
           97      1  gi|356517317|ref|XM_003527287.1|  PREDICTED: Glycine ma...
           98      1  gi|297814701|ref|XM_002875188.1|  Arabidopsis lyrata su...
           99      1  gi|397513516|ref|XM_003827011.1|  PREDICTED: Pan panisc...
\end{verbatim}

We've just begun to scratch the surface of the object model, but you can see that
there's already some useful information. By invoking \verb|print| on the
\verb|QueryResult| object, you can see:

\begin{itemize}
\item The program name and version (blastn version 2.2.27+)
\item The query ID, description, and its sequence length (ID is 42291,
    description is `mystery\_seq', and it is 61 nucleotides long)
\item The target database to search against (refseq\_rna)
\item A quick overview of the resulting hits. For our query sequence, there are
    100 potential hits (numbered 0--99 in the table). For each hit, we can also see
    how many HSPs it contains, its ID, and a snippet of its description. Notice
    here that \verb|Bio.SearchIO| truncates the hit table overview, by showing
    only hits numbered 0--29, and then 97--99.
\end{itemize}

Now let's check our BLAT results using the same procedure as above:

%cont-doctest
\begin{verbatim}
>>> blat_qresult = SearchIO.read('my_blat.psl', 'blat-psl')
>>> print(blat_qresult)
Program: blat (<unknown version>)
  Query: mystery_seq (61)
         <unknown description>
 Target: <unknown target>
   Hits: ----  -----  ----------------------------------------------------------
            #  # HSP  ID + description
         ----  -----  ----------------------------------------------------------
            0     17  chr19  <unknown description>
\end{verbatim}

You'll immediately notice that there are some differences. Some of these are
caused by the way PSL format stores its details, as you'll see. The rest are
caused by the genuine program and target database differences between our BLAST
and BLAT searches:

\begin{itemize}
\item The program name and version. \verb|Bio.SearchIO| knows that the program
    is BLAT, but in the output file there is no information regarding the
    program version so it defaults to `<unknown version>'.
\item The query ID, description, and its sequence length. Notice here that these
    details are slightly different from the ones we saw in BLAST. The ID is
    `mystery\_seq' instead of 42991, there is no known description, but the query
    length is still 61. This is actually a difference introduced by the file
    formats themselves. BLAST sometimes creates its own query IDs and uses your
    original ID as the sequence description.
\item The target database is not known, as it is not stated in the BLAT output
    file.
\item And finally, the list of hits we have is completely different. Here, we
    see that our query sequence only hits the `chr19' database entry, but in it
    we see 17 HSP regions. This should not be surprising however, given that we
    are using a different program, each with its own target database.
\end{itemize}

All the details you saw when invoking the \verb|print| method can be accessed
individually using Python's object attribute access notation (a.k.a. the dot
notation). There are also other format-specific attributes that you can access
using the same method.

%cont-doctest
\begin{verbatim}
>>> print("%s %s" % (blast_qresult.program, blast_qresult.version))
blastn 2.2.27+
>>> print("%s %s" % (blat_qresult.program, blat_qresult.version))
blat <unknown version>
>>> blast_qresult.param_evalue_threshold    # blast-xml specific
10.0
\end{verbatim}

For a complete list of accessible attributes, you can check each format-specific
documentation. Here are the ones
\href{http://biopython.org/DIST/docs/api/Bio.SearchIO.BlastIO-module.html}{for BLAST}
and for
\href{http://biopython.org/DIST/docs/api/Bio.SearchIO.BlatIO-module.html}{BLAT}.

Having looked at using \verb|print| on \verb|QueryResult| objects, let's drill
down deeper. What exactly is a \verb|QueryResult|? In terms of Python objects,
\verb|QueryResult| is a hybrid between a list and a dictionary. In other words,
it is a container object with all the convenient features of lists and
dictionaries.

Like Python lists and dictionaries, \verb|QueryResult| objects are iterable.
Each iteration returns a \verb|Hit| object:

\begin{verbatim}
>>> for hit in blast_qresult:
...     hit
Hit(id='gi|262205317|ref|NR_030195.1|', query_id='42291', 1 hsps)
Hit(id='gi|301171311|ref|NR_035856.1|', query_id='42291', 1 hsps)
Hit(id='gi|270133242|ref|NR_032573.1|', query_id='42291', 1 hsps)
Hit(id='gi|301171322|ref|NR_035857.1|', query_id='42291', 2 hsps)
Hit(id='gi|301171267|ref|NR_035851.1|', query_id='42291', 1 hsps)
...
\end{verbatim}

To check how many items (hits) a \verb|QueryResult| has, you can simply invoke
Python's \verb|len| method:

%cont-doctest
\begin{verbatim}
>>> len(blast_qresult)
100
>>> len(blat_qresult)
1
\end{verbatim}

Like Python lists, you can retrieve items (hits) from a \verb|QueryResult| using
the slice notation:

%cont-doctest
\begin{verbatim}
>>> blast_qresult[0]        # retrieves the top hit
Hit(id='gi|262205317|ref|NR_030195.1|', query_id='42291', 1 hsps)
>>> blast_qresult[-1]       # retrieves the last hit
Hit(id='gi|397513516|ref|XM_003827011.1|', query_id='42291', 1 hsps)
\end{verbatim}

To retrieve multiple hits, you can slice \verb|QueryResult| objects using the
slice notation as well. In this case, the slice will return a new
\verb|QueryResult| object containing only the sliced hits:

%cont-doctest
\begin{verbatim}
>>> blast_slice = blast_qresult[:3]     # slices the first three hits
>>> print(blast_slice)
Program: blastn (2.2.27+)
  Query: 42291 (61)
         mystery_seq
 Target: refseq_rna
   Hits: ----  -----  ----------------------------------------------------------
            #  # HSP  ID + description
         ----  -----  ----------------------------------------------------------
            0      1  gi|262205317|ref|NR_030195.1|  Homo sapiens microRNA 52...
            1      1  gi|301171311|ref|NR_035856.1|  Pan troglodytes microRNA...
            2      1  gi|270133242|ref|NR_032573.1|  Macaca mulatta microRNA ...
\end{verbatim}

Like Python dictionaries, you can also retrieve hits using the hit's ID. This is
particularly useful if you know a given hit ID exists within a search query
results:

%cont-doctest
\begin{verbatim}
>>> blast_qresult['gi|262205317|ref|NR_030195.1|']
Hit(id='gi|262205317|ref|NR_030195.1|', query_id='42291', 1 hsps)
\end{verbatim}

You can also get a full list of \verb|Hit| objects using \verb|hits| and a full
list of \verb|Hit| IDs using \verb|hit_keys|:

\begin{verbatim}
>>> blast_qresult.hits
[...]       # list of all hits
>>> blast_qresult.hit_keys
[...]       # list of all hit IDs
\end{verbatim}

What if you just want to check whether a particular hit is present in the query
results? You can do a simple Python membership test using the \verb|in| keyword:

%cont-doctest
\begin{verbatim}
>>> 'gi|262205317|ref|NR_030195.1|' in blast_qresult
True
>>> 'gi|262205317|ref|NR_030194.1|' in blast_qresult
False
\end{verbatim}

Sometimes, knowing whether a hit is present is not enough; you also want to know
the rank of the hit. Here, the \verb|index| method comes to the rescue:

%cont-doctest
\begin{verbatim}
>>> blast_qresult.index('gi|301171437|ref|NR_035870.1|')
22
\end{verbatim}

Remember that we're using Python's indexing style here, which is zero-based.
This means our hit above is ranked at no. 23, not 22.

Also, note that the hit rank you see here is based on the native hit ordering
present in the original search output file. Different search tools may order
these hits based on different criteria.

If the native hit ordering doesn't suit your taste, you can use the \verb|sort|
method of the \verb|QueryResult| object. It is very similar to Python's
\verb|list.sort| method, with the addition of an option to create a new sorted
\verb|QueryResult| object or not.

Here is an example of using \verb|QueryResult.sort| to sort the hits based on
each hit's full sequence length. For this particular sort, we'll set the
\verb|in_place| flag to \verb|False| so that sorting will return a new
\verb|QueryResult| object and leave our initial object unsorted. We'll also set
the \verb|reverse| flag to \verb|True| so that we sort in descending order.

%cont-doctest
\begin{verbatim}
>>> for hit in blast_qresult[:5]:   # id and sequence length of the first five hits
...     print("%s %i" % (hit.id, hit.seq_len))
...
gi|262205317|ref|NR_030195.1| 61
gi|301171311|ref|NR_035856.1| 60
gi|270133242|ref|NR_032573.1| 85
gi|301171322|ref|NR_035857.1| 86
gi|301171267|ref|NR_035851.1| 80

>>> sort_key = lambda hit: hit.seq_len
>>> sorted_qresult = blast_qresult.sort(key=sort_key, reverse=True, in_place=False)
>>> for hit in sorted_qresult[:5]:
...     print("%s %i" % (hit.id, hit.seq_len))
...
gi|397513516|ref|XM_003827011.1| 6002
gi|390332045|ref|XM_776818.2| 4082
gi|390332043|ref|XM_003723358.1| 4079
gi|356517317|ref|XM_003527287.1| 3251
gi|356543101|ref|XM_003539954.1| 2936
\end{verbatim}

The advantage of having the \verb|in_place| flag here is that we're preserving
the native ordering, so we may use it again later. You should note that this is
not the default behavior of \verb|QueryResult.sort|, however, which is why we
needed to set the \verb|in_place| flag to \verb|True| explicitly.

At this point, you've known enough about \verb|QueryResult| objects to make it
work for you. But before we go on to the next object in the \verb|Bio.SearchIO|
model, let's take a look at two more sets of methods that could make it even
easier to work with \verb|QueryResult| objects: the \verb|filter| and \verb|map|
methods.

If you're familiar with Python's list comprehensions, generator expressions
or the built in \verb|filter| and \verb|map| functions,
you'll know how useful they are for working with list-like objects (if you're
not, check them out!). You can use these built in methods to manipulate
\verb|QueryResult| objects, but you'll end up with regular Python lists and lose
the ability to do more interesting manipulations.

That's why, \verb|QueryResult| objects provide its own flavor of
\verb|filter| and \verb|map| methods. Analogous to \verb|filter|, there are
\verb|hit_filter| and \verb|hsp_filter| methods. As their name implies, these
methods filter its \verb|QueryResult| object either on its \verb|Hit| objects
or \verb|HSP| objects. Similarly, analogous to \verb|map|, \verb|QueryResult|
objects also provide the \verb|hit_map| and \verb|hsp_map| methods. These
methods apply a given function to all hits or HSPs in a \verb|QueryResult|
object, respectively.

Let's see these methods in action, beginning with \verb|hit_filter|. This method
accepts a callback function that checks whether a given \verb|Hit| object passes
the condition you set or not. In other words, the function must accept as its
argument a single \verb|Hit| object and returns \verb|True| or \verb|False|.

Here is an example of using \verb|hit_filter| to filter out \verb|Hit| objects
that only have one HSP:

%cont-doctest
\begin{verbatim}
>>> filter_func = lambda hit: len(hit.hsps) > 1     # the callback function
>>> len(blast_qresult)      # no. of hits before filtering
100
>>> filtered_qresult = blast_qresult.hit_filter(filter_func)
>>> len(filtered_qresult)   # no. of hits after filtering
37
>>> for hit in filtered_qresult[:5]:    # quick check for the hit lengths
...     print("%s %i" % (hit.id, len(hit.hsps)))
gi|301171322|ref|NR_035857.1| 2
gi|262205330|ref|NR_030198.1| 2
gi|301171447|ref|NR_035871.1| 2
gi|262205298|ref|NR_030190.1| 2
gi|270132717|ref|NR_032716.1| 2
\end{verbatim}

\verb|hsp_filter| works the same as \verb|hit_filter|, only instead of looking
at the \verb|Hit| objects, it performs filtering on the \verb|HSP| objects in
each hits.

As for the \verb|map| methods, they too accept a callback function as their
arguments. However, instead of returning \verb|True| or \verb|False|, the
callback function must return the modified \verb|Hit| or \verb|HSP| object
(depending on whether you're using \verb|hit_map| or \verb|hsp_map|).

Let's see an example where we're using \verb|hit_map| to rename the hit IDs:

%cont-doctest
\begin{verbatim}
>>> def map_func(hit):
...     hit.id = hit.id.split('|')[3]   # renames 'gi|301171322|ref|NR_035857.1|' to 'NR_035857.1'
...     return hit
...
>>> mapped_qresult = blast_qresult.hit_map(map_func)
>>> for hit in mapped_qresult[:5]:
...     print(hit.id)
NR_030195.1
NR_035856.1
NR_032573.1
NR_035857.1
NR_035851.1
\end{verbatim}

Again, \verb|hsp_map| works the same as \verb|hit_map|, but on \verb|HSP|
objects instead of \verb|Hit| objects.

\subsection{Hit}
\label{sec:searchio-hit}

\verb|Hit| objects represent all query results from a single database entry.
They are the second-level container in the \verb|Bio.SearchIO| object hierarchy.
You've seen that they are contained by \verb|QueryResult| objects, but they
themselves contain \verb|HSP| objects.

Let's see what they look like, beginning with our BLAST search:

%doctest examples
\begin{verbatim}
>>> from Bio import SearchIO
>>> blast_qresult = SearchIO.read('my_blast.xml', 'blast-xml')
>>> blast_hit = blast_qresult[3]    # fourth hit from the query result
>>> print(blast_hit)
Query: 42291
       mystery_seq
  Hit: gi|301171322|ref|NR_035857.1| (86)
       Pan troglodytes microRNA mir-520c (MIR520C), microRNA
 HSPs: ----  --------  ---------  ------  ---------------  ---------------------
          #   E-value  Bit score    Span      Query range              Hit range
       ----  --------  ---------  ------  ---------------  ---------------------
          0   8.9e-20     100.47      60           [1:61]                [13:73]
          1   3.3e-06      55.39      60           [0:60]                [13:73]
\end{verbatim}

You see that we've got the essentials covered here:

\begin{itemize}
\item The query ID and description is present. A hit is always tied to a query,
    so we want to keep track of the originating query as well. These values can
    be accessed from a hit using the \verb|query_id| and
    \verb|query_description| attributes.
\item We also have the unique hit ID, description, and full sequence lengths.
    They can be accessed using \verb|id|, \verb|description|, and
    \verb|seq_len|, respectively.
\item Finally, there's a table containing quick information about the HSPs this
    hit contains. In each row, we've got the important HSP details listed: the
    HSP index, its e-value, its bit score, its span (the alignment length
    including gaps), its query coordinates, and its hit coordinates.
\end{itemize}

Now let's contrast this with the BLAT search. Remember that in the BLAT search we
had one hit with 17 HSPs.

%cont-doctest
\begin{verbatim}
>>> blat_qresult = SearchIO.read('my_blat.psl', 'blat-psl')
>>> blat_hit = blat_qresult[0]      # the only hit
>>> print(blat_hit)
Query: mystery_seq
       <unknown description>
  Hit: chr19 (59128983)
       <unknown description>
 HSPs: ----  --------  ---------  ------  ---------------  ---------------------
          #   E-value  Bit score    Span      Query range              Hit range
       ----  --------  ---------  ------  ---------------  ---------------------
          0         ?          ?       ?           [0:61]    [54204480:54204541]
          1         ?          ?       ?           [0:61]    [54233104:54264463]
          2         ?          ?       ?           [0:61]    [54254477:54260071]
          3         ?          ?       ?           [1:61]    [54210720:54210780]
          4         ?          ?       ?           [0:60]    [54198476:54198536]
          5         ?          ?       ?           [0:61]    [54265610:54265671]
          6         ?          ?       ?           [0:61]    [54238143:54240175]
          7         ?          ?       ?           [0:60]    [54189735:54189795]
          8         ?          ?       ?           [0:61]    [54185425:54185486]
          9         ?          ?       ?           [0:60]    [54197657:54197717]
         10         ?          ?       ?           [0:61]    [54255662:54255723]
         11         ?          ?       ?           [0:61]    [54201651:54201712]
         12         ?          ?       ?           [8:60]    [54206009:54206061]
         13         ?          ?       ?          [10:61]    [54178987:54179038]
         14         ?          ?       ?           [8:61]    [54212018:54212071]
         15         ?          ?       ?           [8:51]    [54234278:54234321]
         16         ?          ?       ?           [8:61]    [54238143:54238196]
\end{verbatim}

Here, we've got a similar level of detail as with the BLAST hit we saw earlier.
There are some differences worth explaining, though:

\begin{itemize}
\item The e-value and bit score column values. As BLAT HSPs do not have e-values
    and bit scores, the display defaults to `?'.
\item What about the span column? The span values is meant to display the
    complete alignment length, which consists of all residues and any gaps that
    may be present. The PSL format do not have this information readily available
    and \verb|Bio.SearchIO| does not attempt to try guess what it is, so we get a
    `?' similar to the e-value and bit score columns.
\end{itemize}

In terms of Python objects, \verb|Hit| behaves almost the same as Python lists,
but contain \verb|HSP| objects exclusively. If you're familiar with lists, you
should encounter no difficulties working with the \verb|Hit| object.

Just like Python lists, \verb|Hit| objects are iterable, and each iteration
returns one \verb|HSP| object it contains:

%cont-doctest
\begin{verbatim}
>>> for hsp in blast_hit:
...     hsp
HSP(hit_id='gi|301171322|ref|NR_035857.1|', query_id='42291', 1 fragments)
HSP(hit_id='gi|301171322|ref|NR_035857.1|', query_id='42291', 1 fragments)
\end{verbatim}

You can invoke \verb|len| on a \verb|Hit| to see how many \verb|HSP| objects it
has:

%cont-doctest
\begin{verbatim}
>>> len(blast_hit)
2
>>> len(blat_hit)
17
\end{verbatim}

You can use the slice notation on \verb|Hit| objects, whether to retrieve single
\verb|HSP| or multiple \verb|HSP| objects. Like \verb|QueryResult|, if you slice
for multiple \verb|HSP|, a new \verb|Hit| object will be returned containing
only the sliced \verb|HSP| objects:

%cont-doctest
\begin{verbatim}
>>> blat_hit[0]                 # retrieve single items
HSP(hit_id='chr19', query_id='mystery_seq', 1 fragments)
>>> sliced_hit = blat_hit[4:9]  # retrieve multiple items
>>> len(sliced_hit)
5
>>> print(sliced_hit)
Query: mystery_seq
       <unknown description>
  Hit: chr19 (59128983)
       <unknown description>
 HSPs: ----  --------  ---------  ------  ---------------  ---------------------
          #   E-value  Bit score    Span      Query range              Hit range
       ----  --------  ---------  ------  ---------------  ---------------------
          0         ?          ?       ?           [0:60]    [54198476:54198536]
          1         ?          ?       ?           [0:61]    [54265610:54265671]
          2         ?          ?       ?           [0:61]    [54238143:54240175]
          3         ?          ?       ?           [0:60]    [54189735:54189795]
          4         ?          ?       ?           [0:61]    [54185425:54185486]
\end{verbatim}

You can also sort the \verb|HSP| inside a \verb|Hit|, using the exact same
arguments like the sort method you saw in the \verb|QueryResult| object.

Finally, there are also the \verb|filter| and \verb|map| methods you can use
on \verb|Hit| objects. Unlike in the \verb|QueryResult| object, \verb|Hit|
objects only have one variant of \verb|filter| (\verb|Hit.filter|) and one
variant of \verb|map| (\verb|Hit.map|). Both of \verb|Hit.filter| and
\verb|Hit.map| work on the \verb|HSP| objects a \verb|Hit| has.

\subsection{HSP}
\label{sec:searchio-hsp}

\verb|HSP| (high-scoring pair) represents region(s) in the hit sequence that
contains significant alignment(s) to the query sequence. It contains the actual
match between your query sequence and a database entry. As this match is
determined by the sequence search tool's algorithms, the \verb|HSP| object
contains the bulk of the statistics computed by the search tool. This also makes
the distinction between \verb|HSP| objects from different search tools more
apparent compared to the differences you've seen in \verb|QueryResult| or
\verb|Hit| objects.

Let's see some examples from our BLAST and BLAT searches. We'll look at the
BLAST HSP first:

%doctest examples
\begin{verbatim}
>>> from Bio import SearchIO
>>> blast_qresult = SearchIO.read('my_blast.xml', 'blast-xml')
>>> blast_hsp = blast_qresult[0][0]    # first hit, first hsp
>>> print(blast_hsp)
      Query: 42291 mystery_seq
        Hit: gi|262205317|ref|NR_030195.1| Homo sapiens microRNA 520b (MIR520...
Query range: [0:61] (1)
  Hit range: [0:61] (1)
Quick stats: evalue 4.9e-23; bitscore 111.29
  Fragments: 1 (61 columns)
     Query - CCCTCTACAGGGAAGCGCTTTCTGTTGTCTGAAAGAAAAGAAAGTGCTTCCTTTTAGAGGG
             |||||||||||||||||||||||||||||||||||||||||||||||||||||||||||||
       Hit - CCCTCTACAGGGAAGCGCTTTCTGTTGTCTGAAAGAAAAGAAAGTGCTTCCTTTTAGAGGG
\end{verbatim}

Just like \verb|QueryResult| and \verb|Hit|, invoking \verb|print| on an
\verb|HSP| shows its general details:
\begin{itemize}
\item There are the query and hit IDs and descriptions. We need these to
    identify our \verb|HSP|.
\item We've also got the matching range of the query and hit sequences. The
    slice notation we're using here is an indication that the range is displayed
    using Python's indexing style (zero-based, half open). The number inside the
    parenthesis denotes the strand. In this case, both sequences have the plus
    strand.
\item Some quick statistics are available: the e-value and bitscore.
\item There is information about the HSP fragments. Ignore this for now; it will
    be explained later on.
\item And finally, we have the query and hit sequence alignment itself.
\end{itemize}

These details can be accessed on their own using the dot notation, just like in
\verb|QueryResult| and \verb|Hit|:

%cont-doctest
\begin{verbatim}
>>> blast_hsp.query_range
(0, 61)
\end{verbatim}
%hack! since float display may be different across versions
\begin{verbatim}
>>> blast_hsp.evalue
4.91307e-23
\end{verbatim}

They're not the only attributes available, though. \verb|HSP| objects come with
a default set of properties that makes it easy to probe their various
details. Here are some examples:

%cont-doctest
\begin{verbatim}
>>> blast_hsp.hit_start         # start coordinate of the hit sequence
0
>>> blast_hsp.query_span        # how many residues in the query sequence
61
>>> blast_hsp.aln_span          # how long the alignment is
61
\end{verbatim}

Check out the \verb|HSP|
\href{http://biopython.org/DIST/docs/api/Bio.SearchIO._model.hsp-module.html}{documentation}
for a full list of these predefined properties.

Furthermore, each sequence search tool usually computes its own statistics /
details for its \verb|HSP| objects. For example, an XML BLAST search also
outputs the number of gaps and identical residues. These attributes can be
accessed like so:

%cont-doctest
\begin{verbatim}
>>> blast_hsp.gap_num       # number of gaps
0
>>> blast_hsp.ident_num     # number of identical residues
61
\end{verbatim}

These details are format-specific; they may not be present in other formats.
To see which details are available for a given sequence search tool, you
should check the format's documentation in \verb|Bio.SearchIO|. Alternatively,
you may also use \verb|.__dict__.keys()| for a quick list of what's available:

\begin{verbatim}
>>> blast_hsp.__dict__.keys()
['bitscore', 'evalue', 'ident_num', 'gap_num', 'bitscore_raw', 'pos_num', '_items']
\end{verbatim}

Finally, you may have noticed that the \verb|query| and \verb|hit| attributes
of our HSP are not just regular strings:

%cont-doctest
\begin{verbatim}
>>> blast_hsp.query
SeqRecord(seq=Seq('CCCTCTACAGGGAAGCGCTTTCTGTTGTCTGAAAGAAAAGAAAGTGCTTCCTTT...GGG', DNAAlphabet()), id='42291', name='aligned query sequence', description='mystery_seq', dbxrefs=[])
>>> blast_hsp.hit
SeqRecord(seq=Seq('CCCTCTACAGGGAAGCGCTTTCTGTTGTCTGAAAGAAAAGAAAGTGCTTCCTTT...GGG', DNAAlphabet()), id='gi|262205317|ref|NR_030195.1|', name='aligned hit sequence', description='Homo sapiens microRNA 520b (MIR520B), microRNA', dbxrefs=[])
\end{verbatim}

They are \verb|SeqRecord| objects you saw earlier in
Section~\ref{chapter:SeqRecord}! This means that you can do all sorts of
interesting things you can do with \verb|SeqRecord| objects on \verb|HSP.query|
and/or \verb|HSP.hit|.

It should not surprise you now that the \verb|HSP| object has an
\verb|alignment| property which is a \verb|MultipleSeqAlignment| object:

%cont-doctest
\begin{verbatim}
>>> print(blast_hsp.aln)
DNAAlphabet() alignment with 2 rows and 61 columns
CCCTCTACAGGGAAGCGCTTTCTGTTGTCTGAAAGAAAAGAAAG...GGG 42291
CCCTCTACAGGGAAGCGCTTTCTGTTGTCTGAAAGAAAAGAAAG...GGG gi|262205317|ref|NR_030195.1|
\end{verbatim}

Having probed the BLAST HSP, let's now take a look at HSPs from our BLAT
results for a different kind of HSP. As usual, we'll begin by invoking
\verb|print| on it:

%cont-doctest
\begin{verbatim}
>>> blat_qresult = SearchIO.read('my_blat.psl', 'blat-psl')
>>> blat_hsp = blat_qresult[0][0]       # first hit, first hsp
>>> print(blat_hsp)
      Query: mystery_seq <unknown description>
        Hit: chr19 <unknown description>
Query range: [0:61] (1)
  Hit range: [54204480:54204541] (1)
Quick stats: evalue ?; bitscore ?
  Fragments: 1 (? columns)
\end{verbatim}

Some of the outputs you may have already guessed. We have the query and hit IDs
and descriptions and the sequence coordinates. Values for evalue and bitscore is
`?' as BLAT HSPs do not have these attributes. But The biggest difference here
is that you don't see any sequence alignments displayed. If you look closer, PSL
formats themselves do not have any hit or query sequences, so
\verb|Bio.SearchIO| won't create any sequence or alignment objects. What happens
if you try to access \verb|HSP.query|, \verb|HSP.hit|, or \verb|HSP.aln|?
You'll get the default values for these attributes, which is \verb|None|:

%cont-doctest
\begin{verbatim}
>>> blat_hsp.hit is None
True
>>> blat_hsp.query is None
True
>>> blat_hsp.aln is None
True
\end{verbatim}

This does not affect other attributes, though. For example, you can still
access the length of the query or hit alignment. Despite not displaying any
attributes, the PSL format still have this information so \verb|Bio.SearchIO|
can extract them:

%cont-doctest
\begin{verbatim}
>>> blat_hsp.query_span     # length of query match
61
>>> blat_hsp.hit_span       # length of hit match
61
\end{verbatim}

Other format-specific attributes are still present as well:

%cont-doctest
\begin{verbatim}
>>> blat_hsp.score          # PSL score
61
>>> blat_hsp.mismatch_num   # the mismatch column
0
\end{verbatim}

So far so good? Things get more interesting when you look at another `variant'
of HSP present in our BLAT results. You might recall that in BLAT searches,
sometimes we get our results separated into `blocks'. These blocks are
essentially alignment fragments that may have some intervening sequence between
them.

Let's take a look at a BLAT HSP that contains multiple blocks to see how
\verb|Bio.SearchIO| deals with this:

%cont-doctest
\begin{verbatim}
>>> blat_hsp2 = blat_qresult[0][1]      # first hit, second hsp
>>> print(blat_hsp2)
      Query: mystery_seq <unknown description>
        Hit: chr19 <unknown description>
Query range: [0:61] (1)
  Hit range: [54233104:54264463] (1)
Quick stats: evalue ?; bitscore ?
  Fragments: ---  --------------  ----------------------  ----------------------
               #            Span             Query range               Hit range
             ---  --------------  ----------------------  ----------------------
               0               ?                  [0:18]     [54233104:54233122]
               1               ?                 [18:61]     [54264420:54264463]
\end{verbatim}

What's happening here? We still some essential details covered: the IDs and
descriptions, the coordinates, and the quick statistics are similar to what
you've seen before. But the fragments detail is all different. Instead of
showing `Fragments: 1', we now have a table with two data rows.

This is how \verb|Bio.SearchIO| deals with HSPs having multiple fragments. As
mentioned before, an HSP alignment may be separated by intervening sequences
into fragments. The intervening sequences are not part of the query-hit match,
so they should not be considered part of query nor hit sequence. However, they
do affect how we deal with sequence coordinates, so we can't ignore them.

Take a look at the hit coordinate of the HSP above. In the \verb|Hit range:| field,
we see that the coordinate is \verb|[54233104:54264463]|. But looking at the
table rows, we see that not the entire region spanned by this coordinate matches
our query. Specifically, the intervening region spans from \verb|54233122| to
\verb|54264420|.

Why then, is the query coordinates seem to be contiguous, you ask? This is
perfectly fine. In this case it means that the query match is contiguous (no
intervening regions), while the hit match is not.

All these attributes are accessible from the HSP directly, by the way:

%cont-doctest
\begin{verbatim}
>>> blat_hsp2.hit_range         # hit start and end coordinates of the entire HSP
(54233104, 54264463)
>>> blat_hsp2.hit_range_all     # hit start and end coordinates of each fragment
[(54233104, 54233122), (54264420, 54264463)]
>>> blat_hsp2.hit_span          # hit span of the entire HSP
31359
>>> blat_hsp2.hit_span_all      # hit span of each fragment
[18, 43]
>>> blat_hsp2.hit_inter_ranges  # start and end coordinates of intervening regions in the hit sequence
[(54233122, 54264420)]
>>> blat_hsp2.hit_inter_spans   # span of intervening regions in the hit sequence
[31298]
\end{verbatim}

Most of these attributes are not readily available from the PSL file we have,
but \verb|Bio.SearchIO| calculates them for you on the fly when you parse the
PSL file. All it needs are the start and end coordinates of each fragment.

What about the \verb|query|, \verb|hit|, and \verb|aln| attributes? If the
HSP has multiple fragments, you won't be able to use these attributes as they
only fetch single \verb|SeqRecord| or \verb|MultipleSeqAlignment| objects.
However, you can use their \verb|*_all| counterparts: \verb|query_all|,
\verb|hit_all|, and \verb|aln_all|. These properties will return a list containing
\verb|SeqRecord| or \verb|MultipleSeqAlignment| objects from each of the HSP
fragment. There are other attributes that behave similarly, i.e. they only work
for HSPs with one fragment. Check out the \verb|HSP| \href{http://biopython.org/DIST/docs/api/Bio.SearchIO._model.hsp-module.html}{documentation}
for a full list.

Finally, to check whether you have multiple fragments or not, you can use the
\verb|is_fragmented| property like so:

%cont-doctest
\begin{verbatim}
>>> blat_hsp2.is_fragmented     # BLAT HSP with 2 fragments
True
>>> blat_hsp.is_fragmented      # BLAT HSP from earlier, with one fragment
False
\end{verbatim}

Before we move on, you should also know that we can use the slice notation on
\verb|HSP| objects, just like \verb|QueryResult| or \verb|Hit| objects. When
you use this notation, you'll get an \verb|HSPFragment| object in return, the
last component of the object model.

\subsection{HSPFragment}
\label{sec:searchio-hspfragment}

\verb|HSPFragment| represents a single, contiguous match between the query and
hit sequences. You could consider it the core of the object model and search
result, since it is the presence of these fragments that determine whether your
search have results or not.

In most cases, you don't have to deal with \verb|HSPFragment| objects directly
since not that many sequence search tools fragment their HSPs. When you do have
to deal with them, what you should remember is that \verb|HSPFragment| objects
were written with to be as compact as possible. In most cases, they only contain
attributes directly related to sequences: strands, reading frames, alphabets,
coordinates, the sequences themselves, and their IDs and descriptions.

These attributes are readily shown when you invoke \verb|print| on an
\verb|HSPFragment|. Here's an example, taken from our BLAST search:

%doctest examples
\begin{verbatim}
>>> from Bio import SearchIO
>>> blast_qresult = SearchIO.read('my_blast.xml', 'blast-xml')
>>> blast_frag = blast_qresult[0][0][0]    # first hit, first hsp, first fragment
>>> print(blast_frag)
      Query: 42291 mystery_seq
        Hit: gi|262205317|ref|NR_030195.1| Homo sapiens microRNA 520b (MIR520...
Query range: [0:61] (1)
  Hit range: [0:61] (1)
  Fragments: 1 (61 columns)
     Query - CCCTCTACAGGGAAGCGCTTTCTGTTGTCTGAAAGAAAAGAAAGTGCTTCCTTTTAGAGGG
             |||||||||||||||||||||||||||||||||||||||||||||||||||||||||||||
       Hit - CCCTCTACAGGGAAGCGCTTTCTGTTGTCTGAAAGAAAAGAAAGTGCTTCCTTTTAGAGGG
\end{verbatim}

At this level, the BLAT fragment looks quite similar to the BLAST fragment, save
for the query and hit sequences which are not present:

%cont-doctest
\begin{verbatim}
>>> blat_qresult = SearchIO.read('my_blat.psl', 'blat-psl')
>>> blat_frag = blat_qresult[0][0][0]    # first hit, first hsp, first fragment
>>> print(blat_frag)
      Query: mystery_seq <unknown description>
        Hit: chr19 <unknown description>
Query range: [0:61] (1)
  Hit range: [54204480:54204541] (1)
  Fragments: 1 (? columns)
\end{verbatim}

In all cases, these attributes are accessible using our favorite dot notation.
Some examples:

%cont-doctest
\begin{verbatim}
>>> blast_frag.query_start      # query start coordinate
0
>>> blast_frag.hit_strand       # hit sequence strand
1
>>> blast_frag.hit              # hit sequence, as a SeqRecord object
SeqRecord(seq=Seq('CCCTCTACAGGGAAGCGCTTTCTGTTGTCTGAAAGAAAAGAAAGTGCTTCCTTT...GGG', DNAAlphabet()), id='gi|262205317|ref|NR_030195.1|', name='aligned hit sequence', description='Homo sapiens microRNA 520b (MIR520B), microRNA', dbxrefs=[])
\end{verbatim}

\section{A note about standards and conventions}
\label{sec:searchio-standards}

Before we move on to the main functions, there is something you ought to know
about the standards \verb|Bio.SearchIO| uses. If you've worked with multiple
sequence search tools, you might have had to deal with the many different ways
each program deals with things like sequence coordinates. It might not have been
a pleasant experience as these search tools usually have their own standards.
For example, one tools might use one-based coordinates, while the other uses
zero-based coordinates. Or, one program might reverse the start and end
coordinates if the strand is minus, while others don't. In short, these often
creates unnecessary mess must be dealt with.

We realize this problem ourselves and we intend to address it in
\verb|Bio.SearchIO|. After all, one of the goals of \verb|Bio.SearchIO| is to
create a common, easy to use interface to deal with various search output files.
This means creating standards that extend beyond the object model you just saw.

Now, you might complain, "Not another standard!". Well, eventually we have to
choose one convention or the other, so this is necessary. Plus, we're not
creating something entirely new here; just adopting a standard we think is best
for a Python programmer (it is Biopython, after all).

There are three implicit standards that you can expect when working with
\verb|Bio.SearchIO|:

\begin{itemize}
\item The first one pertains to sequence coordinates. In \verb|Bio.SearchIO|,
    all sequence coordinates follows Python's coordinate style: zero-based and
    half open. For example, if in a BLAST XML output file the start and end
    coordinates of an HSP are 10 and 28, they would become 9 and 28 in
    \verb|Bio.SearchIO|. The start coordinate becomes 9 because Python indices
    start from zero, while the end coordinate remains 28 as Python slices omit
    the last item in an interval.
\item The second is on sequence coordinate orders. In \verb|Bio.SearchIO|, start
    coordinates are always less than or equal to end coordinates. This isn't
    always the case with all sequence search tools, as some of them have larger
    start coordinates when the sequence strand is minus.
\item The last one is on strand and reading frame values. For strands, there are
    only four valid choices: \verb|1| (plus strand), \verb|-1| (minus strand),
    \verb|0| (protein sequences), and \verb|None| (no strand). For reading
    frames, the valid choices are integers from \verb|-3| to \verb|3| and
    \verb|None|.
\end{itemize}

Note that these standards only exist in \verb|Bio.SearchIO| objects. If you
write \verb|Bio.SearchIO| objects into an output format, \verb|Bio.SearchIO|
will use the format's standard for the output. It does not force its standard
over to your output file.

\section{Reading search output files}
\label{sec:searchio-input}

There are two functions you can use for reading search output files into
\verb|Bio.SearchIO| objects: \verb|read| and \verb|parse|. They're essentially
similar to \verb|read| and \verb|parse| functions in other submodules like
\verb|Bio.SeqIO| or \verb|Bio.AlignIO|. In both cases, you need to supply the
search output file name and the file format name, both as Python strings. You
can check the documentation for a list of format names \verb|Bio.SearchIO|
recognizes.

\verb|Bio.SearchIO.read| is used for reading search output files with only one
query and returns a \verb|QueryResult| object. You've seen \verb|read| used in
our previous examples. What you haven't seen is that \verb|read| may also accept
additional keyword arguments, depending on the file format.

Here are some examples. In the first one, we use \verb|read| just like
previously to read a BLAST tabular output file. In the second one, we use a
keyword argument to modify so it parses the BLAST tabular variant with comments
in it:

%doctest ../Tests/Blast
\begin{verbatim}
>>> from Bio import SearchIO
>>> qresult = SearchIO.read('tab_2226_tblastn_003.txt', 'blast-tab')
>>> qresult
QueryResult(id='gi|16080617|ref|NP_391444.1|', 3 hits)
>>> qresult2 = SearchIO.read('tab_2226_tblastn_007.txt', 'blast-tab', comments=True)
>>> qresult2
QueryResult(id='gi|16080617|ref|NP_391444.1|', 3 hits)
\end{verbatim}

These keyword arguments differs among file formats. Check the format
documentation to see if it has keyword arguments that modifies its parser's
behavior.

As for the \verb|Bio.SearchIO.parse|, it is used for reading search output
files with any number of queries. The function returns a generator object that
yields a \verb|QueryResult| object in each iteration. Like
\verb|Bio.SearchIO.read|, it also accepts format-specific keyword arguments:

%doctest ../Tests/Blast
\begin{verbatim}
>>> from Bio import SearchIO
>>> qresults = SearchIO.parse('tab_2226_tblastn_001.txt', 'blast-tab')
>>> for qresult in qresults:
...     print(qresult.id)
gi|16080617|ref|NP_391444.1|
gi|11464971:4-101
>>> qresults2 = SearchIO.parse('tab_2226_tblastn_005.txt', 'blast-tab', comments=True)
>>> for qresult in qresults2:
...     print(qresult.id)
random_s00
gi|16080617|ref|NP_391444.1|
gi|11464971:4-101
\end{verbatim}

\section{Dealing with large search output files with indexing}
\label{sec:searchio-index}

Sometimes, you're handed a search output file containing hundreds or thousands
of queries that you need to parse. You can of course use
\verb|Bio.SearchIO.parse| for this file, but that would be grossly inefficient
if you need to access only a few of the queries. This is because \verb|parse|
will parse all queries it sees before it fetches your query of interest.

In this case, the ideal choice would be to index the file using
\verb|Bio.SearchIO.index| or \verb|Bio.SearchIO.index_db|. If the names sound
familiar, it's because you've seen them before in  Section~\ref{sec:SeqIO-index}.
These functions also behave similarly to their \verb|Bio.SeqIO| counterparts,
with the addition of format-specific keyword arguments.

Here are some examples. You can use \verb|index| with just the filename and
format name:

%doctest ../Tests/Blast
\begin{verbatim}
>>> from Bio import SearchIO
>>> idx = SearchIO.index('tab_2226_tblastn_001.txt', 'blast-tab')
>>> sorted(idx.keys())
['gi|11464971:4-101', 'gi|16080617|ref|NP_391444.1|']
>>> idx['gi|16080617|ref|NP_391444.1|']
QueryResult(id='gi|16080617|ref|NP_391444.1|', 3 hits)
>>> idx.close()
\end{verbatim}

Or also with the format-specific keyword argument:

%cont-doctest
\begin{verbatim}
>>> idx = SearchIO.index('tab_2226_tblastn_005.txt', 'blast-tab', comments=True)
>>> sorted(idx.keys())
['gi|11464971:4-101', 'gi|16080617|ref|NP_391444.1|', 'random_s00']
>>> idx['gi|16080617|ref|NP_391444.1|']
QueryResult(id='gi|16080617|ref|NP_391444.1|', 3 hits)
>>> idx.close()
\end{verbatim}

Or with the \verb|key_function| argument, as in \verb|Bio.SeqIO|:

%cont-doctest
\begin{verbatim}
>>> key_function = lambda id: id.upper()    # capitalizes the keys
>>> idx = SearchIO.index('tab_2226_tblastn_001.txt', 'blast-tab', key_function=key_function)
>>> sorted(idx.keys())
['GI|11464971:4-101', 'GI|16080617|REF|NP_391444.1|']
>>> idx['GI|16080617|REF|NP_391444.1|']
QueryResult(id='gi|16080617|ref|NP_391444.1|', 3 hits)
>>> idx.close()
\end{verbatim}

\verb|Bio.SearchIO.index_db| works like as \verb|index|, only it writes the
query offsets into an SQLite database file.

\section{Writing and converting search output files}
\label{sec:searchio-write}

It is occasionally useful to be able to manipulate search results from an output
file and write it again to a new file. \verb|Bio.SearchIO| provides a
\verb|write| function that lets you do exactly this. It takes as its arguments
an iterable returning \verb|QueryResult| objects, the output filename to write
to, the format name to write to, and optionally some format-specific keyword
arguments. It returns a four-item tuple, which denotes the number or
\verb|QueryResult|, \verb|Hit|, \verb|HSP|, and \verb|HSPFragment| objects that
were written.

\begin{verbatim}
>>> from Bio import SearchIO
>>> qresults = SearchIO.parse('mirna.xml', 'blast-xml')     # read XML file
>>> SearchIO.write(qresults, 'results.tab', 'blast-tab')    # write to tabular file
(3, 239, 277, 277)
\end{verbatim}

You should note different file formats require different attributes of the
\verb|QueryResult|, \verb|Hit|, \verb|HSP| and \verb|HSPFragment| objects. If
these attributes are not present, writing won't work. In other words, you can't
always write to the output format that you want. For example, if you read a
BLAST XML file, you wouldn't be able to write the results to a PSL file as PSL
files require attributes not calculated by BLAST (e.g. the number of repeat
matches). You can always set these attributes manually, if you really want to
write to PSL, though.

Like \verb|read|, \verb|parse|, \verb|index|, and \verb|index_db|, \verb|write|
also accepts format-specific keyword arguments. Check out the documentation for
a complete list of formats \verb|Bio.SearchIO| can write to and their arguments.

Finally, \verb|Bio.SearchIO| also provides a \verb|convert| function, which is
simply a shortcut for \verb|Bio.SearchIO.parse| and \verb|Bio.SearchIO.write|.
Using the convert function, our example above would be:

\begin{verbatim}
>>> from Bio import SearchIO
>>> SearchIO.convert('mirna.xml', 'blast-xml', 'results.tab', 'blast-tab')
(3, 239, 277, 277)
\end{verbatim}

As \verb|convert| uses \verb|write|, it is only limited to format conversions
that have all the required attributes. Here, the BLAST XML file provides all the
default values a BLAST tabular file requires, so it works just fine. However,
other format conversions are less likely to work since you need to manually
assign the required attributes first.



%\chapter{Accessing NCBI's Entrez databases}
%\label{chapter:entrez}
\chapter{Accessing NCBI's Entrez databases}
\label{chapter:entrez}

Entrez (\url{http://www.ncbi.nlm.nih.gov/Entrez}) is a data retrieval system that provides users access to NCBI's databases such as PubMed, GenBank, GEO, and many others. You can access Entrez from a web browser to manually enter queries, or you can use Biopython's \verb+Bio.Entrez+ module for programmatic access to Entrez. The latter allows you for example to search PubMed or download GenBank records from within a Python script.

The \verb+Bio.Entrez+ module makes use of the Entrez Programming Utilities (also known as EUtils), consisting of eight tools that are described in detail on NCBI's page at \url{http://www.ncbi.nlm.nih.gov/entrez/utils/}.
Each of these tools corresponds to one Python function in the \verb+Bio.Entrez+ module, as described in the sections below. This module makes sure that the correct URL is used for the queries, and that not more than one request is made every three seconds, as required by NCBI.

The output returned by the Entrez Programming Utilities is typically in XML format. To parse such output, you have several options:
\begin{enumerate}
  \item Use \verb+Bio.Entrez+'s parser to parse the XML output into a Python object;
  \item Use the DOM (Document Object Model) parser in Python's standard library;
  \item Use the SAX (Simple API for XML) parser in Python's standard library;
  \item Read the XML output as raw text, and parse it by string searching and manipulation.
\end{enumerate}
For the DOM and SAX parsers, see the Python documentation. The parser in \verb+Bio.Entrez+ is discussed below.

NCBI uses DTD (Document Type Definition) files to describe the structure of the information contained in XML files. Most of the DTD files used by NCBI are included in the Biopython distribution. The \verb+Bio.Entrez+ parser makes use of the DTD files when parsing an XML file returned by NCBI Entrez.

Occasionally, you may find that the DTD file associated with a specific XML file is missing in the Biopython distribution. In particular, this may happen when NCBI updates its DTD files. If this happens, \verb+Entrez.read+ will show a warning message with the name and URL of the missing DTD file. The parser will proceed to access the missing DTD file through the internet, allowing the parsing of the XML file to continue.  However, the parser is much faster if the DTD file is available locally. For this purpose, please download the DTD file from the URL in the warning message and place it in the directory \verb+...site-packages/Bio/Entrez/DTDs+, containing the other DTD files.  If you don't have write access to this directory, you can also place the DTD file in \verb+~/.biopython/Bio/Entrez/DTDs+, where \verb+~+ represents your home directory. Since this directory is read before the directory \verb+...site-packages/Bio/Entrez/DTDs+, you can also put newer versions of DTD files there if the ones in \verb+...site-packages/Bio/Entrez/DTDs+ become outdated. Alternatively, if you installed Biopython from source, you can add the DTD file to the source code's \verb+Bio/Entrez/DTDs+ directory, and reinstall Biopython. This will install the new DTD file in the correct location together with the other DTD files.

The Entrez Programming Utilities can also generate output in other formats, such as the Fasta or GenBank file formats for sequence databases, or the MedLine format for the literature database, discussed in Section~\ref{sec:entrez-specialized-parsers}.

\section{Entrez Guidelines}
\label{sec:entrez-guidelines}
Before using Biopython to access the NCBI's online resources (via \verb|Bio.Entrez| or some of the other modules), please read the
\href{http://www.ncbi.nlm.nih.gov/books/NBK25497/#chapter2.Usage_Guidelines_and_Requiremen}{NCBI's Entrez User Requirements}.
If the NCBI finds you are abusing their systems, they can and will ban your access!

To paraphrase:

\begin{itemize}
\item For any series of more than 100 requests, do this at weekends or outside USA peak times.  This is up to you to obey.
\item Use the \url{http://eutils.ncbi.nlm.nih.gov} address, not the standard NCBI Web address.  Biopython uses this web address.
\item Make no more than three requests every seconds (relaxed from at most one request every three seconds in early 2009).  This is automatically enforced by Biopython.
\item Use the optional email parameter so the NCBI can contact you if there is a problem.  You can either explicitly set this as a parameter with each call to Entrez (e.g. include {\tt email="A.N.Other@example.com"} in the argument list), or you can set a global email address:
\begin{verbatim}
>>> from Bio import Entrez
>>> Entrez.email = "A.N.Other@example.com"
\end{verbatim}
{\tt Bio.Entrez} will then use this email address with each call to Entrez.  The {\tt example.com} address is a reserved domain name specifically for documentation (RFC 2606).  Please DO NOT use a random email -- it's better not to give an email at all. The email parameter has been mandatory since June 1, 2010. In case of excessive usage, NCBI will attempt to contact a user at the e-mail address provided prior to blocking access to the E-utilities.
\item If you are using Biopython within some larger software suite, use the tool parameter to specify this.  You can either explicitly set the tool name as a parameter with each call to Entrez (e.g. include {\tt tool="MyLocalScript"} in the argument list), or you can set a global tool name:
\begin{verbatim}
>>> from Bio import Entrez
>>> Entrez.tool = "MyLocalScript"
\end{verbatim}
The tool parameter will default to Biopython.
\item For large queries, the NCBI also recommend using their session history feature (the WebEnv session cookie string, see Section~\ref{sec:entrez-webenv}).  This is only slightly more complicated.
\end{itemize}

In conclusion, be sensible with your usage levels.  If you plan to download lots of data, consider other options.  For example, if you want easy access to all the human genes, consider fetching each chromosome by FTP as a GenBank file, and importing these into your own BioSQL database (see Section~\ref{sec:BioSQL}).

\section{EInfo: Obtaining information about the Entrez databases}
\label{sec:entrez-einfo}
EInfo provides field index term counts, last update, and available links for each of NCBI's databases. In addition, you can use EInfo to obtain a list of all database names accessible through the Entrez utilities:
\begin{verbatim}
>>> from Bio import Entrez
>>> Entrez.email = "A.N.Other@example.com"     # Always tell NCBI who you are
>>> handle = Entrez.einfo()
>>> result = handle.read()
\end{verbatim}
The variable \verb+result+ now contains a list of databases in XML format:
\begin{verbatim}
>>> print(result)
<?xml version="1.0"?>
<!DOCTYPE eInfoResult PUBLIC "-//NLM//DTD eInfoResult, 11 May 2002//EN"
 "http://www.ncbi.nlm.nih.gov/entrez/query/DTD/eInfo_020511.dtd">
<eInfoResult>
<DbList>
        <DbName>pubmed</DbName>
        <DbName>protein</DbName>
        <DbName>nucleotide</DbName>
        <DbName>nuccore</DbName>
        <DbName>nucgss</DbName>
        <DbName>nucest</DbName>
        <DbName>structure</DbName>
        <DbName>genome</DbName>
        <DbName>books</DbName>
        <DbName>cancerchromosomes</DbName>
        <DbName>cdd</DbName>
        <DbName>gap</DbName>
        <DbName>domains</DbName>
        <DbName>gene</DbName>
        <DbName>genomeprj</DbName>
        <DbName>gensat</DbName>
        <DbName>geo</DbName>
        <DbName>gds</DbName>
        <DbName>homologene</DbName>
        <DbName>journals</DbName>
        <DbName>mesh</DbName>
        <DbName>ncbisearch</DbName>
        <DbName>nlmcatalog</DbName>
        <DbName>omia</DbName>
        <DbName>omim</DbName>
        <DbName>pmc</DbName>
        <DbName>popset</DbName>
        <DbName>probe</DbName>
        <DbName>proteinclusters</DbName>
        <DbName>pcassay</DbName>
        <DbName>pccompound</DbName>
        <DbName>pcsubstance</DbName>
        <DbName>snp</DbName>
        <DbName>taxonomy</DbName>
        <DbName>toolkit</DbName>
        <DbName>unigene</DbName>
        <DbName>unists</DbName>
</DbList>
</eInfoResult>
\end{verbatim}

Since this is a fairly simple XML file, we could extract the information it contains simply by string searching. Using \verb+Bio.Entrez+'s parser instead, we can directly parse this XML file into a Python object:
\begin{verbatim}
>>> from Bio import Entrez
>>> handle = Entrez.einfo()
>>> record = Entrez.read(handle)
\end{verbatim}
Now \verb+record+ is a dictionary with exactly one key:
\begin{verbatim}
>>> record.keys()
[u'DbList']
\end{verbatim}
The values stored in this key is the list of database names shown in the XML above:
\begin{verbatim}
>>> record["DbList"]
['pubmed', 'protein', 'nucleotide', 'nuccore', 'nucgss', 'nucest',
 'structure', 'genome', 'books', 'cancerchromosomes', 'cdd', 'gap',
 'domains', 'gene', 'genomeprj', 'gensat', 'geo', 'gds', 'homologene',
 'journals', 'mesh', 'ncbisearch', 'nlmcatalog', 'omia', 'omim', 'pmc',
 'popset', 'probe', 'proteinclusters', 'pcassay', 'pccompound',
 'pcsubstance', 'snp', 'taxonomy', 'toolkit', 'unigene', 'unists']
\end{verbatim}

For each of these databases, we can use EInfo again to obtain more information:
\begin{verbatim}
>>> handle = Entrez.einfo(db="pubmed")
>>> record = Entrez.read(handle)
>>> record["DbInfo"]["Description"]
'PubMed bibliographic record'
>>> record["DbInfo"]["Count"]
'17989604'
>>> record["DbInfo"]["LastUpdate"]
'2008/05/24 06:45'
\end{verbatim}
Try \verb+record["DbInfo"].keys()+ for other information stored in this record.
One of the most useful is a list of possible search fields for use with ESearch:

\begin{verbatim}
>>> for field in record["DbInfo"]["FieldList"]:
...     print("%(Name)s, %(FullName)s, %(Description)s" % field)
ALL, All Fields, All terms from all searchable fields
UID, UID, Unique number assigned to publication
FILT, Filter, Limits the records
TITL, Title, Words in title of publication
WORD, Text Word, Free text associated with publication
MESH, MeSH Terms, Medical Subject Headings assigned to publication
MAJR, MeSH Major Topic, MeSH terms of major importance to publication
AUTH, Author, Author(s) of publication
JOUR, Journal, Journal abbreviation of publication
AFFL, Affiliation, Author's institutional affiliation and address
...
\end{verbatim}

That's a long list, but indirectly this tells you that for the PubMed
database, you can do things like \texttt{Jones[AUTH]} to search the
author field, or \texttt{Sanger[AFFL]} to restrict to authors at the
Sanger Centre. This can be very handy - especially if you are not so
familiar with a particular database.

\section{ESearch: Searching the Entrez databases}
\label{sec:entrez-esearch}
To search any of these databases, we use \verb+Bio.Entrez.esearch()+. For example, let's search in PubMed for publications related to Biopython:
\begin{verbatim}
>>> from Bio import Entrez
>>> Entrez.email = "A.N.Other@example.com"     # Always tell NCBI who you are
>>> handle = Entrez.esearch(db="pubmed", term="biopython")
>>> record = Entrez.read(handle)
>>> record["IdList"]
['19304878', '18606172', '16403221', '16377612', '14871861', '14630660', '12230038']
\end{verbatim}
In this output, you see seven PubMed IDs (including 19304878 which is the PMID for the Biopython application note), which can be retrieved by EFetch (see section \ref{sec:efetch}).

You can also use ESearch to search GenBank. Here we'll do a quick
search for the \emph{matK} gene in \emph{Cypripedioideae} orchids
(see Section~\ref{sec:entrez-einfo} about EInfo for one way to
find out which fields you can search in each Entrez database):

\begin{verbatim}
>>> handle = Entrez.esearch(db="nucleotide", term="Cypripedioideae[Orgn] AND matK[Gene]")
>>> record = Entrez.read(handle)
>>> record["Count"]
'25'
>>> record["IdList"]
['126789333', '37222967', '37222966', '37222965', ..., '61585492']
\end{verbatim}

\noindent Each of the IDs (126789333, 37222967, 37222966, \ldots) is a GenBank identifier.
See section~\ref{sec:efetch} for information on how to actually download these GenBank records.

Note that instead of a species name like \texttt{Cypripedioideae[Orgn]}, you can restrict the search using an NCBI taxon identifier, here this would be \texttt{txid158330[Orgn]}.  This isn't currently documented on the ESearch help page - the NCBI explained this in reply to an email query.  You can often deduce the search term formatting by playing with the Entrez web interface.  For example, including \texttt{complete[prop]} in a genome search restricts to just completed genomes.

As a final example, let's get a list of computational journal titles:
\begin{verbatim}
>>> handle = Entrez.esearch(db="nlmcatalog", term="computational[Journal]", retmax='20')
>>> record = Entrez.read(handle)
>>> print("{} computational journals found".format(record["Count"]))
117 computational Journals found
>>> print("The first 20 are\n{}".format(record['IdList']))
['101660833', '101664671', '101661657', '101659814', '101657941',
 '101653734', '101669877', '101649614', '101647835', '101639023',
 '101627224', '101647801', '101589678', '101585369', '101645372',
 '101586429', '101582229', '101574747', '101564639', '101671907']
\end{verbatim}
Again, we could use EFetch to obtain more information for each of these journal IDs.

ESearch has many useful options --- see the \href{https://www.ncbi.nlm.nih.gov/books/NBK25499/#chapter4.ESearch}{ESearch help page} for more information.

\section{EPost: Uploading a list of identifiers}
EPost uploads a list of UIs for use in subsequent search strategies; see the
\href{http://www.ncbi.nlm.nih.gov/entrez/query/static/epost\_help.html}{EPost help page} for more information. It is available from Biopython through
the \verb+Bio.Entrez.epost()+ function.

To give an example of when this is useful, suppose you have a long list of IDs
you want to download using EFetch (maybe sequences, maybe citations --
anything). When you make a  request with EFetch your list of IDs, the database
etc, are all turned into a long URL sent to the server.  If your list of IDs is
long, this URL gets long, and long URLs can break (e.g. some proxies don't
cope well).

Instead, you can break this up into two steps, first uploading the list of IDs
using EPost (this uses an ``HTML post'' internally, rather than an ``HTML get'',
getting round the long URL problem).  With the history support, you can then
refer to this long list of IDs, and download the associated data with EFetch.

Let's look at a simple example to see how EPost works -- uploading some PubMed identifiers:
\begin{verbatim}
>>> from Bio import Entrez
>>> Entrez.email = "A.N.Other@example.com"     # Always tell NCBI who you are
>>> id_list = ["19304878", "18606172", "16403221", "16377612", "14871861", "14630660"]
>>> print(Entrez.epost("pubmed", id=",".join(id_list)).read())
<?xml version="1.0"?>
<!DOCTYPE ePostResult PUBLIC "-//NLM//DTD ePostResult, 11 May 2002//EN"
 "http://www.ncbi.nlm.nih.gov/entrez/query/DTD/ePost_020511.dtd">
<ePostResult>
	<QueryKey>1</QueryKey>
	<WebEnv>NCID_01_206841095_130.14.22.101_9001_1242061629</WebEnv>
</ePostResult>
\end{verbatim}
\noindent The returned XML includes two important strings, \verb|QueryKey| and \verb|WebEnv| which together define your history session.
You would extract these values for use with another Entrez call such as EFetch:
\begin{verbatim}
>>> from Bio import Entrez
>>> Entrez.email = "A.N.Other@example.com"     # Always tell NCBI who you are
>>> id_list = ["19304878", "18606172", "16403221", "16377612", "14871861", "14630660"]
>>> search_results = Entrez.read(Entrez.epost("pubmed", id=",".join(id_list)))
>>> webenv = search_results["WebEnv"]
>>> query_key = search_results["QueryKey"]
\end{verbatim}
\noindent Section~\ref{sec:entrez-webenv} shows how to use the history feature.

\section{ESummary: Retrieving summaries from primary IDs}
ESummary retrieves document summaries from a list of primary IDs (see the  \href{http://www.ncbi.nlm.nih.gov/entrez/query/static/esummary\_help.html}{ESummary help page} for more information). In Biopython, ESummary is available as \verb+Bio.Entrez.esummary()+. Using the search result above, we can for example find out more about the journal with ID 30367:
\begin{verbatim}
>>> from Bio import Entrez
>>> Entrez.email = "A.N.Other@example.com"     # Always tell NCBI who you are
>>> handle = Entrez.esummary(db="nlmcatalog", id="101660833")
>>> record = Entrez.read(handle)
>>> info = record[0]['TitleMainList'][0]
>>> print("Journal info\nid: {}\nTitle: {}".format(record[0]["Id"], info["Title"]))
Journal info
id: 101660833
Title: IEEE transactions on computational imaging.
\end{verbatim}

\section{EFetch: Downloading full records from Entrez}
\label{sec:efetch}

EFetch is what you use when you want to retrieve a full record from Entrez.
This covers several possible databases, as described on the main \href{http://eutils.ncbi.nlm.nih.gov/entrez/query/static/efetch_help.html}{EFetch Help page}.

For most of their databases, the NCBI support several different file formats. Requesting a specific file format from Entrez using \verb|Bio.Entrez.efetch()| requires specifying the \verb|rettype| and/or \verb|retmode| optional arguments.  The different combinations are described for each database type on the pages linked to on \href{http://www.ncbi.nlm.nih.gov/entrez/query/static/efetch_help.html}{NCBI efetch webpage} (e.g. \href{http://eutils.ncbi.nlm.nih.gov/corehtml/query/static/efetchlit_help.html}{literature}, \href{http://eutils.ncbi.nlm.nih.gov/corehtml/query/static/efetchseq_help.html}{sequences} and \href{http://eutils.ncbi.nlm.nih.gov/corehtml/query/static/efetchtax_help.html}{taxonomy}).

One common usage is downloading sequences in the FASTA or GenBank/GenPept plain text formats (which can then be parsed with \verb|Bio.SeqIO|, see Sections~\ref{sec:SeqIO_GenBank_Online} and~\ref{sec:efetch}). From the \emph{Cypripedioideae} example above, we can download GenBank record 186972394 using \verb+Bio.Entrez.efetch+:

\begin{verbatim}
>>> from Bio import Entrez
>>> Entrez.email = "A.N.Other@example.com"     # Always tell NCBI who you are
>>> handle = Entrez.efetch(db="nucleotide", id="186972394", rettype="gb", retmode="text")
>>> print(handle.read())
LOCUS       EU490707                1302 bp    DNA     linear   PLN 05-MAY-2008
DEFINITION  Selenipedium aequinoctiale maturase K (matK) gene, partial cds;
            chloroplast.
ACCESSION   EU490707
VERSION     EU490707.1  GI:186972394
KEYWORDS    .
SOURCE      chloroplast Selenipedium aequinoctiale
  ORGANISM  Selenipedium aequinoctiale
            Eukaryota; Viridiplantae; Streptophyta; Embryophyta; Tracheophyta;
            Spermatophyta; Magnoliophyta; Liliopsida; Asparagales; Orchidaceae;
            Cypripedioideae; Selenipedium.
REFERENCE   1  (bases 1 to 1302)
  AUTHORS   Neubig,K.M., Whitten,W.M., Carlsward,B.S., Blanco,M.A.,
            Endara,C.L., Williams,N.H. and Moore,M.J.
  TITLE     Phylogenetic utility of ycf1 in orchids
  JOURNAL   Unpublished
REFERENCE   2  (bases 1 to 1302)
  AUTHORS   Neubig,K.M., Whitten,W.M., Carlsward,B.S., Blanco,M.A.,
            Endara,C.L., Williams,N.H. and Moore,M.J.
  TITLE     Direct Submission
  JOURNAL   Submitted (14-FEB-2008) Department of Botany, University of
            Florida, 220 Bartram Hall, Gainesville, FL 32611-8526, USA
FEATURES             Location/Qualifiers
     source          1..1302
                     /organism="Selenipedium aequinoctiale"
                     /organelle="plastid:chloroplast"
                     /mol_type="genomic DNA"
                     /specimen_voucher="FLAS:Blanco 2475"
                     /db_xref="taxon:256374"
     gene            <1..>1302
                     /gene="matK"
     CDS             <1..>1302
                     /gene="matK"
                     /codon_start=1
                     /transl_table=11
                     /product="maturase K"
                     /protein_id="ACC99456.1"
                     /db_xref="GI:186972395"
                     /translation="IFYEPVEIFGYDNKSSLVLVKRLITRMYQQNFLISSVNDSNQKG
                     FWGHKHFFSSHFSSQMVSEGFGVILEIPFSSQLVSSLEEKKIPKYQNLRSIHSIFPFL
                     EDKFLHLNYVSDLLIPHPIHLEILVQILQCRIKDVPSLHLLRLLFHEYHNLNSLITSK
                     KFIYAFSKRKKRFLWLLYNSYVYECEYLFQFLRKQSSYLRSTSSGVFLERTHLYVKIE
                     HLLVVCCNSFQRILCFLKDPFMHYVRYQGKAILASKGTLILMKKWKFHLVNFWQSYFH
                     FWSQPYRIHIKQLSNYSFSFLGYFSSVLENHLVVRNQMLENSFIINLLTKKFDTIAPV
                     ISLIGSLSKAQFCTVLGHPISKPIWTDFSDSDILDRFCRICRNLCRYHSGSSKKQVLY
                     RIKYILRLSCARTLARKHKSTVRTFMRRLGSGLLEEFFMEEE"
ORIGIN
        1 attttttacg aacctgtgga aatttttggt tatgacaata aatctagttt agtacttgtg
       61 aaacgtttaa ttactcgaat gtatcaacag aattttttga tttcttcggt taatgattct
      121 aaccaaaaag gattttgggg gcacaagcat tttttttctt ctcatttttc ttctcaaatg
      181 gtatcagaag gttttggagt cattctggaa attccattct cgtcgcaatt agtatcttct
      241 cttgaagaaa aaaaaatacc aaaatatcag aatttacgat ctattcattc aatatttccc
      301 tttttagaag acaaattttt acatttgaat tatgtgtcag atctactaat accccatccc
      361 atccatctgg aaatcttggt tcaaatcctt caatgccgga tcaaggatgt tccttctttg
      421 catttattgc gattgctttt ccacgaatat cataatttga atagtctcat tacttcaaag
      481 aaattcattt acgccttttc aaaaagaaag aaaagattcc tttggttact atataattct
      541 tatgtatatg aatgcgaata tctattccag tttcttcgta aacagtcttc ttatttacga
      601 tcaacatctt ctggagtctt tcttgagcga acacatttat atgtaaaaat agaacatctt
      661 ctagtagtgt gttgtaattc ttttcagagg atcctatgct ttctcaagga tcctttcatg
      721 cattatgttc gatatcaagg aaaagcaatt ctggcttcaa agggaactct tattctgatg
      781 aagaaatgga aatttcatct tgtgaatttt tggcaatctt attttcactt ttggtctcaa
      841 ccgtatagga ttcatataaa gcaattatcc aactattcct tctcttttct ggggtatttt
      901 tcaagtgtac tagaaaatca tttggtagta agaaatcaaa tgctagagaa ttcatttata
      961 ataaatcttc tgactaagaa attcgatacc atagccccag ttatttctct tattggatca
     1021 ttgtcgaaag ctcaattttg tactgtattg ggtcatccta ttagtaaacc gatctggacc
     1081 gatttctcgg attctgatat tcttgatcga ttttgccgga tatgtagaaa tctttgtcgt
     1141 tatcacagcg gatcctcaaa aaaacaggtt ttgtatcgta taaaatatat acttcgactt
     1201 tcgtgtgcta gaactttggc acggaaacat aaaagtacag tacgcacttt tatgcgaaga
     1261 ttaggttcgg gattattaga agaattcttt atggaagaag aa
//
\end{verbatim}

The arguments \verb+rettype="gb"+ and \verb+retmode="text"+ let us download this record in the GenBank format.

Note that until Easter 2009, the Entrez EFetch API let you use ``genbank'' as the
return type, however the NCBI now insist on using the official return types of
``gb'' or ``gbwithparts'' (or ``gp'' for proteins) as described on online.
Also not that until Feb 2012, the Entrez EFetch API would default to returning
plain text files, but now defaults to XML.

Alternatively, you could for example use \verb+rettype="fasta"+ to get the Fasta-format; see the \href{http://www.ncbi.nlm.nih.gov/entrez/query/static/efetchseq\_help.html}{EFetch Sequences Help page} for other options. Remember -- the available formats depend on which database you are downloading from - see the main \href{http://eutils.ncbi.nlm.nih.gov/entrez/query/static/efetch\_help.html}{EFetch Help page}.

If you fetch the record in one of the formats accepted by \verb+Bio.SeqIO+ (see Chapter~\ref{chapter:Bio.SeqIO}), you could directly parse it into a \verb+SeqRecord+:

\begin{verbatim}
>>> from Bio import Entrez, SeqIO
>>> handle = Entrez.efetch(db="nucleotide", id="186972394", rettype="gb", retmode="text")
>>> record = SeqIO.read(handle, "genbank")
>>> handle.close()
>>> print(record)
ID: EU490707.1
Name: EU490707
Description: Selenipedium aequinoctiale maturase K (matK) gene, partial cds; chloroplast.
Number of features: 3
...
Seq('ATTTTTTACGAACCTGTGGAAATTTTTGGTTATGACAATAAATCTAGTTTAGTA...GAA', IUPACAmbiguousDNA())
\end{verbatim}

Note that a more typical use would be to save the sequence data to a local file, and \emph{then} parse it with \verb|Bio.SeqIO|.  This can save you having to re-download the same file repeatedly while working on your script, and places less load on the NCBI's servers.  For example:

\begin{verbatim}
import os
from Bio import SeqIO
from Bio import Entrez
Entrez.email = "A.N.Other@example.com"  # Always tell NCBI who you are
filename = "gi_186972394.gbk"
if not os.path.isfile(filename):
    # Downloading...
    net_handle = Entrez.efetch(db="nucleotide", id="186972394", rettype="gb", retmode="text")
    out_handle = open(filename, "w")
    out_handle.write(net_handle.read())
    out_handle.close()
    net_handle.close()
    print("Saved")

print("Parsing...")
record = SeqIO.read(filename, "genbank")
print(record)
\end{verbatim}

To get the output in XML format, which you can parse using the \verb+Bio.Entrez.read()+ function, use \verb+retmode="xml"+:

\begin{verbatim}
>>> from Bio import Entrez
>>> handle = Entrez.efetch(db="nucleotide", id="186972394", retmode="xml")
>>> record = Entrez.read(handle)
>>> handle.close()
>>> record[0]["GBSeq_definition"]
'Selenipedium aequinoctiale maturase K (matK) gene, partial cds; chloroplast'
>>> record[0]["GBSeq_source"]
'chloroplast Selenipedium aequinoctiale'
\end{verbatim}

So, that dealt with sequences. For examples of parsing file formats specific to the other databases (e.g. the \verb+MEDLINE+ format used in PubMed), see Section~\ref{sec:entrez-specialized-parsers}.

If you want to perform a search with \verb|Bio.Entrez.esearch()|, and then download the records with \verb|Bio.Entrez.efetch()|, you should use the WebEnv history feature -- see Section~\ref{sec:entrez-webenv}.

\section{ELink: Searching for related items in NCBI Entrez}
\label{sec:elink}

ELink, available from Biopython as \verb+Bio.Entrez.elink()+, can be used to find related items in the NCBI Entrez databases. For example, you can us this to find nucleotide entries for an entry in the gene database,
and other cool stuff.

Let's use ELink to find articles related to the Biopython application note published in \textit{Bioinformatics} in 2009. The PubMed ID of this article is 19304878:

\begin{verbatim}
>>> from Bio import Entrez
>>> Entrez.email = "A.N.Other@example.com"
>>> pmid = "19304878"
>>> record = Entrez.read(Entrez.elink(dbfrom="pubmed", id=pmid))
\end{verbatim}

The \verb+record+ variable consists of a Python list, one for each database in which we searched. Since we specified only one PubMed ID to search for, \verb+record+ contains only one item. This item is a dictionary containing information about our search term, as well as all the related items that were found:

\begin{verbatim}
>>> record[0]["DbFrom"]
'pubmed'
>>> record[0]["IdList"]
['19304878']
\end{verbatim}

The \verb+"LinkSetDb"+ key contains the search results, stored as a list consisting of one item for each target database. In our search results, we only find hits in the PubMed database (although sub-divided into categories):

\begin{verbatim}
>>> len(record[0]["LinkSetDb"])
5
>>> for linksetdb in record[0]["LinkSetDb"]:
...     print(linksetdb["DbTo"], linksetdb["LinkName"], len(linksetdb["Link"]))
...
pubmed pubmed_pubmed 110
pubmed pubmed_pubmed_combined 6
pubmed pubmed_pubmed_five 6
pubmed pubmed_pubmed_reviews 5
pubmed pubmed_pubmed_reviews_five 5
\end{verbatim}

The actual search results are stored as under the \verb+"Link"+ key. In total, 110 items were found under
standard search.
Let's now at the first search result:
\begin{verbatim}
>>> record[0]["LinkSetDb"][0]["Link"][0]
{u'Id': '19304878'}
\end{verbatim}

\noindent This is the article we searched for, which doesn't help us much, so let's look at the second search result:

\begin{verbatim}
>>> record[0]["LinkSetDb"][0]["Link"][1]
{u'Id': '14630660'}
\end{verbatim}

\noindent This paper, with PubMed ID 14630660, is about the Biopython PDB parser.

We can use a loop to print out all PubMed IDs:
\begin{verbatim}
>>> for link in record[0]["LinkSetDb"][0]["Link"]:
...     print(link["Id"])
19304878
14630660
18689808
17121776
16377612
12368254
......
\end{verbatim}

Now that was nice, but personally I am often more interested to find out if a paper has been cited.
Well, ELink can do that too -- at least for journals in Pubmed Central (see Section~\ref{sec:elink-citations}).

For help on ELink, see the \href{http://www.ncbi.nlm.nih.gov/entrez/query/static/elink\_help.html}{ELink help page}.
There is an entire sub-page just for the \href{http://eutils.ncbi.nlm.nih.gov/corehtml/query/static/entrezlinks.html}{link names}, describing how different databases can be cross referenced.

\section{EGQuery: Global Query - counts for search terms}
EGQuery provides counts for a search term in each of the Entrez databases (i.e. a global query). This is particularly useful to find out how many items your search terms would find in each database without actually performing lots of separate searches with ESearch (see the example in \ref{subsec:entrez_example_genbank} below).

In this example, we use \verb+Bio.Entrez.egquery()+ to obtain the counts for ``Biopython'':

\begin{verbatim}
>>> from Bio import Entrez
>>> Entrez.email = "A.N.Other@example.com"     # Always tell NCBI who you are
>>> handle = Entrez.egquery(term="biopython")
>>> record = Entrez.read(handle)
>>> for row in record["eGQueryResult"]:
...     print(row["DbName"], row["Count"])
...
pubmed 6
pmc 62
journals 0
...
\end{verbatim}
See the \href{http://www.ncbi.nlm.nih.gov/entrez/query/static/egquery\_help.html}{EGQuery help page} for more information.

\section{ESpell: Obtaining spelling suggestions}
ESpell retrieves spelling suggestions. In this example, we use \verb+Bio.Entrez.espell()+ to obtain the correct spelling of Biopython:

\begin{verbatim}
>>> from Bio import Entrez
>>> Entrez.email = "A.N.Other@example.com"     # Always tell NCBI who you are
>>> handle = Entrez.espell(term="biopythooon")
>>> record = Entrez.read(handle)
>>> record["Query"]
'biopythooon'
>>> record["CorrectedQuery"]
'biopython'
\end{verbatim}
See the \href{http://www.ncbi.nlm.nih.gov/entrez/query/static/espell\_help.html}{ESpell help page} for more information.
The main use of this is for GUI tools to provide automatic suggestions for search terms.

\section{Parsing huge Entrez XML files}

The \verb+Entrez.read+ function reads the entire XML file returned by Entrez into a single Python object, which is kept in memory. To parse Entrez XML files too large to fit in memory, you can use the function \verb+Entrez.parse+. This is a generator function that reads records in the XML file one by one. This function is only useful if the XML file reflects a Python list object (in other words, if \verb+Entrez.read+ on a computer with infinite memory resources would return a Python list).

For example, you can download the entire Entrez Gene database for a given organism as a file from NCBI's ftp site. These files can be very large. As an example, on September 4, 2009, the file \verb+Homo_sapiens.ags.gz+, containing the Entrez Gene database for human, had a size of 116576 kB. This file, which is in the \verb+ASN+ format, can be converted into an XML file using NCBI's \verb+gene2xml+ program (see NCBI's ftp site for more information):

\begin{verbatim}
gene2xml -b T -i Homo_sapiens.ags -o Homo_sapiens.xml
\end{verbatim}

The resulting XML file has a size of 6.1 GB. Attempting \verb+Entrez.read+ on this file will result in a \verb+MemoryError+ on many computers.

The XML file \verb+Homo_sapiens.xml+ consists of a list of Entrez gene records, each corresponding to one Entrez gene in human. \verb+Entrez.parse+ retrieves these gene records one by one. You can then print out or store the relevant information in each record by iterating over the records. For example, this script iterates over the Entrez gene records and prints out the gene numbers and names for all current genes:

\begin{verbatim}
>>> from Bio import Entrez
>>> handle = open("Homo_sapiens.xml")
>>> records = Entrez.parse(handle)

>>> for record in records:
...     status = record['Entrezgene_track-info']['Gene-track']['Gene-track_status']
...     if status.attributes['value']=='discontinued':
...         continue
...     geneid = record['Entrezgene_track-info']['Gene-track']['Gene-track_geneid']
...     genename = record['Entrezgene_gene']['Gene-ref']['Gene-ref_locus']
...     print(geneid, genename)
\end{verbatim}

This will print:
\begin{verbatim}
1 A1BG
2 A2M
3 A2MP
8 AA
9 NAT1
10 NAT2
11 AACP
12 SERPINA3
13 AADAC
14 AAMP
15 AANAT
16 AARS
17 AAVS1
...
\end{verbatim}


\section{Handling errors}

Three things can go wrong when parsing an XML file:
\begin{itemize}
\item The file may not be an XML file to begin with;
\item The file may end prematurely or otherwise be corrupted;
\item The file may be correct XML, but contain items that are not represented in the associated DTD.
\end{itemize}

The first case occurs if, for example, you try to parse a Fasta file as if it were an XML file:
\begin{verbatim}
>>> from Bio import Entrez
>>> handle = open("NC_005816.fna") # a Fasta file
>>> record = Entrez.read(handle)
Traceback (most recent call last):
  File "<stdin>", line 1, in <module>
  File "/usr/local/lib/python2.7/site-packages/Bio/Entrez/__init__.py", line 257, in read
    record = handler.read(handle)
  File "/usr/local/lib/python2.7/site-packages/Bio/Entrez/Parser.py", line 164, in read
    raise NotXMLError(e)
Bio.Entrez.Parser.NotXMLError: Failed to parse the XML data (syntax error: line 1, column 0). Please make sure that the input data are in XML format.
\end{verbatim}
Here, the parser didn't find the \verb|<?xml ...| tag with which an XML file is supposed to start, and therefore decides (correctly) that the file is not an XML file.

When your file is in the XML format but is corrupted (for example, by ending prematurely), the parser will raise a CorruptedXMLError.
Here is an example of an XML file that ends prematurely:
\begin{verbatim}
<?xml version="1.0"?>
<!DOCTYPE eInfoResult PUBLIC "-//NLM//DTD eInfoResult, 11 May 2002//EN" "http://www.ncbi.nlm.nih.gov/entrez/query/DTD/eInfo_020511.dtd">
<eInfoResult>
<DbList>
        <DbName>pubmed</DbName>
        <DbName>protein</DbName>
        <DbName>nucleotide</DbName>
        <DbName>nuccore</DbName>
        <DbName>nucgss</DbName>
        <DbName>nucest</DbName>
        <DbName>structure</DbName>
        <DbName>genome</DbName>
        <DbName>books</DbName>
        <DbName>cancerchromosomes</DbName>
        <DbName>cdd</DbName>
\end{verbatim}
which will generate the following traceback:
\begin{verbatim}
>>> Entrez.read(handle)
Traceback (most recent call last):
  File "<stdin>", line 1, in <module>
  File "/usr/local/lib/python2.7/site-packages/Bio/Entrez/__init__.py", line 257, in read
    record = handler.read(handle)
  File "/usr/local/lib/python2.7/site-packages/Bio/Entrez/Parser.py", line 160, in read
    raise CorruptedXMLError(e)
Bio.Entrez.Parser.CorruptedXMLError: Failed to parse the XML data (no element found: line 16, column 0). Please make sure that the input data are not corrupted.

>>>
\end{verbatim}
Note that the error message tells you at what point in the XML file the error was detected.

The third type of error occurs if the XML file contains tags that do not have a description in the corresponding DTD file. This is an example of such an XML file:

\begin{verbatim}
<?xml version="1.0"?>
<!DOCTYPE eInfoResult PUBLIC "-//NLM//DTD eInfoResult, 11 May 2002//EN" "http://www.ncbi.nlm.nih.gov/entrez/query/DTD/eInfo_020511.dtd">
<eInfoResult>
        <DbInfo>
        <DbName>pubmed</DbName>
        <MenuName>PubMed</MenuName>
        <Description>PubMed bibliographic record</Description>
        <Count>20161961</Count>
        <LastUpdate>2010/09/10 04:52</LastUpdate>
        <FieldList>
                <Field>
...
                </Field>
        </FieldList>
        <DocsumList>
                <Docsum>
                        <DsName>PubDate</DsName>
                        <DsType>4</DsType>
                        <DsTypeName>string</DsTypeName>
                </Docsum>
                <Docsum>
                        <DsName>EPubDate</DsName>
...
        </DbInfo>
</eInfoResult>
\end{verbatim}

In this file, for some reason the tag \verb|<DocsumList>| (and several others) are not listed in the DTD file \verb|eInfo_020511.dtd|, which is specified on the second line as the DTD for this XML file. By default, the parser will stop and raise a ValidationError if it cannot find some tag in the DTD:

\begin{verbatim}
>>> from Bio import Entrez
>>> handle = open("einfo3.xml")
>>> record = Entrez.read(handle)
Traceback (most recent call last):
  File "<stdin>", line 1, in <module>
  File "/usr/local/lib/python2.7/site-packages/Bio/Entrez/__init__.py", line 257, in read
    record = handler.read(handle)
  File "/usr/local/lib/python2.7/site-packages/Bio/Entrez/Parser.py", line 154, in read
    self.parser.ParseFile(handle)
  File "/usr/local/lib/python2.7/site-packages/Bio/Entrez/Parser.py", line 246, in startElementHandler
    raise ValidationError(name)
Bio.Entrez.Parser.ValidationError: Failed to find tag 'DocsumList' in the DTD. To skip all tags that are not represented in the DTD, please call Bio.Entrez.read or Bio.Entrez.parse with validate=False.
\end{verbatim}
Optionally, you can instruct the parser to skip such tags instead of raising a ValidationError. This is done by calling \verb|Entrez.read| or \verb|Entrez.parse| with the argument \verb|validate| equal to False:
\begin{verbatim}
>>> from Bio import Entrez
>>> handle = open("einfo3.xml")
>>> record = Entrez.read(handle, validate=False)
>>>
\end{verbatim}
Of course, the information contained in the XML tags that are not in the DTD are not present in the record returned by \verb|Entrez.read|.


\section{Specialized parsers}
\label{sec:entrez-specialized-parsers}

The \verb|Bio.Entrez.read()| function can parse most (if not all) XML output returned by Entrez. Entrez typically allows you to retrieve records in other formats, which may have some advantages compared to the XML format in terms of readability (or download size).

To request a specific file format from Entrez using \verb|Bio.Entrez.efetch()| requires specifying the \verb|rettype| and/or \verb|retmode| optional arguments.  The different combinations are described for each database type on the \href{http://www.ncbi.nlm.nih.gov/entrez/query/static/efetch_help.html}{NCBI efetch webpage}.

One obvious case is you may prefer to download sequences in the FASTA or GenBank/GenPept plain text formats (which can then be parsed with \verb|Bio.SeqIO|, see Sections~\ref{sec:SeqIO_GenBank_Online} and~\ref{sec:efetch}).  For the literature databases, Biopython contains a parser for the \verb+MEDLINE+ format used in PubMed.

\subsection{Parsing Medline records}
\label{subsec:entrez-and-medline}
You can find the Medline parser in \verb+Bio.Medline+. Suppose we want to parse the file \verb+pubmed_result1.txt+, containing one Medline record. You can find this file in Biopython's \verb+Tests\Medline+ directory. The file looks like this:

\begin{verbatim}
PMID- 12230038
OWN - NLM
STAT- MEDLINE
DA  - 20020916
DCOM- 20030606
LR  - 20041117
PUBM- Print
IS  - 1467-5463 (Print)
VI  - 3
IP  - 3
DP  - 2002 Sep
TI  - The Bio* toolkits--a brief overview.
PG  - 296-302
AB  - Bioinformatics research is often difficult to do with commercial software. The
      Open Source BioPerl, BioPython and Biojava projects provide toolkits with
...
\end{verbatim}
We first open the file and then parse it:
%doctest ../Tests/Medline
\begin{verbatim}
>>> from Bio import Medline
>>> with open("pubmed_result1.txt") as handle:
...    record = Medline.read(handle)
...
\end{verbatim}
The \verb+record+ now contains the Medline record as a Python dictionary:
%cont-doctest
\begin{verbatim}
>>> record["PMID"]
'12230038'
\end{verbatim}
%TODO - doctest wrapping?
\begin{verbatim}
>>> record["AB"]
'Bioinformatics research is often difficult to do with commercial software.
The Open Source BioPerl, BioPython and Biojava projects provide toolkits with
multiple functionality that make it easier to create customised pipelines or
analysis. This review briefly compares the quirks of the underlying languages
and the functionality, documentation, utility and relative advantages of the
Bio counterparts, particularly from the point of view of the beginning
biologist programmer.'
\end{verbatim}
The key names used in a Medline record can be rather obscure; use
\begin{verbatim}
>>> help(record)
\end{verbatim}
for a brief summary.

To parse a file containing multiple Medline records, you can use the \verb+parse+ function instead:
%doctest ../Tests/Medline
\begin{verbatim}
>>> from Bio import Medline
>>> with open("pubmed_result2.txt") as handle:
...     for record in Medline.parse(handle):
...         print(record["TI"])
...
A high level interface to SCOP and ASTRAL implemented in python.
GenomeDiagram: a python package for the visualization of large-scale genomic data.
Open source clustering software.
PDB file parser and structure class implemented in Python.
\end{verbatim}

Instead of parsing Medline records stored in files, you can also parse Medline records downloaded by \verb+Bio.Entrez.efetch+. For example, let's look at all Medline records in PubMed related to Biopython:
\begin{verbatim}
>>> from Bio import Entrez
>>> Entrez.email = "A.N.Other@example.com"     # Always tell NCBI who you are
>>> handle = Entrez.esearch(db="pubmed", term="biopython")
>>> record = Entrez.read(handle)
>>> record["IdList"]
['19304878', '18606172', '16403221', '16377612', '14871861', '14630660', '12230038']
\end{verbatim}
We now use \verb+Bio.Entrez.efetch+ to download these Medline records:
\begin{verbatim}
>>> idlist = record["IdList"]
>>> handle = Entrez.efetch(db="pubmed", id=idlist, rettype="medline", retmode="text")
\end{verbatim}
Here, we specify \verb+rettype="medline", retmode="text"+ to obtain the Medline records in plain-text Medline format. Now we use \verb+Bio.Medline+ to parse these records:
\begin{verbatim}
>>> from Bio import Medline
>>> records = Medline.parse(handle)
>>> for record in records:
...     print(record["AU"])
['Cock PJ', 'Antao T', 'Chang JT', 'Chapman BA', 'Cox CJ', 'Dalke A', ..., 'de Hoon MJ']
['Munteanu CR', 'Gonzalez-Diaz H', 'Magalhaes AL']
['Casbon JA', 'Crooks GE', 'Saqi MA']
['Pritchard L', 'White JA', 'Birch PR', 'Toth IK']
['de Hoon MJ', 'Imoto S', 'Nolan J', 'Miyano S']
['Hamelryck T', 'Manderick B']
['Mangalam H']
\end{verbatim}

For comparison, here we show an example using the XML format:
\begin{verbatim}
>>> idlist = record["IdList"]
>>> handle = Entrez.efetch(db="pubmed", id=idlist, rettype="medline", retmode="xml")
>>> records = Entrez.read(handle)
>>> for record in records:
...     print(record["MedlineCitation"]["Article"]["ArticleTitle"])
Biopython: freely available Python tools for computational molecular biology and
 bioinformatics.
Enzymes/non-enzymes classification model complexity based on composition, sequence,
 3D and topological indices.
A high level interface to SCOP and ASTRAL implemented in python.
GenomeDiagram: a python package for the visualization of large-scale genomic data.
Open source clustering software.
PDB file parser and structure class implemented in Python.
The Bio* toolkits--a brief overview.
\end{verbatim}

Note that in both of these examples, for simplicity we have naively combined ESearch and EFetch.
In this situation, the NCBI would expect you to use their history feature,
as illustrated in Section~\ref{sec:entrez-webenv}.


\subsection{Parsing GEO records}

GEO (\href{http://www.ncbi.nlm.nih.gov/geo/}{Gene Expression Omnibus})
is a data repository of high-throughput gene expression and hybridization
array data. The \verb|Bio.Geo| module can be used to parse GEO-formatted
data.

The following code fragment shows how to parse the example GEO file
\verb|GSE16.txt| into a record and print the record:

\begin{verbatim}
>>> from Bio import Geo
>>> handle = open("GSE16.txt")
>>> records = Geo.parse(handle)
>>> for record in records:
...     print(record)
\end{verbatim}

You can search the ``gds'' database (GEO datasets) with ESearch:

\begin{verbatim}
>>> from Bio import Entrez
>>> Entrez.email = "A.N.Other@example.com" # Always tell NCBI who you are
>>> handle = Entrez.esearch(db="gds", term="GSE16")
>>> record = Entrez.read(handle)
>>> record["Count"]
2
>>> record["IdList"]
['200000016', '100000028']
\end{verbatim}

From the Entrez website, UID ``200000016'' is GDS16 while the other hit
``100000028'' is for the associated platform, GPL28.  Unfortunately, at the
time of writing the NCBI don't seem to support downloading GEO files using
Entrez (not as XML, nor in the \textit{Simple Omnibus Format in Text} (SOFT)
format).

However, it is actually pretty straight forward to download the GEO files by FTP
from \url{ftp://ftp.ncbi.nih.gov/pub/geo/} instead.  In this case you might want
\url{ftp://ftp.ncbi.nih.gov/pub/geo/DATA/SOFT/by_series/GSE16/GSE16_family.soft.gz}
(a compressed file, see the Python module gzip).

\subsection{Parsing UniGene records}

UniGene is an NCBI database of the transcriptome, with each UniGene record showing the set of transcripts that are associated with a particular gene in a specific organism. A typical UniGene record looks like this:

\begin{verbatim}
ID          Hs.2
TITLE       N-acetyltransferase 2 (arylamine N-acetyltransferase)
GENE        NAT2
CYTOBAND    8p22
GENE_ID     10
LOCUSLINK   10
HOMOL       YES
EXPRESS      bone| connective tissue| intestine| liver| liver tumor| normal| soft tissue/muscle tissue tumor| adult
RESTR_EXPR   adult
CHROMOSOME  8
STS         ACC=PMC310725P3 UNISTS=272646
STS         ACC=WIAF-2120 UNISTS=44576
STS         ACC=G59899 UNISTS=137181
...
STS         ACC=GDB:187676 UNISTS=155563
PROTSIM     ORG=10090; PROTGI=6754794; PROTID=NP_035004.1; PCT=76.55; ALN=288
PROTSIM     ORG=9796; PROTGI=149742490; PROTID=XP_001487907.1; PCT=79.66; ALN=288
PROTSIM     ORG=9986; PROTGI=126722851; PROTID=NP_001075655.1; PCT=76.90; ALN=288
...
PROTSIM     ORG=9598; PROTGI=114619004; PROTID=XP_519631.2; PCT=98.28; ALN=288

SCOUNT      38
SEQUENCE    ACC=BC067218.1; NID=g45501306; PID=g45501307; SEQTYPE=mRNA
SEQUENCE    ACC=NM_000015.2; NID=g116295259; PID=g116295260; SEQTYPE=mRNA
SEQUENCE    ACC=D90042.1; NID=g219415; PID=g219416; SEQTYPE=mRNA
SEQUENCE    ACC=D90040.1; NID=g219411; PID=g219412; SEQTYPE=mRNA
SEQUENCE    ACC=BC015878.1; NID=g16198419; PID=g16198420; SEQTYPE=mRNA
SEQUENCE    ACC=CR407631.1; NID=g47115198; PID=g47115199; SEQTYPE=mRNA
SEQUENCE    ACC=BG569293.1; NID=g13576946; CLONE=IMAGE:4722596; END=5'; LID=6989; SEQTYPE=EST; TRACE=44157214
...
SEQUENCE    ACC=AU099534.1; NID=g13550663; CLONE=HSI08034; END=5'; LID=8800; SEQTYPE=EST
//
\end{verbatim}

This particular record shows the set of transcripts (shown in the \verb+SEQUENCE+ lines) that originate from the human gene NAT2, encoding en N-acetyltransferase. The \verb+PROTSIM+ lines show proteins with significant similarity to NAT2, whereas the \verb+STS+ lines show the corresponding sequence-tagged sites in the genome.

To parse UniGene files, use the \verb+Bio.UniGene+ module:
\begin{verbatim}
>>> from Bio import UniGene
>>> input = open("myunigenefile.data")
>>> record = UniGene.read(input)
\end{verbatim}

The \verb+record+ returned by \verb+UniGene.read+ is a Python object with attributes corresponding to the fields in the UniGene record. For example,
\begin{verbatim}
>>> record.ID
"Hs.2"
>>> record.title
"N-acetyltransferase 2 (arylamine N-acetyltransferase)"
\end{verbatim}

The \verb+EXPRESS+ and \verb+RESTR_EXPR+ lines are stored as Python lists of strings:
\begin{verbatim}
['bone', 'connective tissue', 'intestine', 'liver', 'liver tumor', 'normal', 'soft tissue/muscle tissue tumor', 'adult']
\end{verbatim}

Specialized objects are returned for the \verb+STS+, \verb+PROTSIM+, and \verb+SEQUENCE+ lines, storing the keys shown in each line as attributes:
\begin{verbatim}
>>> record.sts[0].acc
'PMC310725P3'
>>> record.sts[0].unists
'272646'
\end{verbatim}
and similarly for the \verb+PROTSIM+ and \verb+SEQUENCE+ lines.

To parse a file containing more than one UniGene record, use the \verb+parse+ function in \verb+Bio.UniGene+:

\begin{verbatim}
>>> from Bio import UniGene
>>> input = open("unigenerecords.data")
>>> records = UniGene.parse(input)
>>> for record in records:
...     print(record.ID)
\end{verbatim}

\section{Using a proxy}

Normally you won't have to worry about using a proxy, but if this is an issue
on your network here is how to deal with it.  Internally, \verb|Bio.Entrez|
uses the standard Python library \verb|urllib| for accessing the NCBI servers.
This will check an environment variable called \verb|http_proxy| to configure
any simple proxy automatically.  Unfortunately this module does not support
the use of proxies which require authentication.

You may choose to set the \verb|http_proxy| environment variable once (how you
do this will depend on your operating system).  Alternatively you can set this
within Python at the start of your script, for example:

\begin{verbatim}
import os
os.environ["http_proxy"] = "http://proxyhost.example.com:8080"
\end{verbatim}

\noindent See the \href{http://www.python.org/doc/lib/module-urllib.html}
{urllib documentation} for more details.

\section{Examples}
\label{sec:entrez_examples}

\subsection{PubMed and Medline}
\label{subsec:pub_med}

If you are in the medical field or interested in human issues (and many times even if you are not!), PubMed (\url{http://www.ncbi.nlm.nih.gov/PubMed/}) is an excellent source of all kinds of goodies. So like other things, we'd like to be able to grab information from it and use it in Python scripts.

In this example, we will query PubMed for all articles having to do with orchids (see section~\ref{sec:orchids} for our motivation). We first check how many of such articles there are:

\begin{verbatim}
>>> from Bio import Entrez
>>> Entrez.email = "A.N.Other@example.com"     # Always tell NCBI who you are
>>> handle = Entrez.egquery(term="orchid")
>>> record = Entrez.read(handle)
>>> for row in record["eGQueryResult"]:
...     if row["DbName"]=="pubmed":
...         print(row["Count"])
463
\end{verbatim}

Now we use the \verb+Bio.Entrez.efetch+ function to download the PubMed IDs of these 463 articles:
\begin{verbatim}
>>> handle = Entrez.esearch(db="pubmed", term="orchid", retmax=463)
>>> record = Entrez.read(handle)
>>> idlist = record["IdList"]
>>> print(idlist)
\end{verbatim}


This returns a Python list containing all of the PubMed IDs of articles related to orchids:
\begin{verbatim}
['18680603', '18665331', '18661158', '18627489', '18627452', '18612381',
'18594007', '18591784', '18589523', '18579475', '18575811', '18575690',
...
\end{verbatim}

Now that we've got them, we obviously want to get the corresponding Medline records and extract the information from them. Here, we'll download the Medline records in the Medline flat-file format, and use the \verb+Bio.Medline+ module to parse them:
\begin{verbatim}
>>> from Bio import Medline
>>> handle = Entrez.efetch(db="pubmed", id=idlist, rettype="medline",
                           retmode="text")
>>> records = Medline.parse(handle)
\end{verbatim}

NOTE - We've just done a separate search and fetch here, the NCBI much prefer you to take advantage of their history support in this situation.  See Section~\ref{sec:entrez-webenv}.

Keep in mind that \verb+records+ is an iterator, so you can iterate through the records only once. If you want to save the records, you can convert them to a list:
\begin{verbatim}
>>> records = list(records)
\end{verbatim}

Let's now iterate over the records to print out some information about each record:
%TODO - Replace the print blank line with print()?
\begin{verbatim}
>>> for record in records:
...     print("title:", record.get("TI", "?"))
...     print("authors:", record.get("AU", "?"))
...     print("source:", record.get("SO", "?"))
...     print("")
\end{verbatim}

The output for this looks like:
\begin{verbatim}
title: Sex pheromone mimicry in the early spider orchid (ophrys sphegodes):
patterns of hydrocarbons as the key mechanism for pollination by sexual
deception [In Process Citation]
authors: ['Schiestl FP', 'Ayasse M', 'Paulus HF', 'Lofstedt C', 'Hansson BS',
'Ibarra F', 'Francke W']
source: J Comp Physiol [A] 2000 Jun;186(6):567-74
\end{verbatim}

Especially interesting to note is the list of authors, which is returned as a standard Python list. This makes it easy to manipulate and search using standard Python tools. For instance, we could loop through a whole bunch of entries searching for a particular author with code like the following:
\begin{verbatim}
>>> search_author = "Waits T"

>>> for record in records:
...     if not "AU" in record:
...         continue
...     if search_author in record["AU"]:
...         print("Author %s found: %s" % (search_author, record["SO"]))
\end{verbatim}

Hopefully this section gave you an idea of the power and flexibility of the Entrez and Medline interfaces and how they can be used together.

\subsection{Searching, downloading, and parsing Entrez Nucleotide records}
\label{subsec:entrez_example_genbank}

Here we'll show a simple example of performing a remote Entrez query. In section~\ref{sec:orchids} of the parsing examples, we talked about using NCBI's Entrez website to search the NCBI nucleotide databases for info on Cypripedioideae, our friends the lady slipper orchids. Now, we'll look at how to automate that process using a Python script. In this example, we'll just show how to connect, get the results, and parse them, with the Entrez module doing all of the work.

First, we use EGQuery to find out the number of results we will get before actually downloading them.  EGQuery will tell us how many search results were found in each of the databases, but for this example we are only interested in nucleotides:
\begin{verbatim}
>>> from Bio import Entrez
>>> Entrez.email = "A.N.Other@example.com"     # Always tell NCBI who you are
>>> handle = Entrez.egquery(term="Cypripedioideae")
>>> record = Entrez.read(handle)
>>> for row in record["eGQueryResult"]:
...     if row["DbName"]=="nuccore":
...         print(row["Count"])
814
\end{verbatim}

So, we expect to find 814 Entrez Nucleotide records (this is the number I obtained in 2008; it is likely to increase in the future). If you find some ridiculously high number of hits, you may want to reconsider if you really want to download all of them, which is our next step:
\begin{verbatim}
>>> from Bio import Entrez
>>> handle = Entrez.esearch(db="nucleotide", term="Cypripedioideae", retmax=814)
>>> record = Entrez.read(handle)
\end{verbatim}

Here, \verb+record+ is a Python dictionary containing the search results and some auxiliary information. Just for information, let's look at what is stored in this dictionary:
\begin{verbatim}
>>> print(record.keys())
[u'Count', u'RetMax', u'IdList', u'TranslationSet', u'RetStart', u'QueryTranslation']
\end{verbatim}
First, let's check how many results were found:
\begin{verbatim}
>>> print(record["Count"])
'814'
\end{verbatim}
which is the number we expected. The 814 results are stored in \verb+record['IdList']+:
\begin{verbatim}
>>> len(record["IdList"])
814
\end{verbatim}
Let's look at the first five results:
\begin{verbatim}
>>> record["IdList"][:5]
['187237168', '187372713', '187372690', '187372688', '187372686']
\end{verbatim}

\label{sec:entrez-batched-efetch}
We can download these records using \verb+efetch+.
While you could download these records one by one, to reduce the load on NCBI's servers, it is better to fetch a bunch of records at the same time, shown below.
However, in this situation you should ideally be using the history feature described later in Section~\ref{sec:entrez-webenv}.

\begin{verbatim}
>>> idlist = ",".join(record["IdList"][:5])
>>> print(idlist)
187237168,187372713,187372690,187372688,187372686
>>> handle = Entrez.efetch(db="nucleotide", id=idlist, retmode="xml")
>>> records = Entrez.read(handle)
>>> len(records)
5
\end{verbatim}
Each of these records corresponds to one GenBank record.
\begin{verbatim}
>>> print(records[0].keys())
[u'GBSeq_moltype', u'GBSeq_source', u'GBSeq_sequence',
 u'GBSeq_primary-accession', u'GBSeq_definition', u'GBSeq_accession-version',
 u'GBSeq_topology', u'GBSeq_length', u'GBSeq_feature-table',
 u'GBSeq_create-date', u'GBSeq_other-seqids', u'GBSeq_division',
 u'GBSeq_taxonomy', u'GBSeq_references', u'GBSeq_update-date',
 u'GBSeq_organism', u'GBSeq_locus', u'GBSeq_strandedness']

>>> print(records[0]["GBSeq_primary-accession"])
DQ110336

>>> print(records[0]["GBSeq_other-seqids"])
['gb|DQ110336.1|', 'gi|187237168']

>>> print(records[0]["GBSeq_definition"])
Cypripedium calceolus voucher Davis 03-03 A maturase (matR) gene, partial cds;
mitochondrial

>>> print(records[0]["GBSeq_organism"])
Cypripedium calceolus
\end{verbatim}

You could use this to quickly set up searches -- but for heavy usage, see Section~\ref{sec:entrez-webenv}.

\subsection{Searching, downloading, and parsing GenBank records}
\label{sec:entrez-search-fetch-genbank}

The GenBank record format is a very popular method of holding information about sequences, sequence features, and other associated sequence information. The format is a good way to get information from the NCBI databases at \url{http://www.ncbi.nlm.nih.gov/}.

In this example we'll show how to query the NCBI databases,to retrieve the records from the query, and then parse them using \verb+Bio.SeqIO+  - something touched on in Section~\ref{sec:SeqIO_GenBank_Online}.
For simplicity, this example \emph{does not} take advantage of the WebEnv history feature -- see Section~\ref{sec:entrez-webenv} for this.

First, we want to make a query and find out the ids of the records to retrieve. Here we'll do a quick search for one of our favorite organisms, \emph{Opuntia} (prickly-pear cacti). We can do quick search and get back the GIs (GenBank identifiers) for all of the corresponding records. First we check how many records there are:

\begin{verbatim}
>>> from Bio import Entrez
>>> Entrez.email = "A.N.Other@example.com"     # Always tell NCBI who you are
>>> handle = Entrez.egquery(term="Opuntia AND rpl16")
>>> record = Entrez.read(handle)
>>> for row in record["eGQueryResult"]:
...     if row["DbName"]=="nuccore":
...         print(row["Count"])
...
9
\end{verbatim}
Now we download the list of GenBank identifiers:
\begin{verbatim}
>>> handle = Entrez.esearch(db="nuccore", term="Opuntia AND rpl16")
>>> record = Entrez.read(handle)
>>> gi_list = record["IdList"]
>>> gi_list
['57240072', '57240071', '6273287', '6273291', '6273290', '6273289', '6273286',
'6273285', '6273284']
\end{verbatim}

Now we use these GIs to download the GenBank records - note that with older versions of Biopython you had to supply a comma separated list of GI numbers to Entrez, as of Biopython 1.59 you can pass a list and this is converted for you:

\begin{verbatim}
>>> gi_str = ",".join(gi_list)
>>> handle = Entrez.efetch(db="nuccore", id=gi_str, rettype="gb", retmode="text")
\end{verbatim}

If you want to look at the raw GenBank files, you can read from this handle and print out the result:

\begin{verbatim}
>>> text = handle.read()
>>> print(text)
LOCUS       AY851612                 892 bp    DNA     linear   PLN 10-APR-2007
DEFINITION  Opuntia subulata rpl16 gene, intron; chloroplast.
ACCESSION   AY851612
VERSION     AY851612.1  GI:57240072
KEYWORDS    .
SOURCE      chloroplast Austrocylindropuntia subulata
  ORGANISM  Austrocylindropuntia subulata
            Eukaryota; Viridiplantae; Streptophyta; Embryophyta; Tracheophyta;
            Spermatophyta; Magnoliophyta; eudicotyledons; core eudicotyledons;
            Caryophyllales; Cactaceae; Opuntioideae; Austrocylindropuntia.
REFERENCE   1  (bases 1 to 892)
  AUTHORS   Butterworth,C.A. and Wallace,R.S.
...
\end{verbatim}

In this case, we are just getting the raw records. To get the records in a more Python-friendly form, we can use \verb+Bio.SeqIO+ to parse the GenBank data into \verb|SeqRecord| objects, including \verb|SeqFeature| objects (see Chapter~\ref{chapter:Bio.SeqIO}):

\begin{verbatim}
>>> from Bio import SeqIO
>>> handle = Entrez.efetch(db="nuccore", id=gi_str, rettype="gb", retmode="text")
>>> records = SeqIO.parse(handle, "gb")
\end{verbatim}

\noindent We can now step through the records and look at the information we are interested in:
\begin{verbatim}
>>> for record in records:
>>> ...    print("%s, length %i, with %i features" \
>>> ...           % (record.name, len(record), len(record.features)))
AY851612, length 892, with 3 features
AY851611, length 881, with 3 features
AF191661, length 895, with 3 features
AF191665, length 902, with 3 features
AF191664, length 899, with 3 features
AF191663, length 899, with 3 features
AF191660, length 893, with 3 features
AF191659, length 894, with 3 features
AF191658, length 896, with 3 features
\end{verbatim}

Using these automated query retrieval functionality is a big plus over doing things by hand.   Although the module should obey the NCBI's max three queries per second rule, the NCBI have other recommendations like avoiding peak hours.  See Section~\ref{sec:entrez-guidelines}.
In particular, please note that for simplicity, this example does not use the WebEnv history feature.  You should use this for any non-trivial search and download work, see Section~\ref{sec:entrez-webenv}.

Finally, if plan to repeat your analysis, rather than downloading the files from the NCBI and parsing them immediately (as shown in this example), you should just download the records \emph{once} and save them to your hard disk, and then parse the local file.

\subsection{Finding the lineage of an organism}

Staying with a plant example, let's now find the lineage of the Cypripedioideae orchid family. First, we search the Taxonomy database for Cypripedioideae, which yields exactly one NCBI taxonomy identifier:
\begin{verbatim}
>>> from Bio import Entrez
>>> Entrez.email = "A.N.Other@example.com"     # Always tell NCBI who you are
>>> handle = Entrez.esearch(db="Taxonomy", term="Cypripedioideae")
>>> record = Entrez.read(handle)
>>> record["IdList"]
['158330']
>>> record["IdList"][0]
'158330'
\end{verbatim}
Now, we use \verb+efetch+ to download this entry in the Taxonomy database, and then parse it:
\begin{verbatim}
>>> handle = Entrez.efetch(db="Taxonomy", id="158330", retmode="xml")
>>> records = Entrez.read(handle)
\end{verbatim}
Again, this record stores lots of information:
\begin{verbatim}
>>> records[0].keys()
[u'Lineage', u'Division', u'ParentTaxId', u'PubDate', u'LineageEx',
 u'CreateDate', u'TaxId', u'Rank', u'GeneticCode', u'ScientificName',
 u'MitoGeneticCode', u'UpdateDate']
\end{verbatim}
We can get the lineage directly from this record:
\begin{verbatim}
>>> records[0]["Lineage"]
'cellular organisms; Eukaryota; Viridiplantae; Streptophyta; Streptophytina;
 Embryophyta; Tracheophyta; Euphyllophyta; Spermatophyta; Magnoliophyta;
 Liliopsida; Asparagales; Orchidaceae'
\end{verbatim}

The record data contains much more than just the information shown here - for example look under \texttt{"LineageEx"} instead of \texttt{"Lineage"} and you'll get the NCBI taxon identifiers of the lineage entries too.

\section{Using the history and WebEnv}
\label{sec:entrez-webenv}

Often you will want to make a series of linked queries.  Most typically,
running a search, perhaps refining the search, and then retrieving detailed
search results.  You \emph{can} do this by making a series of separate calls
to Entrez.  However, the NCBI prefer you to take advantage of their history
support - for example combining ESearch and EFetch.

Another typical use of the history support would be to combine EPost and
EFetch.  You use EPost to upload a list of identifiers, which starts a new
history session.  You then download the records with EFetch by referring
to the session (instead of the identifiers).

\subsection{Searching for and downloading sequences using the history}
Suppose we want to search and download all the \textit{Opuntia} rpl16
nucleotide sequences, and store them in a FASTA file.  As shown in
Section~\ref{sec:entrez-search-fetch-genbank}, we can naively combine
\verb|Bio.Entrez.esearch()| to get a list of GI numbers, and then call
\verb|Bio.Entrez.efetch()| to download them all.

However, the approved approach is to run the search with the history
feature.  Then, we can fetch the results by reference to the search
results - which the NCBI can anticipate and cache.

To do this, call \verb|Bio.Entrez.esearch()| as normal, but with the
additional argument of \verb|usehistory="y"|,

\begin{verbatim}
>>> from Bio import Entrez
>>> Entrez.email = "history.user@example.com"
>>> search_handle = Entrez.esearch(db="nucleotide",term="Opuntia[orgn] and rpl16",
                                   usehistory="y")
>>> search_results = Entrez.read(search_handle)
>>> search_handle.close()
\end{verbatim}

\noindent When you get the XML output back, it will still include the usual search results:

\begin{verbatim}
>>> gi_list = search_results["IdList"]
>>> count = int(search_results["Count"])
>>> assert count == len(gi_list)
\end{verbatim}

\noindent However, you also get given two additional pieces of information, the {\tt WebEnv} session cookie, and the {\tt QueryKey}:

\begin{verbatim}
>>> webenv = search_results["WebEnv"]
>>> query_key = search_results["QueryKey"]
\end{verbatim}

Having stored these values in variables {\tt session\_cookie} and {\tt query\_key} we can use them as parameters to \verb|Bio.Entrez.efetch()| instead of giving the GI numbers as identifiers.

While for small searches you might be OK downloading everything at once, it is better to download in batches.  You use the {\tt retstart} and {\tt retmax} parameters to specify which range of search results you want returned (starting entry using zero-based counting, and maximum number of results to return).  Sometimes you will get intermittent errors from Entrez, HTTPError 5XX, we use a try except pause retry block to address this.
For example,

\begin{verbatim}
# This assumes you have already run a search as shown above,
# and set the variables count, webenv, query_key

try:
    from urllib.error import HTTPError  # for Python 3
except ImportError:
    from urllib2 import HTTPError  # for Python 2

batch_size = 3
out_handle = open("orchid_rpl16.fasta", "w")
for start in range(0, count, batch_size):
    end = min(count, start+batch_size)
    print("Going to download record %i to %i" % (start+1, end))
    attempt = 0
    while attempt < 3:
        attempt += 1
        try:
            fetch_handle = Entrez.efetch(db="nucleotide",
                                         rettype="fasta", retmode="text",
                                         retstart=start, retmax=batch_size,
                                         webenv=webenv, query_key=query_key)
        except HTTPError as err:
            if 500 <= err.code <= 599:
                print("Received error from server %s" % err)
                print("Attempt %i of 3" % attempt)
                time.sleep(15)
            else:
                raise
    data = fetch_handle.read()
    fetch_handle.close()
    out_handle.write(data)
out_handle.close()
\end{verbatim}

\noindent For illustrative purposes, this example downloaded the FASTA records in batches of three.  Unless you are downloading genomes or chromosomes, you would normally pick a larger batch size.

\subsection{Searching for and downloading abstracts using the history}
Here is another history example, searching for papers published in the last year about the \textit{Opuntia}, and then downloading them into a file in MedLine format:

\begin{verbatim}
from Bio import Entrez
import time
try:
    from urllib.error import HTTPError  # for Python 3
except ImportError:
    from urllib2 import HTTPError  # for Python 2
Entrez.email = "history.user@example.com"
search_results = Entrez.read(Entrez.esearch(db="pubmed",
                                            term="Opuntia[ORGN]",
                                            reldate=365, datetype="pdat",
                                            usehistory="y"))
count = int(search_results["Count"])
print("Found %i results" % count)

batch_size = 10
out_handle = open("recent_orchid_papers.txt", "w")
for start in range(0,count,batch_size):
    end = min(count, start+batch_size)
    print("Going to download record %i to %i" % (start+1, end))
    attempt = 1
    while attempt <= 3:
        try:
            fetch_handle = Entrez.efetch(db="pubmed",rettype="medline",
                                         retmode="text",retstart=start,
                                         retmax=batch_size,
                                         webenv=search_results["WebEnv"],
                                         query_key=search_results["QueryKey"])
        except HTTPError as err:
            if 500 <= err.code <= 599:
                print("Received error from server %s" % err)
                print("Attempt %i of 3" % attempt)
                attempt += 1
                time.sleep(15)
            else:
                raise
    data = fetch_handle.read()
    fetch_handle.close()
    out_handle.write(data)
out_handle.close()
\end{verbatim}

\noindent At the time of writing, this gave 28 matches - but because this is a date dependent search, this will of course vary.  As described in Section~\ref{subsec:entrez-and-medline} above, you can then use \verb|Bio.Medline| to parse the saved records.

\subsection{Searching for citations}
\label{sec:elink-citations}

Back in Section~\ref{sec:elink} we mentioned ELink can be used to search for citations of a given paper.
Unfortunately this only covers journals indexed for PubMed Central
(doing it for all the journals in PubMed would mean a lot more work for the NIH).
Let's try this for the Biopython PDB parser paper, PubMed ID 14630660:

\begin{verbatim}
>>> from Bio import Entrez
>>> Entrez.email = "A.N.Other@example.com"
>>> pmid = "14630660"
>>> results = Entrez.read(Entrez.elink(dbfrom="pubmed", db="pmc",
...                                    LinkName="pubmed_pmc_refs", id=pmid))
>>> pmc_ids = [link["Id"] for link in results[0]["LinkSetDb"][0]["Link"]]
>>> pmc_ids
['2744707', '2705363', '2682512', ..., '1190160']
\end{verbatim}

Great - eleven articles. But why hasn't the Biopython application note been
found (PubMed ID 19304878)? Well, as you might have guessed from the variable
names, there are not actually PubMed IDs, but PubMed Central IDs. Our
application note is the third citing paper in that list, PMCID 2682512.

So, what if (like me) you'd rather get back a list of PubMed IDs? Well we
can call ELink again to translate them. This becomes a two step process,
so by now you should expect to use the history feature to accomplish it
(Section~\ref{sec:entrez-webenv}).

But first, taking the more straightforward approach of making a second
(separate) call to ELink:

\begin{verbatim}
>>> results2 = Entrez.read(Entrez.elink(dbfrom="pmc", db="pubmed", LinkName="pmc_pubmed",
...                                     id=",".join(pmc_ids)))
>>> pubmed_ids = [link["Id"] for link in results2[0]["LinkSetDb"][0]["Link"]]
>>> pubmed_ids
['19698094', '19450287', '19304878', ..., '15985178']
\end{verbatim}

\noindent This time you can immediately spot the Biopython application note
as the third hit (PubMed ID 19304878).

Now, let's do that all again but with the history \ldots
\textit{TODO}.

And finally, don't forget to include your \emph{own} email address in the Entrez calls.


%\chapter{Swiss-Prot and ExPASy}
%\label{chapter:swiss_prot}
\include{Tutorial/chapter_uniprot}

%\chapter{Going 3D: The PDB module}
\include{Tutorial/chapter_pdb}

%\chapter{Bio.PopGen: Population genetics}
\include{Tutorial/chapter_popgen}

%\chapter{Phylogenetics with Bio.Phylo}
%\label{sec:Phylo}
\include{Tutorial/chapter_phylo}

%\chapter{Sequence motif analysis using Bio.motifs}
\include{Tutorial/chapter_motifs}

%\chapter{Cluster analysis}
\include{Tutorial/chapter_cluster}

%\chapter{Supervised learning methods}
\include{Tutorial/chapter_learning}

%\chapter{Graphics including GenomeDiagram}
%\label{chapter:graphics}
\include{Tutorial/chapter_graphics}

%\chapter{KEGG}
%\label{chap:kegg}
\chapter{KEGG}
\label{chap:kegg}

KEGG (\url{http://www.kegg.jp/}) is a database resource for understanding high-level functions and utilities of the biological system, such as the cell, the organism and the ecosystem, from molecular-level information, especially large-scale molecular datasets generated by genome sequencing and other high-throughput experimental technologies.

Please note that the KEGG parser implementation in Biopython is incomplete. While the KEGG website indicates many flat file formats, only parsers and writers for compound, enzyme, and map are currently implemented. However, a generic parser is implemented to handle the other formats.

\section{Parsing KEGG records}
Parsing a KEGG record is as simple as using any other file format parser in Biopython. 
(Before running the following codes, please open \url{http://rest.kegg.jp/get/ec:5.4.2.2} with your web browser and save it as \verb|ec_5.4.2.2.txt|.)

%doctest examples
\begin{verbatim}
>>> from Bio.KEGG import Enzyme
>>> records = Enzyme.parse(open("ec_5.4.2.2.txt"))
>>> record = list(records)[0]
>>> record.classname
['Isomerases;', 'Intramolecular transferases;', 'Phosphotransferases (phosphomutases)']
>>> record.entry
'5.4.2.2'
\end{verbatim}

Alternatively, if the input KEGG file has exactly one entry, you can use \verb|read|:

%doctest examples
\begin{verbatim}
>>> from Bio.KEGG import Enzyme
>>> record = Enzyme.read(open("ec_5.4.2.2.txt"))
>>> record.classname
['Isomerases;', 'Intramolecular transferases;', 'Phosphotransferases (phosphomutases)']
>>> record.entry
'5.4.2.2'
\end{verbatim}

The following section will shows how to download the above enzyme using the KEGG api as well as how to use the generic parser with data that does not have a custom parser implemented.

\section{Querying the KEGG API}

Biopython has full support for the querying of the KEGG api. Querying all KEGG endpoints are supported; all methods documented by KEGG (\url{http://www.kegg.jp/kegg/rest/keggapi.html}) are supported. The interface has some validation of queries which follow rules defined on the KEGG site. However, invalid queries which return a 400 or 404 must be handled by the user.

First, here is how to extend the above example by downloading the relevant enzyme and passing it through the Enzyme parser.

%want online doctest here
\begin{verbatim}
>>> from Bio.KEGG import REST
>>> from Bio.KEGG import Enzyme
>>> request = REST.kegg_get("ec:5.4.2.2")
>>> open("ec_5.4.2.2.txt", 'w').write(request.read())
>>> records = Enzyme.parse(open("ec_5.4.2.2.txt"))
>>> record = list(records)[0]
>>> record.classname
['Isomerases;', 'Intramolecular transferases;', 'Phosphotransferases (phosphomutases)']
>>> record.entry
'5.4.2.2'
\end{verbatim}

Now, here's a more realistic example which shows a combination of querying the KEGG API. This will demonstrate how to extract a unique set of all human pathway gene symbols which relate to DNA repair. The steps that need to be taken to do so are as follows. First, we need to get a list of all human pathways. Secondly, we need to filter those for ones which relate to "repair". Lastly, we need to get a list of all the gene symbols in all repair pathways.

%want online doctest here
\begin{verbatim}
from Bio.KEGG import REST

human_pathways = REST.kegg_list("pathway", "hsa").read()

# Filter all human pathways for repair pathways
repair_pathways = []
for line in human_pathways.rstrip().split("\n"):
    entry, description = line.split("\t")
    if "repair" in description:
        repair_pathways.append(entry)

# Get the genes for pathways and add them to a list
repair_genes = [] 
for pathway in repair_pathways:
    pathway_file = REST.kegg_get(pathway).read()  # query and read each pathway

    # iterate through each KEGG pathway file, keeping track of which section
    # of the file we're in, only read the gene in each pathway
    current_section = None
    for line in pathway_file.rstrip().split("\n"):
        section = line[:12].strip()  # section names are within 12 columns
        if not section == "":
            current_section = section
        
        if current_section == "GENE":
            gene_identifiers, gene_description = line[12:].split("; ")
            gene_id, gene_symbol = gene_identifiers.split()

            if not gene_symbol in repair_genes:
                repair_genes.append(gene_symbol)

print("There are %d repair pathways and %d repair genes. The genes are:" % \
      (len(repair_pathways), len(repair_genes)))
print(", ".join(repair_genes))
\end{verbatim}

The KEGG API wrapper is compatible with all endpoints. Usage is essentially replacing all slashes in the url with commas and using that list as arguments to the corresponding method in the KEGG module. Here are a few examples from the api documentation (\url{http://www.kegg.jp/kegg/docs/keggapi.html}).

\begin{verbatim}
/list/hsa:10458+ece:Z5100	         -> REST.kegg_list(["hsa:10458", "ece:Z5100"])
/find/compound/300-310/mol_weight	 -> REST.kegg_find("compound", "300-310", "mol_weight")
/get/hsa:10458+ece:Z5100/aaseq	    -> REST.kegg_get(["hsa:10458", "ece:Z5100"], "aaseq")
\end{verbatim}


%\chapter{Bio.phenotype: analyse phenotypic data}
%\label{chap:phenotype}
\include{Tutorial/chapter_phenotype}

%\chapter{Bio.VCF: handling VCF and phased filese}
\chapter{Bio.VCF: handling VCF and phased files}

Since this functionality is based on \verb|pyVCF|
package, this chapter is partly based on the original \verb|pyVCF| documentation. Not all of the original functionality
is described below, therefore for further details of \verb|Bio.VCF| module, please see
\href{http://pyvcf.readthedocs.io/en/latest/INTRO.html}{pyVCF}.

\section{VCF.Reader class}

\verb|VCF.Reader| class takes a file-like object and acts as reader. It is intended to parse the content of each record
based on the data types specified in the meta-information lines, specifically the \verb|##INFO| and \verb|##FORMAT| lines.
If these lines are missing or incomplete it will check against the reserved types mentioned in the spec.
Failing that, it will just return strings.
\\
\\

\noindent Regarding the coordinates associated with each instance:

\begin{description}
  \item[\texttt{POS}] \

    per VCF specification, is the one-based index (the first base of the contig has an index of 1)
    of the first base of the \verb|REF| sequence. Positions are sorted numerically,
    in increasing order.

  \item[\texttt{start, end}] \

    denote the coordinates of the entire \verb|REF| sequence in the zero-based, half-open coordinate
    system (see \href{http://genomewiki.ucsc.edu/index.php/Coordinate_Transforms}{Coordinate\_Transforms}), where
    the first base of the contig has an index of 0, and the interval runs up to, but does not include, the base
    at the end index. This indexing scheme is analagous to Python slice notation.

  \item[\texttt{affected\_start, affected\_end}] \

    are also in the zero-based, half-open coordinate system. These coordinates indicate the precise region
    of the reference genome actually affected by the events denoted in \verb|ALT| (i.e., the minimum
    \verb|affected_start| and maximum \verb|affected_end|).
    For SNPs and structural variants, the affected region includes all bases of \verb|REF|, including the
    first base (i.e., \verb|affected_start| = \verb|start| = \verb|POS| - 1).
    For deletions, the region includes all bases of \verb|REF| except the first base, which flanks
    upstream the actual deletion event, per VCF specification.
    For insertions, the \verb|affected_start| and \verb|affected_end| coordinates represent a
    0 bp-length region between the two flanking bases (i.e., \verb|affected_start| = \verb|affected_end|).
    This is analagous to Python slice notation (see \href{http://stackoverflow.com/a/2947881/38140}{Stackoverflow}).
    Neither the upstream nor downstream flanking bases are included in the region.

\end{description}

\noindent \verb|samples| and \verb|genotype|\\

not being the title of any column, are left lowercase. The format of the fixed
fields is from the spec. Comma-separated lists in the \verb|VCF| are converted to lists. In particular,
one-entry \verb|VCF| lists are converted to one-entry Python lists (see, e.g., \verb|Record.ALT|).
Semicolon-delimited lists of key=value pairs are converted to Python dictionaries, with flags being given a True value.

\
\



\noindent Metadata regarding the VCF file itself can be investigated through the following attributes:

\

\verb|Reader.metadata|, \verb|Reader.infos|, \verb|Reader.filters|, \verb|Reader.formats|, \verb|Reader.samples|.

\

\noindent The attributes of a \verb|Record| object are the 8 fixed fields from the VCF spec: \

\

\verb|Record.CHROM|, \verb|Record.POS|, \verb|Record.ID|, \verb|Record.REF|, \verb|Record.ALT|, \verb|Record.QUAL| \
  \verb|Record.FILTER|, \verb|Record.INFO|

\

\noindent Plus attributes to handle genotype information:

\

\verb|Record.FORMAT|, \verb|Record.samples|, \verb|Record.genotype|.

\

\noindent Suppose we have a VCF file and want to see all the records in it. We might simply create a VCF.Reader object:

\begin{verbatim}
>>> from Bio.VCF import parser
>>> vcf_reader = parser.Reader(open('example-4.0.vcf', 'r'))
>>> for record in vcf_reader:
...     print record
Record(CHROM=20, POS=14370, REF=G, ALT=[A])
Record(CHROM=20, POS=17330, REF=T, ALT=[A])
Record(CHROM=20, POS=1110696, REF=A, ALT=[G, T])
Record(CHROM=20, POS=1230237, REF=T, ALT=[None])
Record(CHROM=20, POS=1234567, REF=GTCT, ALT=[G, GTACT])
\end{verbatim}


\noindent Now, we are interested in position and alteration type of the first record from the VCF file. We can obtain the information by typing:

\begin{verbatim}
>>> record = next(vcf_reader)
>>> print record.POS
14370
>>> print record.ALT
[A]
>>> print record.INFO['AF']
[0.5]
\end{verbatim}

\noindent Moreover there are a number of convenience methods and properties for each \verb|VCF.Record| object allowing you to analyse VCF object:

\begin{verbatim}
>>> print record.num_called, record.call_rate, record.num_unknown
3 1.0 0
>>> print record.num_hom_ref, record.num_het, record.num_hom_alt
1 1 1
>>> print record.nucl_diversity, record.aaf, record.heterozygosity
0.6 [0.5] 0.5
>>> print record.get_hets()
[Call(sample=NA00002, CallData(GT=1|0, GQ=48, DP=8, HQ=[51, 51]))]
>>> print record.is_snp, record.is_indel, record.is_transition, record.is_deletion
True False True False
>>> print record.var_type, record.var_subtype
snp ts
>>> print record.is_monomorphic
False
\end{verbatim}

\verb|record.FORMAT| will be a string specifying the format of the genotype fields. In case the \verb|FORMAT|
 column does not exist, \verb|record.FORMAT| is None. Finally, \verb|record.samples| is a list of dictionaries
 containing the parsed sample column and \verb|record.genotype| is a way of looking up genotypes by sample name:

 \begin{verbatim}
>>> record = next(vcf_reader)
>>> for sample in record.samples:
...     print sample['GT']
0|0
0|1
0/0
>>> print record.genotype('NA00001')['GT']
0|0
\end{verbatim}

\noindent The genotypes are represented by \verb|Call| objects, which have three attributes: the corresponding \verb|Record|
\verb|site|, the sample name in \verb|sample| and a dictionary of call data in \verb|data|:

\begin{verbatim}
>>> call = record.genotype('NA00001')
>>> print call.site
Record(CHROM=20, POS=17330, REF=T, ALT=[A])
>>> print call.sample
NA00001
>>> print call.data
CallData(GT=0|0, GQ=49, DP=3, HQ=[58, 50])
\end{verbatim}

\noindent There are also a number of methods for \verb|Call| objects:

\begin{verbatim}
>>> print call.called, call.gt_type, call.gt_bases, call.phased
True 0 T|T True
\end{verbatim}

\noindent User can be interested in some details of the VCF file, for example file's creation date, samples included
and filters. The functionality of this module supports retrieval of such data:

\begin{verbatim}
>>> vcf_reader.metadata['fileDate']
'20090805'
>>> vcf_reader.samples
['NA00001', 'NA00002', 'NA00003']
>>> vcf_reader.filters
OrderedDict([('q10', Filter(id='q10', desc='Quality below 10')), \
('s50', Filter(id='s50', desc='Less than 50% of samples have data'))])
>>> vcf_reader.infos['AA'].desc
'Ancestral Allele'
\end{verbatim}

\noindent When you are interested in \verb|ALT| records, you can interrogate them to achieve information:

\begin{verbatim}
>>> reader = parser.Reader(open('example-4.1-bnd.vcf'))
>>> _ = next(reader); row = next(reader)
>>> print row
Record(CHROM=1, POS=2, REF=T, ALT=[T[2:3[])
>>> bnd = row.ALT[0]
>>> print bnd.withinMainAssembly, bnd.orientation, bnd.remoteOrientation,
bnd.connectingSequence
True False True T
\end{verbatim}

\subsection{Record parameters}

\noindent Below there is a brief description of other methods and parameters available for Reader's records.

\begin{description}
  \item[\texttt{aaf}] \

   A list of allele frequencies of alternate alleles. NOTE: Denominator calc’ed from \_called\_ genotypes.

  \item[\texttt{alleles}] \

   list of alleles. [0] = REF, [1:] = ALTS

  \item[\texttt{call\_rate}] \

  The fraction of genotypes that were actually called.

  \item[\texttt{end}] \

  zero-based, half-open end coordinate of \verb|REF|.

  \item[\texttt{get\_hets()}] \

    The list of het genotypes.

  \item[\texttt{get\_hom\_alts()}] \

    The list of hom alt genotypes.

  \item[\texttt{get\_hom\_refs()}] \

    The list of hom ref genotypes.

  \item[\texttt{heterozygosity}] \

    Heterozygosity of a site. Heterozygosity gives the probability that two randomly chosen chromosomes from the
    population have different alleles, giving a measure of the degree of polymorphism in a population.
    If there are i alleles with frequency p\_i, H=1-sum\_i(p\_i\^2)

\end{description}

\noindent There are also a number of boolean methods, such as:

\verb|is_deletion|, \verb|is_filtered|, \verb|is_indel|, \verb|is_monomorphic|, \verb|is_snp|, \verb|is_svp|, \verb|is_transition|.

\subsection{Additional utilities}

\noindent Below there is a brief description of additional utilities for VCF files.


\begin{description}

    \item[\texttt{utils.walk\_together(*readers, **kwargs)}] \

    Simultaneously iteratate over two or more \verb|VCF| readers. For each genomic position with a variant,
    return a list of size equal to the number of VCF readers. This list contains the \verb|VCF| record from
    readers that have this variant, and None for readers that don’t have it. The caller must make sure
    that inputs are sorted in the same way and use the same reference otherwise behaviour is undefined.
    \
    \verb|Args|:
            \verb|vcf_record_sort_key|: \

    Function that takes a VCF record and returns a
    tuple that can be used as a key for comparing and sorting \verb|VCF| records across all readers.
    This tuple defines what it means for two variants to be equal (eg. whether it’s only their
    position or also their allele values), and implicitly determines the chromosome ordering since
    the tuple’s 1st element is typically the chromosome name (or calculated from it).


    \item[\texttt{vcf.utils.trim\_common\_suffix(*sequences)}] \

    Trim a list of sequences by removing the longest common suffix while leaving all of them at least one
    character in length. Standard convention with VCF is to place an indel at the left-most position,
    but some tools add additional context to the right of the sequences (e.g. samtools).
    These common suffixes are undesirable when comparing variants, for example in variant databases.

    \begin{verbatim}
    >>> trim_common_suffix('TATATATA', 'TATATA')
    ['TAT', 'T']
    \end{verbatim}

    \begin{verbatim}
    >>> trim_common_suffix('ACCCCC', 'ACCCCCCCC', 'ACCCCCCC', 'ACCCCCCCCC')
    ['A', 'ACCC', 'ACC', 'ACCCC']
    \end{verbatim}

   
    \end{description}

\section{VCF.Writer class}

\noindent The \verb|VCF.Writer| class provides a way of writing a VCF file.
\verb|VCF.parser| module supports \verb|close()|, \verb|flush()| and \verb|write_record()| methods.
\
\noindent Currently, when writing new VCF file you must specify a template \verb|VCF.Reader| object
which provides the metadata, like this:

\begin{verbatim}
>>> from Bio.VCF import parser
>>> vcf_reader = parser.Reader(filename='tb.vcf.gz')
>>> vcf_writer = parser.Writer(open('/dev/null', 'w'), vcf_reader)
>>> for record in vcf_reader:
...     vcf_writer.write_record(record)
\end{verbatim}




\section{Retrieving records from VCF file}

\noindent This chapter gives an overview of the functionality of fetch methods included in \verb|VCF| \verb|parser| module.
Those functions enable users to retrieve particular records from VCF file describing structural variants,
depending on provided argument - for example interval of positions within which we seek for structural variants, a list
of such positions or feature type within which structural variants are present.
\

\noindent Functionality of those fetch methods includes retrieving structural variants corresponding to ranges
provided in BED files as well as features from GFF files. If you are not familiar with BED file format
you might be interested in \href{http://www.ensembl.org/info/website/upload/bed.html}{BED} and
\href{http://gmod.org/wiki/GFF3}{GFF}.

\
\noindent Most of implementation is based on \verb|pybedtools| module - \href{https://daler.github.io/pybedtools/}{pybedtools},
which support many file formats including VCF, BED and GFF, therefore \verb|pybedtools| package is required.
\

\noindent Due to enormous sizes of some BED and GFF files, some fetch methods are intended to enable user to provide
stream of particular GFF/BED file from chosen database. It is a beneficial solution allowing users to avoid
the necessity of saving very big files to local disc.
\

\noindent Below there is a brief description of each fetch method.
All of them have arguments \verb|verbose| and \verb|vcf| specifying if result should be printed to stdout
and whether new VCF.Reader object should be created, respectively. When running fetch on two files, one of them must be
saved to local disc.

\begin{description}
  \item[\texttt{fetch\_bed(bed\_file, verbose = True, vcf = None)}] \

    Fetches VCF file records that correspond to regions included in a BED file.
    Fetch is based on pybedtools 'intersect' method and returns a BedTool / VCF.Reader object of selected features.
    BED file must be specified.

  \item[\texttt{fetch\_bed\_fsock(stream, verbose = False, vcf=None)}] \

    This fetch works exactly the same as fetch\_bed(), except the BED file is not required.
    Intervals used for intersection with a VCF file are provided in stream object of chosen BED file.
    This method returns a BedTool object of selected VCF features or new VCF.Reader object.
    Stream of gzipped BED file is required.

  \item[\texttt{fetch\_multilocal(chrom, local\_list, verbose = False, vcf = None)}] \

    Fetches VCF records that correspond to intervals provided in 'local\_list'.
    Local\_list must be a list of tuples (start, end), where start and end coordinates are in the
    zero-based, half-open coordinate system. Function returns selected records as a BedTool object
    or new VCF.Reader object. Chromosome must be specified.

  \item[\texttt{fetch(chrom, interval = None, verbose = True, vcf=None)}] \

    Fetches those records from VCF file that correspond to selected chromosome and
    fit in selected interval (if provided).
    This method creates one-line pybedtool feature based on selected chromosome (and interval = [start,stop]),
    and then uses it in pybedtools intersection method.
    Function returns BedTool object representing selected VCF records or new VCF.Reader object.
    Chromosome must be specified and interval is optional.

  \item[\texttt{fetch\_gff(gff\_file, chrom, feature\_type, location = None, verbose = False, vcf = None)}] \
    This method enables to select desired features from a GFF/GFF2/GFF3 file and fetch VCF records
    that correspond to position of those features. Fetch is based on pybedtools 'intersection' method and returns
    a BedTool object of chosen VCF records or new VCF.Reader object.
    Gff file, chromosome and feature type are required.
    Selection of desired features from a GFF/GFF2/GFF3 file with a specified location is possible when
    provided optional parameter 'location=[start,end]'.

  \item[\texttt{fetch\_gff\_fsock(self, stream, chrom, feature\_type, location = None, verbose = False, vcf = None)}] \

    This method works exactly the same as fetch\_gff(), except the GFF file is not required.
    The GFF file is replaced with a stream object from chosen database.
    Method returns a BedTool object of selected VCF records or new VCF.Reader object.
    Stream of gzipped file must be provided.

\end{description}


\noindent The following sections present exemplary use of fetch methods.


\subsection{Create VCF.Reader object}
\label{sec:object}


\noindent Since all fetch methods are included in VCF.Reader class, we need to create \verb|Reader| object. But first of all,
we need to import Bio.VCF library:


\begin{verbatim}
>>> from Bio import VCF
>>> from VCF import parser

\end{verbatim}

\noindent and then create a VCF.Reader obcject. We can either create object from a local file or we can provide fsock
of corresponding VCF file:


\begin{verbatim}
>>> vcf=parser.Reader(open('Tests/VCF/chr13.vcf'))

\end{verbatim}

\subsubsection{Fetching records depending on BED/GFF file}


\noindent Suppose we have a BED file with positions of CTCF motif binding sites. We want to collect all structural variants
located within those sites. If we have this BED file locally on our computer, then we should simply run:


\begin{verbatim}
>>> vcf.fetch_bed('VCF/chr13bed.bed')

\end{verbatim}


\noindent or if we don't, we should provide a stream of chosen BED file and use:


\begin{verbatim}
>>> stream="ftp://ftp.ncbi.nih.gov/snp/organisms/human_9606/BED/bed_chr_13.bed.gz"
>>> vcf.fetch_bed_fsock(stream)

\end{verbatim}


\noindent Selected structural variants are returned in BedTools object (if vcf = None) and can be easily accessed, for example
if we want to see all information of every selected record:


\begin{verbatim}
>>> bed = vcf.fetch_bed('chr13bed.bed')
>>> for b in bed:
>>>     print (b.fields)
['chr13', '10', '.', 'G', 'GGT', '.', 'PASS', 'DP=91;SS=1;SSC=2;GPV=3.0109E-23;SPV=5.8324E-1',
'GT:GQ:DP:RD:AD:FREQ:DP4','0/1:.:36:13:22:62.86%:2,11,1,21']
['chr13', '20', '.', 'GT', 'G', '.', 'PASS', 'DP=77;SS=1;SSC=2;GPV=2.4504E-29;SPV=6.0772E-1',
'GT:GQ:DP:RD:AD:FREQ:DP4','1/1:.:28:5:22:81.48%:0,5,1,21']
['chr13', '40', '.', 'AAAC', 'A', '.', 'PASS', 'DP=42;SS=1;SSC=12;GPV=7.3092E-18;SPV=6.278E-2',
'GT:GQ:DP:RD:AD:FREQ:DP4','0/1:.:13:4:9:69.23%:4,0,9,0']
['chr13', '50', '.', 'TC', 'T', '.', 'PASS', 'DP=41;SS=1;SSC=2;GPV=9.8874E-23;SPV=5.3659E-1',
'GT:GQ:DP:RD:AD:FREQ:DP4','1/1:.:22:1:21:95.45%:1,0,15,6']
['chr13', '60', '.', 'T', 'TTAA', '.', 'PASS', 'DP=27;SS=1;SSC=2;GPV=1.4382E-14;SPV=5.5556E-1',
'GT:GQ:DP:RD:AD:FREQ:DP4','1/1:.:12:0:12:100%:0,0,0,12']
\end{verbatim}


\noindent Similarly, we can extract structural variants corresponding to positions of chosen features from GFF file. Suppose
we have a GFF3 file of Homo sapiens, but we are interested only in variants located within pseudogenes
at 13th chromosome. \verb|fetch_gff()| method is what we need:


\begin{verbatim}
>>> vcf.fetch_gff('HS_fetch_gff.gff3', '13', 'pseudogene')
\end{verbatim}


\noindent What is more, if we want variant within pseudogenes which are within specific positions, we can add
\verb|location| argument:


\begin{verbatim}
>>> vcf.fetch_gff('HS_fetch_gff.gff3', '13', 'pseudogene', location=[1, 18270822])
Finding SV corresponding to pseudogene and chosen position
chr13	20	.	GT	G	.	PASS	DP=77;SS=1;SSC=2;GPV=2.4504E-29;SPV=6.0772E-1	GT:GQ:DP:RD:AD:FREQ:DP4
1/1:.:28:5:22:81.48%:0,5,1,21
chr13	30	.	AC	A	.	PASS	DP=22;SS=1;SSC=7;GPV=1.3117E-10;SPV=1.9481E-1	GT:GQ:DP:RD:AD:FREQ:DP4
1/1:.:10:2:8:80%:0,2,0,8
chr13	40	.	AAAC	A	.	PASS	DP=42;SS=1;SSC=12;GPV=7.3092E-18;SPV=6.278E-2	GT:GQ:DP:RD:AD:FREQ:DP4
0/1:.:13:4:9:69.23%:4,0,9,0
chr13	50	.	TC	T	.	PASS	DP=41;SS=1;SSC=2;GPV=9.8874E-23;SPV=5.3659E-1	GT:GQ:DP:RD:AD:FREQ:DP4
1/1:.:22:1:21:95.45%:1,0,15,6
chr13	60	.	T	TTAA	.	PASS	DP=27;SS=1;SSC=2;GPV=1.4382E-14;SPV=5.5556E-1	GT:GQ:DP:RD:AD:FREQ:DP4
1/1:.:12:0:12:100%:0,0,0,12
chr13	9542346	.	T	TTAA	.	PASS	DP=27;SS=1;SSC=2;GPV=1.4382E-14;SPV=5.5556E-1	GT:GQ:DP:RD:AD:FREQ:DP4
1/1:.:12:0:12:100%:0,0,0,12
\end{verbatim}


\noindent As in case of fetch depending on BED file, we can use stream of GFF file instead of local file.


\subsubsection{Fetching records depending on interval}


\noindent We might have a VCF file which have records referring to different chromosomes. In such case, we want to be able to
analyze only those variants that are located on chromosome of our interest. We can simply use \verb|fetch()| method
on our \verb|vcf| object, but as we want to analyze fetched records further, we should use argument specifying
new vcf.Reader object:


\begin{verbatim}
>>> v=vcf.fetch('chr13',vcf='new_vcf')
>>> v
<parser.Reader object at 0x7eff4df8f2b0>

\end{verbatim}


\noindent Now, when we have all variants located on chromosome 13, we can choose only those within specified intervals:


\begin{verbatim}
>>> v.fetch_multilocal('chr13',[[10, 30], [80, 100], [85837129, 100000000]])
chr13	20	.	GT	G	.	PASS	DP=77;SS=1;SSC=2;GPV=2.4504E-29;SPV=6.0772E-1	GT:GQ:DP:RD:AD:FREQ:DP4
1/1:.:28:5:22:81.48%:0,5,1,21
chr13	30	.	AC	A	.	PASS	DP=22;SS=1;SSC=7;GPV=1.3117E-10;SPV=1.9481E-1	GT:GQ:DP:RD:AD:FREQ:DP4
1/1:.:10:2:8:80%:0,2,0,8
chr13	85837130	.	T	TTAA	.	PASS	DP=27;SS=1;SSC=2;GPV=1.4382E-14;SPV=5.5556E-1	GT:GQ:DP:RD:AD:FREQ:DP4
1/1:.:12:0:12:100%:0,0,0,12
<BedTool(/tmp/pybedtools.dbyw5lct.tmp)>

\end{verbatim}


\noindent VCF file created after \verb|vcf.fetch| is saved to local disc and can be easily achieved with
VCF.Reader() anytime needed.



\section{VCF Filters}

\subsection{The filters.py script}
\noindent Filtering a VCF file based on some properties of interest is a common enough operation that PyVCF offers an extensible script.

Existing Filter's Classes:

\verb|Bio.VCF.filters.SiteQuality|
\verb|Bio.VCF.filters.VariantGenotypeQuality|
\verb|Bio.VCF.filters.ErrorBiasFilter|
\verb|Bio.VCF.filters.DepthPerSample|
\verb|Bio.VCF.filters.AvgDepthPerSample|
\verb|Bio.VCF.filters.SnpOnly|



\subsection{Adding a filter}
\noindent You can reuse this work by providing a filter class, rather than writing your own filter. For example, lets say I want to filter each site based on the quality of the site. I can create a class like this::
\begin{verbatim}   
import Bio.VCF.filters
class SiteQuality(filters.Base):
    'Filter sites by quality'
    name = 'sq'
    @classmethod
    def customize_parser(self, parser):
        parser.add_argument('--site-quality', type=int, default=30,
            help='Filter sites below this quality')
    def __init__(self, args):
        self.threshold = args.site_quality
    def __call__(self, record):
        if record.QUAL < self.threshold:
            return record.QUAL
\end{verbatim}

\noindent This class subclasses \verb|Bio.VCF.filters.Base| which provides the interface for VCF filters. The docstring  and \verb|name| are metadata about the parser.  The docstring provides the help for the script, and the first line is included in the FILTER metadata when applied to a file.

\noindent The \verb|customize_parser| method allows you to add arguments to the script. We use the \verb|__init__| method to grab the argument of interest from the parser. Finally, the \verb|__call__| method processes each record and returns a value if the filter failed.  The base class uses the \verb|name| and \verb|threshold| to create the filter ID in the VCF file.

\noindent Unfortunately \verb|Bio.VCF.filters.scripts.vcf_filters.py| don't work as creator of pyVCF wanted. On official VCF-github site you can finde more information about that. We're hoping that that bugs will be soon fixed. But before it happens we recomand to use filters carefully some of them doestn't work.

\noindent Example of using filters form filters.py file:
\begin{verbatim}
import Bio.VCF
>>>#creating Reader object on our vcf file
>>>reader_object=Bio.VCF.Reader(open('Tests/VCF/example-4.1.vcf','rt'))
>>>
>>>#creating object Base (form filters.py file) with Reder object
>>>base_object=Bio.VCF.filters.Base(reader_object)
>>>#choose filter for example VCF.filters.SnpOnly
>>>
>>>so= Bio.VCF.filters.SnpOnly(base_object)
>>>
>>>#than we call function on reader_bejct lines to find this which agree which choosen value of site quality
>>>for i in readerobject:
...		print so(i)
...
\end{verbatim}


\subsection{sample_filters.py script}
\noindent \verb|Bio.VCF.sample_filters| are functions created to filter VCF files by haplotyp's sample. Usually we're interseted just in specified mutants, thats why it's very helpfull to get information to further analysis only about this variants.

\noindent How to use it:

\begin{verbatim}
positional arguments:
  file          VCF file to filter

optional arguments:
  -h, --help    show this help message and exit
  -o outfile    File to write out filtered samples
  -f filters    Comma-separated list of sample indices or names to filter
  -i, --invert  Keep rather than discard the filtered samples
  -q, --quiet   Less output
\end{verbatim}

\noindent Example of usage:
\begin{verbatim}
$ python ../Bio/VCF/scripts/vcf_sample_filter.py VCF/example-4.1.vcf
Samples:
0: NA00001
1: NA00002
2: NA00003

$python ../Bio/VCF/scripts/vcf_sample_filter.py VCF/example-4.1.vcf -f 1,2 --quiet
##fileformat=VCFv4.1
##fileDate=20090805
##source=myImputationProgramV3.1
##reference=file:///seq/references/1000GenomesPilot-NCBI36.fasta
##phasing=partial
##INFO=<ID=NS,Number=1,Type=Integer,Description="Number of Samples With Data">
##INFO=<ID=DP,Number=1,Type=Integer,Description="Total Depth">
##INFO=<ID=AF,Number=A,Type=Float,Description="Allele Frequency">
##INFO=<ID=AA,Number=1,Type=String,Description="Ancestral Allele">
##INFO=<ID=DB,Number=0,Type=Flag,Description="dbSNP membership, build 129">
##INFO=<ID=H2,Number=0,Type=Flag,Description="HapMap2 membership">
##FORMAT=<ID=GT,Number=1,Type=String,Description="Genotype">
##FORMAT=<ID=GQ,Number=1,Type=Integer,Description="Genotype Quality">
##FORMAT=<ID=DP,Number=1,Type=Integer,Description="Read Depth">
##FORMAT=<ID=HQ,Number=2,Type=Integer,Description="Haplotype Quality">
##FILTER=<ID=q10,Description="Quality below 10">
##FILTER=<ID=s50,Description="Less than 50% of samples have data">
##contig=<ID=20,length=62435964>
#CHROM	POS	ID	REF	ALT	QUAL	FILTER	INFO	FORMAT	NA00001
20	14370	rs6054257	G	A	29	PASS	NS=3;DP=14;AF=0.5;DB;H2	GT:GQ:DP:HQ	0|0:48:1:51,51
20	17330	.	T	A	3	q10	NS=3;DP=11;AF=0.017	GT:GQ:DP:HQ	0|0:49:3:58,50
20	1110696	rs6040355	A	G,T	67	PASS	NS=2;DP=10;AF=0.333,0.667;AA=T;DB	GT:GQ:DP:HQ	1|2:21:6:23,27
20	1230237	.	T	.	47	PASS	NS=3;DP=13;AA=T	GT:GQ:DP:HQ	0|0:54:7:56,60
20	1234567	microsat1	GTC	G,GTCT	50	PASS	NS=3;DP=9;AA=G	GT:GQ:DP	0/1:35:4
\end{verbatim}   


\section{Available VCF Databases}

\noindent Bio.VCF.databases is a Biopython module that supports \verb|1001 Genomes| and \verb|dbSNP| databases.
Since most of VCF files are of enormous sizes, it is convenient to use streams instead of local files.
Functionality of this package enables users to search through above-mentioned databases, and use selected streams
of VCF files as VCF.Reader objects in further analysis.


\noindent If you are not familiar with \verb|1001 Genomes| and \verb|dbSNP| databases you might be interested in
\href{http://1001genomes.org/}{1001 Genomes} and \href{https://www.ncbi.nlm.nih.gov/SNP/}{dbSNP}. However, following
sections provide brief description of methods searching through both databases as well as some information about them.


\subsection{1001 Genomes database}
\label{sec:object}


\verb|1001 Genomes| database was launched to discover whole-genome variation of the \verb|Arabidopsis Thaliana| strains.
\verb|Bio.VCf.databases| module supports current database's version which includes analysis of 1135 genomes achieved
during the first phase of the project.
Searching through the database can be based on strain's name, ec number, eco type, as well as on information of its
origin - such as country, longitude or latitude.


\noindent Below there is a brief description of \verb|thousandgenomes()| methods:


\begin{description}
  \item[\texttt{thousandgenomes(name, ecotype, ecnumber, country, longitude, latitude)}] \

    This method enables to search through \verb|1001 Genomes| database for VCF file corresponding to selected
    Arabidopsis Thaliana strain, origin country or longitude and/or latitude where Arabidopsis Thaliana live.

    Country name must be chosen from: "USA", "FRA", "CZE", "AUT", "KGZ", "TJK", "SWE", "UK", "GER", "KAZ",
    "BEL", "CPV", "ESP", "RUS", "NED", "FIN", "SUI", "ITA", "IRL", "POR", "EST", "DEN", "IND", "LTU", "JPN", "POL", "NOR",
    "CAN", "UKR", "AZE", "GEO", "ARM", "MAR", "CRO", "BUL", "GRC", "SVK", "ROU", "UZB", "SRB", "CHN", "IRN", "LBN", "MAR",
    "AFG".
    Longitude and latitude should be provided as an interval - longitude = (int1, int2), latitude = (int1,int2).
    Strain names, eco types and ec numbers are unique, country, while longitude and latitude enables users to seek for
    multiple VCF files corresponding to selected features.
    Method returns VCF.Reader object of selected VCF file stream from 1001 Genomes Database or a list of
    such VCF.Reader objects.

\item[\texttt{download(vcf\_reader, path\_filename)}] \

    Downloads VCF file corresponding to stream on which provided VCF.Reader object is initialized.
    VCF.Reader object as well as download directory and filename are required.

\item[\texttt{merge(vcf\_list,vcf\_template,vcf\_name)}] \

    Method creates VCF.Reader object from provided list of VCF.Reader objects.
    It is based on VCF.Writer class and uses Writer.write\_record() method - new sorted VCF file is saved to
    bio-VCF/Bio/VCF directory.
    Template VCF.Reader object and name for new VCF file are required.

\end{description}

\noindent Let's try some functionality of the module:

\noindent Suppose we want to analyse structural variants identified in "CYR" strain of Arabidopsis Thaliana. First of all,
we need to find VCF file corresponding to our selected strain. To do so, we can simply type:


\begin{verbatim}
>>> from Bio.VCF import databases
>>> vcf = databases.thousandgenomes("CYR")
>>> vcf
<Bio.VCF.parser.Reader object at 0x7f4bf49feef0>

\end{verbatim}


\noindent Now, we can continue analysis for example by running methods from VCF.Reader class.


\noindent We may also be interested in retrieving all structural variants that are identified in strains occupying a particular
territory, such as area of longitude (-73.1494, -73.1492) and latitude (40.9063, 40.9064). Module's \verb|thousandgenomes()|
method can manage this:


\begin{verbatim}
>>> vcf_list = databases.thousandgenomes(longitude=(-73.1494, -73.1492), latitude=(40.9063, 40.9064))
>>> vcf_list
[<Bio.VCF.parser.Reader object at 0x7f4bf4a0d978>, <Bio.VCF.parser.Reader object at 0x7f4bf4a1e080>]

\end{verbatim}


\noindent Method returned two VCF.Reader objects which we can analyse further or simply download to local disc:


\begin{verbatim}
>>> for v in vcf_list:
>>>     databases.download(v,'database_download.gz')
\end{verbatim}

\subsection{dbSNP database}

\verb|dbSNP| Is the NCBI Short Genetic Variations database. It catalogs short variations in nucleotide sequences froma wide range of organisms. 
\verb|Bio.VCf.databases| module can be used to obtain avaiable VCF files from dbSNP. List of organism for which VCF files exists can be found in supplementary file \verb|organisms.txt|. The database is regulary updated. User can update list od organisms using attached methods. Searching through the database can be based on organism and taxon ID proposed in file \verb|organisms.txt| and a chosen chromosome. Unfortunatelly dbSNP does not contain human SNPs divided into chromosomes. User can operate only on a very big file (350GB).

\noindent Below there is a brief description of \verb|dbSNP| methods:

\begin{description}
  \item[\texttt{dbSNP\_download(organism\_taxon, chromosome = None)}] \

    This method enables to search through \verb|dbSNP| database for VCF file corresponding to selected
    organism, taxon and chromosome.
    
    Organism and corresponding taxon ID must be chosen from list of organisms in \verb|organisms.txt|. User can update this list by running
    \verb|check_VCF()| method.
    
    Chromosome must be chosen from existing chromosomes in given organism. Leaving \verb|chromosome=None| will cause loading VCF file
    containing all the chromosomes (but not every organism has this file).
    
    Method returns VCF.Reader object of selected VCF file stream from dbSNP.
    
    \item[\texttt{check\_VCF()}] \
    
    This method enables to update organisms list in \verb|organisms.txt|. dbSNP is regullary updated. If you can't find chosen organisms in 
    the list we recommend running this method. 

\end{description}

\noindent Let's try some functionality of the module:

\noindent Suppose we want to analyse SNPs found in chromosome 2R in fruitfly. We check \verb|organisms.txt| where we can find identifier 
fruitfly\_7227.

\begin{verbatim}
>>> from Bio.VCF import databases
>>> vcf = databases.dbSNP_download('fruitfly_7227',chromosome='2R')
>>> vcf
<Bio.VCF.parser.Reader object at 0x7f1d813f14d0>
\end{verbatim}

\noindent Now we can continue our analysis using methods proposed in VCF.Reader class.

\noindent We can also want to analyse all SNPs found for flycatcher:

\begin{verbatim}
>>> vcf = databases.dbSNP_download('flycatcher_59894')
>>> vcf
<Bio.VCF.parser.Reader object at 0x7f1d82c83ad0>
\end{verbatim}

\noindent If you want to check if new VCF files are avaiable just run:

\begin{verbatim}
>>> from Bio.VCF import databases
>>> databases.check_VCF()
Connecting to database...
Searching for VCF files...
Searching for VCF files...
Searching for VCF files...
Searching for VCF files...
Searching for VCF files...
.
.
.
List updated.
\end{verbatim}
    
    
\section {Retrieving records from phasing files}
%todo Hans

This chapter gives an overview of the usage of phasing files parser included in \verb|Bio.VCF| \verb|phase| module.
This module makes handling haplotype data easier, allowing users to retrieve particular records from phasing files
and even filtering SNPs by matching them to those included in VCF file.

The whole module is specifficaly written for handling data from \href{ftp://ftp.hapmap.org/hapmap/phasing/2009-02\_phaseIII/HapMap3\_r2/}{HapMap 3} project.
Format of the data handled by functions is as staded in \href{ftp://ftp.hapmap.org/hapmap/phasing/2009-02\_phaseIII/HapMap3\_r2/hapmap3\_r2\_phasing\_summary.doc}{hapmap3\_r2\_phasing\_summary.doc},
but also assuming that SNPs in those files are sorted increasingly by position.


\subsection{Parsing and reading phasing files}

Phasing data can be read as \verb|VCF.PhasedReader| object. It is similiar to \verb|VCF.Reader| class - works with both compressed (.gz) and
uncompressed files, can take fsocks (streams) and filenames as input, with specified encoding (default is ascii).
\verb|VCF.PhasedReader| holds information extracted from the name of the file (if available), haplotypes, and SNP's in form of \verb|VCF._PhasedRecord| objects.
It also contains several useful functions explained further in the coresponding subsections in \verb|VCF.PhasedReader functions| section below.
Important thing is, if you are downloading the .phased file to your computer, rather not change the file name as it gives important
information not stated anywhere in the file. Files without original name will be parsed normally but will lack in information like:
\begin{enumerate}
  \item chromosome
  \item region
  \item if the file is for unrelated haplotypes, duos or trios.
\end{enumerate}

\subsubsection{Creating VCF.PhasedReader objects}

\noindent Suppose we have a phasing file downloaded from the \href{ftp://ftp.hapmap.org/hapmap/phasing/2009-02\_phaseIII/HapMap3\_r2/}{HapMap 3} site.
We can read it in several ways, in example supporting only the filename:

\begin{verbatim}
>>> from Bio.VCF import PhasedReader
>>> phase_reader = PhasedReader('hapmap3_r2_b36_fwd.consensus.qc.poly.chr10_yri.D.phased')
>>> phase_reader
<Bio.VCF.phase.PhasedReader object at 0x7fee0c0445c0>

\end{verbatim}

\noindent or with opening the file first:

\begin{verbatim}
>>> from Bio.VCF import PhasedReader
>>> phase_file = open('hapmap3_r2_b36_fwd.consensus.qc.poly.chr10_yri.D.phased')
>>> phase_reader = PhasedReader(fsock = phase_file)
>>> phase_reader
<Bio.VCF.phase.PhasedReader object at 0x7fee0c04bd30>

\end{verbatim}

\noindent If the file is compressed (.gz), the parser will try to guess it and uncompress it without stating anything, but it is advised that the uncompressed argument is set
to true:

\begin{verbatim}
>>> from Bio.VCF import PhasedReader
>>> phase_reader = PhasedReader('hapmap3_r2_b36_fwd.consensus.qc.poly.chr10_yri.D.phased.gz',
compressed=True)
>>> phase_reader
<Bio.VCF.phase.PhasedReader object at 0x7fedf6df1278>

\end{verbatim}

\subsubsection{Exploring data held by VCF.PhasedReader}

\noindent Data extracted from the filename is stored in the VCF.PhasedReader.filedata dict.

\begin{verbatim}
>>> phase_reader.filedata
{'region': 'yri', 'chrom': '10', 'data_type': 'duos'}

\end{verbatim}

\noindent It is possible to get to the particular information by calling its key from dict:

\begin{verbatim}
>>> phase_reader.filedata['region']
'yri'
>>> phase_reader.filedata['chrom']
'10'
>>> phase_reader.filedata['data_type']
'duos'

\end{verbatim}

\noindent Haplotypes included in the file are stored in the haplotypes list, as \_Haplotype objects:

\begin{verbatim}
>>> phase_reader.haplotypes
[<Bio.VCF.phase._Haplotype object at 0x7fb16ed6e828>,
<Bio.VCF.phase._Haplotype object at 0x7fb16ed6e898>,
<Bio.VCF.phase._Haplotype object at 0x7fb16ed6e908>,
<Bio.VCF.phase._Haplotype object at 0x7fb16ed6e940>,
<Bio.VCF.phase._Haplotype object at 0x7fb16ed6e9b0>,
.
.
.

\end{verbatim}

\noindent Let's see the first haplotype in the file:

\begin{verbatim}
>>> phase_reader.haplotypes[0]
<Bio.VCF.phase._Haplotype object at 0x7fb16ed6e828>
>>> print(phase_reader.haplotypes[0])
NA18855, transmitted: True

\end{verbatim}

\noindent The \_Haplotype object holds information about the name of haplotype and if it is transmitted:

\begin{verbatim}
>>> phase_reader.haplotypes[0].name
'NA18855'
>>> phase_reader.haplotypes[0].is_transmitted
True

\end{verbatim}

\subsubsection{Parsing phasing files with VCF.PhasedReader}

\noindent To get the first (and then next, and then next...) line of the file, you just call next() on the PhasedReader:

\begin{verbatim}
>>> rec = next(phase_reader)
>>> rec
<Bio.VCF.phase._PhasedRecord object at 0x7fb16ed7f6d8>
>>> print(rec)
Record(rs12255619 at 88481: C, A, A, A, A, A, A, A, A, A, A, A)

\end{verbatim}

\noindent Parsing the phasing file by iterating over it is also possible:

\begin{verbatim}
>>> for rec in phase_reader:
...     print(rec)
...
Record(rs12255619 at 88481: C, A, A, A, A, A, A, A, A, A, A, A)
Record(rs11252546 at 94427: T, T, T, T, T, T, T, T, C, T, T, T)
Record(rs17156316 at 191762: T, T, T, T, T, T, T, C, C, T, T, T)
Record(rs10903451 at 193471: A, A, A, A, G, G, A, A, A, A, A, G)
Record(rs11250575 at 199448: T, C, C, C, C, C, C, C, C, C, C, C)
.
.
.

\end{verbatim}

\noindent Let's explore what is stored in the \_PhasedRecord:
\begin{enumerate}
\item rsID of the SNP
\begin{verbatim}
>>> rec.rsID
'rs12255619'
\end{verbatim}

\item record position on the chromosome:
\begin{verbatim}
>>> rec.pos
88481
\end{verbatim}

\item samples included in the record (as list of \_Sample objects):
\begin{verbatim}
>>> rec.samples
[<Bio.VCF.phase._Sample object at 0x7f38a3d20a90>,
<Bio.VCF.phase._Sample object at 0x7f38a3d3d048>,
<Bio.VCF.phase._Sample object at 0x7f38a3d3d080>,
<Bio.VCF.phase._Sample object at 0x7f38a3d3d0b8>,
<Bio.VCF.phase._Sample object at 0x7f38a3d3d160>,
.
.
.
\end{verbatim}
\end{enumerate}

\noindent The \_Sample object corresponds to the nucleotide in the particular haplotype for the particular SNP:

\begin{verbatim}
>>> rec.samples[0]
<Bio.VCF.phase._Sample object at 0x7f38a3d20a90>
>>> rec.samples[0].rsID, rec.samples[0].haplotype.name, rec.samples[0].nucleotide
('rs12255619', 'NA18855', 'C')
\end{verbatim}

\noindent \_Sample also holds information if the particular SNP for the haplotype exists and if it's unresolved:

\begin{verbatim}
>>> rec.samples[0].exists, rec.samples[0].is_unresolved
(True, False)
\end{verbatim}

\noindent There is also the is\_not\_matching\_snp argument, but it is further discussed in the \verb|VCF.PhasedReader.fetch()|
section.

\subsection{Writing phasing files with VCF.PhasedWriter}

\noindent The VCF.PhasedWriter works similarily to the VCF.Writer. You need to supply the stream to save the contents and
the template (PhasedReader object):

\begin{verbatim}
>>> phase_reader = PhasedReader('hapmap3_r2_b36_fwd.consensus.qc.poly.chr10_yri.D.phased')
>>> out = open('somenewname.phased','w')
>>> phase_writer = PhasedWriter(stream = out, template = phase_reader)
>>> phase_writer
<Bio.VCF.phase.PhasedWriter object at 0x7f991e9ba4e0>
\end{verbatim}

\noindent and then, you write in the records you want, i.e fetched from the VCF file:

\begin{verbatim}
>>> for rec in phase_reader.fetch(filename = './Tests/VCF/chr10.vcf'):
...     phase_writer.write(rec)
...
rs12255619	88481	C A A A A A A A A A A A
rs1904671	1004603	A G. A A A A G. G. A A A A
rs11812734	31713857	A G A A G G G G A G G G

\end{verbatim}

\noindent The output in this case should be stored in the somenewname.phased file.

\subsection{VCF.PhasedReader functions}

\subsubsection{VCF.PhasedReader.fetch() - Fetching SNP's from VCF file}

\noindent The fetch metod retrieves SNPs corresponding to those in provided VCF file. It creates \verb|VCF.Reader|
and filters it to have only the SNP's, then finds the matching ones in the phasing file. It returns the next PhasedReader object, but only with matching records.
It takes the same arguments as VCF.Reader when it is created (see in the section about \verb|VCF.Reader| above) and a few more:

\begin{enumerate}
  \item verbose (default True) - if the found records should be printed out
  \item vcf (default None) - the argument for VCF.Reader.fetch, see above
  \item not\_matching\_snp (default ".") - what should be put after the nucleotide not matching any of the alleles in the VCF.
  If None, nucleotide is outputted normally.
\end{enumerate}

\noindent Remember that all the functions in the PhasedReader objects, especially fetch, treats the phasing file as sorted by position.
\\

\noindent Suppose you wanted to filter the SNPs in the phasing files with the VCF file with some records for the same chromosome.
First you need to create the PhasedReader object, and then fetch it by supplying the same arguments as you would when creating the VCF.Reader object.

\begin{verbatim}
>>> phase_reader = PhasedReader('Tests/VCF/hapmap3_r2_b36_fwd.consensus.qc.poly.chr10_yri.D.phased.gz',
compressed = True)
>>> phase_reader.fetch(filename = 'Tests/VCF/chr10.vcf',compressed = True)
rs12255619	88481	C A A A A A A A A A A A
rs1904671	1004603	A G. A A A A G. G. A A A A
rs11812734	31713857	A G A A G G G G A G G G
<Bio.VCF.phase.PhasedReader object at 0x7f38a3d3d128>
\end{verbatim}

\noindent If you don't want dots, or any sign after the not-matching SNPs, set not\_matching\_snp to None:

\begin{verbatim}
>>> phase_reader.fetch(filename = 'Tests/VCF/chr10.vcf',not_matching_snp = None)
rs12255619	88481	C A A A A A A A A A A A
rs1904671	1004603	A G A A A A G G A A A A
rs11812734	31713857	A G A A G G G G A G G G
<Bio.VCF.phase.PhasedReader object at 0x7f38a3d20b70>
\end{verbatim}

\noindent Or if you don't want the lines to be printed and just have the new reader, simply change the verbose to False:

\begin{verbatim}
>>> phase_reader.fetch(filename = 'Tests/VCF/chr10.vcf',verbose=False)
<Bio.VCF.phase.PhasedReader object at 0x7f38a3d3d518>
\end{verbatim}

\subsubsection{VCF.PhasedReader.get_specific_snp(rsID) - Getting SNP with user-given rsID}

\noindent rsID argument provides information on rsID of searched SNP.

\noindent As the output user receives a _PhasedRecord object, which represents found SNP, or information that no SNP with given rsID was found.

\subsubsection{VCF.PhasedReader.get_snp_with_specific_id(rsID) - Getting SNP with user-given rsID}

\noindent rsID argument provides information on rsID of searched SNP.

\noindent As the output user receives a _PhasedRecord object, which represents found SNP, or information that no SNP with given rsID was found.

\begin{verbatim}
>>> from Bio.VCF import PhasedReader
>>> reader = PhasedReader('Tests/VCF/hapmap3_r2_b36_fwd.consensus.qc.poly.chr10_yri.D.phased')
>>> reader.get_snp_with_specific_id('rs2066314')
<Bio.VCF.phase._PhasedRecord object at 0x7fe67d7a7e50>

\end{verbatim}

\subsubsection{VCF.PhasedReader.get_snp_within_range(pos1, pos2) - Getting SNPs within user-given range of positions}

\noindent pos1 and pos2 arguments stand for a beginning and an ending position of the range (respectively) within which we are looking for SNPs.

\noindent As the output user receives information on how many SNPs within given range were found and a list of _PhasedRecord objects, where each of them represents
one found SNP, or information that no SNP within given range was found.

\begin{verbatim}
>>> from Bio.VCF import PhasedReader
>>> reader = PhasedReader('Tests/VCF/hapmap3_r2_b36_fwd.consensus.qc.poly.chr10_yri.D.phased')
>>> reader.get_snp_within_range(418076, 504032)
SNPs found: 3

[<Bio.VCF.phase._PhasedRecord object at 0x7f3e13b30dd0>,
<Bio.VCF.phase._PhasedRecord object at 0x7f3e13b30750>,
<Bio.VCF.phase._PhasedRecord object at 0x7f3e13b33150>]

\end{verbatim}

\subsection{VCF.PhasedReader.get_snp_with_specific_sample(haplotype, nucleotide) - Getting SNPs which contain user-given sample}

\noindent haplotype argument provides information on the sample's haplotype name.
nucleotide argument provides the information on the sample's nucleotide ('A', 'T', 'G', 'C'). 

\noindent As the output user receives information whether the haplotype provided in the arguments is included in the given file (if not, corresponding information 
is being printed), how many SNPs containing searched sample were found and a list of _PhasedRecord objects, where each of them represents one found SNP. If no SNP 
meeting the requirements provided in the arguments was found, corresponding information is being printed. 

\begin{verbatim}
>>> from Bio.VCF import PhasedReader
>>> reader = PhasedReader('Tests/VCF/hapmap3_r2_b36_fwd.consensus.qc.poly.chr10_yri.D.phased')
>>> reader.get_snp_with_specific_sample('NA18855_B', 'T')
Searched haplotype is included in the given file...
SNPs found: 7

[<Bio.VCF.phase._PhasedRecord object at 0x7fe75042c7d0>, <Bio.VCF.phase._PhasedRecord object at 0x7fe75042c110>, <Bio.VCF.phase._PhasedRecord object at 0x7fe75042cb50>, <Bio.VCF.phase._PhasedRecord object at 0x7fe75042ce90>, <Bio.VCF.phase._PhasedRecord object at 0x7fe75042f550>, <Bio.VCF.phase._PhasedRecord object at 0x7fe75042f890>, <Bio.VCF.phase._PhasedRecord object at 0x7fe75042f210>]
\end{verbatim}

\subsection{VCF.PhasedReader.get_samples_from_specific_hap(haplotype) - Getting samples within user-given haplotype}

\noindent haplotype argument provides information on the name of the haplotype, within which user is searching for samples.

\subsection{VCF.PhasedReader.get_snp_with_specific_sample(haplotype) - Getting SNPs with user-given details of the searched sample}

\noindent haplotype argument provides information on the name of the haplotype, within which user is searching for samples.

\noindent As the output user receives information whether the haplotype provided in the arguments is included in the given file (if not, corresponding information 
is being printed), how many samples within searched haplotype were found and a list of _Sample objects, where each of them represents one found sample. Moreover, 
information on the number of unresolved and not existing samples (within given haplotype), which are found during parsing a file, is being printed. If no sample 
meeting the requirements provided in the arguments was found, corresponding information is also being printed.

\begin{verbatim}
>>> from Bio.VCF import PhasedReader
>>> reader = PhasedReader('Tests/VCF/hapmap3_r2_b36_fwd.consensus.qc.poly.chr10_yri.D.phased')
>>> reader.get_samples_in_specific_hap('NA18855_NA18856_A')
Searched haplotype is included in the given file...
Samples found: 32
SNPs found unresolved: 0
SNPs found not existing: 0

[<Bio.VCF.phase._Sample object at 0x7fe75042fbd0>, <Bio.VCF.phase._Sample object at 0x7fe750435490>, <Bio.VCF.phase._Sample object at 0x7fe75042fc50>, <Bio.VCF.phase._Sample object at 0x7fe7504354d0>, <Bio.VCF.phase._Sample object at 0x7fe75042fc90>, <Bio.VCF.phase._Sample object at 0x7fe750435510>, <Bio.VCF.phase._Sample object at 0x7fe75042fcd0>, <Bio.VCF.phase._Sample object at 0x7fe750435550>, <Bio.VCF.phase._Sample object at 0x7fe75042fd10>, <Bio.VCF.phase._Sample object at 0x7fe750435050>, <Bio.VCF.phase._Sample object at 0x7fe75042fd50>, <Bio.VCF.phase._Sample object at 0x7fe750435090>, <Bio.VCF.phase._Sample object at 0x7fe75042fd90>, <Bio.VCF.phase._Sample object at 0x7fe7504350d0>, <Bio.VCF.phase._Sample object at 0x7fe75042fdd0>, <Bio.VCF.phase._Sample object at 0x7fe750435110>, <Bio.VCF.phase._Sample object at 0x7fe75042fe10>, <Bio.VCF.phase._Sample object at 0x7fe750435150>, <Bio.VCF.phase._Sample object at 0x7fe75042fe50>, <Bio.VCF.phase._Sample object at 0x7fe750435190>, <Bio.VCF.phase._Sample object at 0x7fe750435590>, <Bio.VCF.phase._Sample object at 0x7fe7504351d0>, <Bio.VCF.phase._Sample object at 0x7fe7504355d0>, <Bio.VCF.phase._Sample object at 0x7fe750435610>, <Bio.VCF.phase._Sample object at 0x7fe750435650>, <Bio.VCF.phase._Sample object at 0x7fe750435690>, <Bio.VCF.phase._Sample object at 0x7fe7504356d0>, <Bio.VCF.phase._Sample object at 0x7fe750435710>, <Bio.VCF.phase._Sample object at 0x7fe750435750>, <Bio.VCF.phase._Sample object at 0x7fe750435790>, <Bio.VCF.phase._Sample object at 0x7fe7504357d0>, <Bio.VCF.phase._Sample object at 0x7fe750435810>]
\end{verbatim}

\subsection{VCF.PhasedReader.get_specific_sample(rsID, haplotype) - Getting a sample with user-given details}

\noindent rsID argument provides information on the rsID of SNP, within which user is searching for a specific sample.
haplotype argument provides information on the name of the haplotype, within which user is searching for a specific sample.

\noindent As the output user receives information whether the haplotype provided in the arguments is included in the given file (if not, corresponding information 
is being printed) and a _Sample object, which represents found sample. Moreover, if searched sample turns out to be unresolved or not existing, corresponding 
information is being printed. If no sample meeting the requirements provided in the arguments was found, corresponding information is also being printed.

\begin{verbatim}
>>> from Bio.VCF import PhasedReader
>>> reader = PhasedReader('Tests/VCF/hapmap3_r2_b36_fwd.consensus.qc.poly.chr10_yri.D.phased')
>>> reader.get_specific_sample('rs11252546', 'NA18855_NA18856_A')
Searched haplotype is included in the given file...
<Bio.VCF.phase._Sample object at 0x7f8bb5d4c7d0>

/end{verbatim}


%\chapter{Cookbook -- Cool things to do with it}
%\label{chapter:cookbook}
\include{Tutorial/chapter_cookbook}

%\chapter{The Biopython testing framework}
%\label{sec:regr_test}
\include{Tutorial/chapter_testing}

%\chapter{Advanced}
%\label{chapter:advanced}
\include{Tutorial/chapter_advanced}

%\chapter{Where to go from here -- contributing to Biopython}
\include{Tutorial/chapter_contributing}

%\chapter{Appendix: Useful stuff about Python}
%\label{sec:appendix}
\include{Tutorial/chapter_appendix}

\begin{thebibliography}{99}
\bibitem{cock2009}
Peter J. A. Cock, Tiago Antao, Jeffrey T. Chang, Brad A. Chapman, Cymon J. Cox, Andrew Dalke, Iddo Friedberg, Thomas Hamelryck, Frank Kauff, Bartek Wilczynski, Michiel J. L. de Hoon: ``Biopython: freely available Python tools for computational molecular biology and bioinformatics''. {\it Bioinformatics} {\bf 25} (11), 1422--1423 (2009). \href{http://dx.doi.org/10.1093/bioinformatics/btp163}{doi:10.1093/bioinformatics/btp163},
\bibitem{pritchard2006}
Leighton Pritchard, Jennifer A. White, Paul R.J. Birch, Ian K. Toth: ``GenomeDiagram: a python package for the visualization of large-scale genomic data''.  {\it Bioinformatics} {\bf 22} (5): 616--617 (2006).
\href{http://dx.doi.org/10.1093/bioinformatics/btk021}{doi:10.1093/bioinformatics/btk021},
\bibitem{toth2006}
Ian K. Toth, Leighton Pritchard, Paul R. J. Birch: ``Comparative genomics reveals what makes an enterobacterial plant pathogen''. {\it Annual Review of Phytopathology} {\bf 44}: 305--336 (2006).
\href{http://dx.doi.org/10.1146/annurev.phyto.44.070505.143444}{doi:10.1146/annurev.phyto.44.070505.143444},
\bibitem{vanderauwera2009}
G\'eraldine A. van der Auwera, Jaroslaw E. Kr\'ol, Haruo Suzuki, Brian Foster, Rob van Houdt, Celeste J. Brown, Max Mergeay, Eva M. Top: ``Plasmids captured in C. metallidurans CH34: defining the PromA family of broad-host-range plasmids''.
\textit{Antonie van Leeuwenhoek} {\bf 96} (2): 193--204 (2009).
\href{http://dx.doi.org/10.1007/s10482-009-9316-9}{doi:10.1007/s10482-009-9316-9}
\bibitem{proux2002}
Caroline Proux, Douwe van Sinderen, Juan Suarez, Pilar Garcia, Victor Ladero, Gerald F. Fitzgerald, Frank Desiere, Harald Br\"ussow:
``The dilemma of phage taxonomy illustrated by comparative genomics of Sfi21-Like Siphoviridae in lactic acid bacteria''.  \textit{Journal of Bacteriology} {\bf 184} (21): 6026--6036 (2002).
\href{http://dx.doi.org/10.1128/JB.184.21.6026-6036.2002}{http://dx.doi.org/10.1128/JB.184.21.6026-6036.2002}
\bibitem{jupe2012}
Florian Jupe, Leighton Pritchard, Graham J. Etherington, Katrin MacKenzie, Peter JA Cock, Frank Wright, Sanjeev Kumar Sharma1, Dan Bolser, Glenn J Bryan, Jonathan DG Jones, Ingo Hein: ``Identification and localisation of the NB-LRR gene family within the potato genome''. \textit{BMC Genomics} {\bf 13}: 75 (2012).
\href{http://dx.doi.org/10.1186/1471-2164-13-75}{http://dx.doi.org/10.1186/1471-2164-13-75}
\bibitem{cock2010}
Peter J. A. Cock, Christopher J. Fields, Naohisa Goto, Michael L. Heuer, Peter M. Rice: ``The Sanger FASTQ file format for sequences with quality scores, and the Solexa/Illumina FASTQ variants''.  \textit{Nucleic Acids Research} {\bf 38} (6): 1767--1771 (2010). \href{http://dx.doi.org/10.1093/nar/gkp1137}{doi:10.1093/nar/gkp1137}
\bibitem{brown1999}
Patrick O. Brown, David Botstein: ``Exploring the new world of the genome with DNA microarrays''. \textit{Nature Genetics} {\bf 21} (Supplement 1), 33--37 (1999). \href{http://dx.doi.org/10.1038/4462}{doi:10.1038/4462}
\bibitem{talevich2012}
Eric Talevich, Brandon M. Invergo, Peter J.A. Cock, Brad A. Chapman: ``Bio.Phylo: A unified toolkit for processing, analyzing and visualizing phylogenetic trees in Biopython''.  \textit{BMC Bioinformatics} {\bf 13}: 209 (2012).  \href{http://dx.doi.org/10.1186/1471-2105-13-209}{doi:10.1186/1471-2105-13-209}
\bibitem{cornish1985}
Athel Cornish-Bowden: ``Nomenclature for incompletely specified bases in nucleic acid sequences: Recommendations 1984.'' \textit{Nucleic Acids Research} {\bf 13} (9): 3021--3030 (1985). \href{http://dx.doi.org/10.1093/nar/13.9.3021}{doi:10.1093/nar/13.9.3021}
\bibitem{cavener1987}
Douglas R. Cavener: ``Comparison of the consensus sequence flanking translational start sites in Drosophila and vertebrates.'' \textit{Nucleic Acids Research} {\bf 15} (4): 1353--1361 (1987). \href{http://dx.doi.org/10.1093/nar/15.4.1353}{doi:10.1093/nar/15.4.1353}
\bibitem{bailey1994}
Timothy L. Bailey and Charles Elkan: ``Fitting a mixture model by expectation maximization to discover motifs in biopolymers'', \textit{Proceedings of the Second International Conference on Intelligent Systems for Molecular Biology} 28--36. AAAI Press, Menlo Park, California (1994).
\bibitem{chapman2000}
Brad Chapman and Jeff Chang: ``Biopython: Python tools for computational biology''. \textit{ACM SIGBIO Newsletter} {\bf 20} (2): 15--19 (August 2000).
\bibitem{dehoon2004}
Michiel J. L. de Hoon, Seiya Imoto, John Nolan, Satoru Miyano: ``Open source clustering software''. \textit{Bioinformatics} {\bf 20} (9): 1453--1454 (2004). \href{http://dx.doi.org/10.1093/bioinformatics/bth078}{doi:10.1093/bioinformatics/bth078}
\bibitem{eisen1998}
Michiel B. Eisen, Paul T. Spellman, Patrick O. Brown, David Botstein: ``Cluster analysis and display of genome-wide expression patterns''. \textit{Proceedings of the National Academy of Science USA} {\bf 95} (25): 14863--14868 (1998). \href{http://dx.doi.org/10.1073/pnas.96.19.10943-c}{doi:10.1073/pnas.96.19.10943-c}
\bibitem{golub1971}
Gene H. Golub, Christian Reinsch: ``Singular value decomposition and least squares solutions''. In \textit{Handbook for Automatic Computation}, {\bf 2}, (Linear Algebra) (J. H. Wilkinson and C. Reinsch, eds), 134--151. New York: Springer-Verlag (1971).
\bibitem{golub1989}
Gene H. Golub, Charles F. Van Loan: \textit{Matrix computations}, 2nd edition (1989).
\bibitem{hamelryck2003a}
Thomas Hamelryck and  Bernard Manderick: 11PDB parser and structure class
implemented in Python''. \textit{Bioinformatics}, \textbf{19} (17): 2308--2310 (2003) \href{http://dx.doi.org/10.1093/bioinformatics/btg299}{doi: 10.1093/bioinformatics/btg299}. 
\bibitem{hamelryck2003b}
Thomas Hamelryck: ``Efficient identification of side-chain patterns using a multidimensional index tree''. \textit{Proteins} {\bf 51} (1): 96--108 (2003). \href{http://dx.doi.org/10.1002/prot.10338}{doi:10.1002/prot.10338}
\bibitem{hamelryck2005}
Thomas Hamelryck: ``An amino acid has two sides; A new 2D measure provides a different view of solvent exposure''. \textit{Proteins} {\bf 59} (1): 29--48 (2005). \href{http://dx.doi.org/10.1002/prot.20379}{doi:10.1002/prot.20379}.
\bibitem{hartigan1975}
John A. Hartiga. \textit{Clustering algorithms}. New York: Wiley (1975).
\bibitem{jain1988}
Anil L. Jain, Richard C. Dubes: \textit{Algorithms for clustering data}. Englewood Cliffs, N.J.: Prentice Hall (1988).
\bibitem{kachitvichyanukul1988}
Voratas Kachitvichyanukul, Bruce W. Schmeiser: Binomial Random Variate Generation. \textit{Communications of the ACM} {\bf 31} (2): 216--222 (1988). \href{http://dx.doi.org/10.1145/42372.42381}{doi:10.1145/42372.42381}
\bibitem{kohonen1997}
Teuvo Kohonen: ``Self-organizing maps'', 2nd Edition. Berlin; New York: Springer-Verlag (1997).
\bibitem{lecuyer1988}
Pierre L'Ecuyer: ``Efficient and Portable Combined Random Number Generators.''
\textit{Communications of the ACM} {\bf 31} (6): 742--749,774 (1988). \href{http://dx.doi.org/10.1145/62959.62969}{doi:10.1145/62959.62969}
\bibitem{majumdar2005}
Indraneel Majumdar, S. Sri Krishna, Nick V. Grishin: ``PALSSE: A program to delineate linear secondary structural elements from protein structures.'' \textit{BMC Bioinformatics}, {\bf 6}: 202 (2005). \href{http://dx.doi.org/10.1186/1471-2105-6-202}{doi:10.1186/1471-2105-6-202}.
\bibitem{matys2003}
V. Matys, E. Fricke, R. Geffers, E. G\"ossling, M. Haubrock, R. Hehl, K. Hornischer, D. Karas, A.E. Kel, O.V. Kel-Margoulis, D.U. Kloos, S. Land, B. Lewicki-Potapov, H. Michael, R. M\"unch, I. Reuter, S. Rotert, H. Saxel, M. Scheer, S. Thiele, E. Wingender E: ``TRANSFAC: transcriptional regulation, from patterns to profiles.'' Nucleic Acids Research {\bf 31} (1): 374--378 (2003). \href{http://dx.doi.org/10.1093/nar/gkg108}{doi:10.1093/nar/gkg108}
\bibitem{sibson1973}
Robin Sibson: ``SLINK: An optimally efficient algorithm for the single-link cluster method''. \textit{The Computer Journal} {\bf 16} (1): 30--34 (1973). \href{http://dx.doi.org/10.1093/comjnl/16.1.30}{doi:10.1093/comjnl/16.1.30}
\bibitem{snedecor1989}
George W. Snedecor, William G. Cochran: \textit{Statistical methods}. Ames, Iowa: Iowa State University Press (1989).
\bibitem{tamayo1999}
Pablo Tamayo, Donna Slonim, Jill Mesirov, Qing Zhu, Sutisak Kitareewan, Ethan Dmitrovsky, Eric S. Lander, Todd R. Golub: ``Interpreting patterns of gene expression with self-organizing maps: Methods and application to hematopoietic differentiation''. \textit{Proceedings of the National Academy of Science USA} {\bf 96} (6): 2907--2912 (1999). \href{http://dx.doi.org/10.1073/pnas.96.6.2907}{doi:10.1073/pnas.96.6.2907}
\bibitem{tryon1970}
Robert C. Tryon, Daniel E. Bailey: \textit{Cluster analysis}. New York: McGraw-Hill (1970).
\bibitem{tukey1977}
John W. Tukey: ``Exploratory data analysis''. Reading, Mass.: Addison-Wesley Pub. Co. (1977).
\bibitem{yeung2001}
Ka Yee Yeung, Walter L. Ruzzo: ``Principal Component Analysis for clustering gene expression data''. \textit{Bioinformatics} {\bf 17} (9): 763--774 (2001). \href{http://dx.doi.org/10.1093/bioinformatics/17.9.763}{doi:10.1093/bioinformatics/17.9.763}
\bibitem{saldanha2004}
Alok Saldanha: ``Java Treeview---extensible visualization of microarray data''. \textit{Bioinformatics} {\bf 20} (17): 3246--3248 (2004). 
\href{http://dx.doi.org/10.1093/bioinformatics/bth349}{http://dx.doi.org/10.1093/bioinformatics/bth349}
\bibitem{dale2011}
Ryan K. Dale, Brent S. Pedersen, Aaron R. Quinlan: ``Pybedtools: a flexible Python library for manipulating genomic datasets and annotations''.
\textit{Bioinformatics} {\bf 27} (24): 3423--3424 (2011).
\href{https://academic.oup.com/bioinformatics/article/27/24/3423/304825/Pybedtools-a-flexible-Python-library-for}{https://academic.oup.com/bioinformatics/article/27/24/3423/304825/Pybedtools-a-flexible-Python-library-for}
\end{thebibliography}
\end{document}

