\chapter{Retrieving records from VCF file}

This chapter gives an overview of the functionality of fetch methods included in \verb|Bio.VCF| \verb|parser| module.
Those functions enable users to retrieve particular records from VCF file describing structural variants,
depending on provided argument - for example interval of positions within which we seek for structural variants, a list
of such positions or feature type within which structural variants are present.


Functionality of those fetch methods includes retrieving structural variants corresponding to ranges
provided in BED files as well as features from GFF files. If you are not familiar with BED file format
you might be interested in \href{http://www.ensembl.org/info/website/upload/bed.html}{BED} and
\href{http://gmod.org/wiki/GFF3}{GFF}.


Most of implementation is based on \verb|pybedtools| module - \href{https://daler.github.io/pybedtools/}{pybedtools},
which support many file formats including VCF, BED and GFF, therefore \verb|pybedtools| package is required.


Due to enormous sizes of some BED and GFF files, some fetch methods are intended to enable user to provide
stream of particular GFF/BED file from chosen database. It is a beneficial solution allowing users to avoid
the necessity of saving very big files to local disc.


Below there is a brief description of each fetch method.
All of them have arguments \verb|verbose| and \verb|vcf| specifying if result should be printed to stdout
and whether new VCF.Reader object should be created, respectively.

\begin{description}
  \item[\texttt{fetch\_bed(bed\_file, verbose = True, vcf = None)}] \

    Fetches VCF file records that correspond to regions included in a BED file.
    Fetch is based on pybedtools 'intersect' method and returns a BedTool / VCF.Reader object of selected features.
    BED file must be specified.

  \item[\texttt{fetch\_bed\_fsock(stream, verbose = False, vcf=None)}] \

    This fetch works exactly the same as fetch\_bed(), except the BED file is not required.
    Intervals used for intersection with a VCF file are provided in stream object of chosen BED file.
    This method returns a BedTool object of selected VCF features or new VCF.Reader object.
    Stream of gzipped BED file is required.

  \item[\texttt{fetch\_multilocal(chrom, local\_list, verbose = False, vcf = None)}] \

    Fetches VCF records that correspond to intervals provided in 'local\_list'.
    Local\_list must be a list of tuples (start, end), where start and end coordinates are in the
    zero-based, half-open coordinate system. Function returns selected records as a BedTool object
    or new VCF.Reader object. Chromosome must be specified.

  \item[\texttt{fetch(chrom, interval = None, verbose = True, vcf=None)}] \

    Fetches those records from VCF file that correspond to selected chromosome and
    fit in selected interval (if provided).
    This method creates one-line pybedtool feature based on selected chromosome (and interval = [start,stop]),
    and then uses it in pybedtools intersection method.
    Function returns BedTool object representing selected VCF records or new VCF.Reader object.
    Chromosome must be specified and interval is optional.

  \item[\texttt{fetch\_gff(gff\_file, chrom, feature\_type, location = None, verbose = False, vcf = None)}] \

    This method enables to select desired features from a GFF/GFF2/GFF3 file and fetch VCF records
    that correspond to position of those features. Fetch is based on pybedtools 'intersection' method and returns
    a BedTool object of chosen VCF records or new VCF.Reader object.
    Gff file, chromosome and feature type are required.
    Selection of desired features from a GFF/GFF2/GFF3 file with a specified location is possible when
    provided optional parameter 'location=[start,end]'.

  \item[\texttt{fetch\_gff\_fsock(self, stream, chrom, feature\_type, location = None, verbose = False, vcf = None)}] \

    This method works exactly the same as fetch\_gff(), except the GFF file is not required.
    The GFF file is replaced with a stream object from chosen database.
    Method returns a BedTool object of selected VCF records or new VCF.Reader object.
    Stream of gzipped file must be provided.

\end{description}


The following sections present exemplary use of fetch methods.


\section{Create VCF.Reader object}
\label{sec:object}


Since all fetch methods are included in VCF.Reader class, we need to create \verb|Reader| object. But first of all,
we need to import Bio.VCF library:


\begin{verbatim}
>>> from Bio import VCF
>>> from VCF import parser

\end{verbatim}

and then create a VCF.Reader obcject. We can either create object from a local file or we can provide fsock
of corresponding VCF file:


\begin{verbatim}
>>> vcf=parser.Reader(open('Tests/VCF/chr13.vcf'))

\end{verbatim}

\subsection{Fetching records depending on BED/GFF file}


Suppose we have a BED file with positions of CTCF motif binding sites. We want to collect all structural variants
located within those sites. If we have this BED file locally on our computer, then we should simply run:


\begin{verbatim}
>>> vcf.fetch_bed('VCF/chr13bed.bed')

\end{verbatim}


or if we don't, we should provide a stream of chosen BED file and use:


\begin{verbatim}
>>> stream="ftp://ftp.ncbi.nih.gov/snp/organisms/human_9606/BED/bed_chr_13.bed.gz"
>>> vcf.fetch_bed_fsock(stream)

\end{verbatim}


Selected structural variants are returned in BedTools object (if vcf = None) and can be easily accessed, for example
if we want to see all information of every selected record:


\begin{verbatim}
>>> bed = vcf.fetch_bed('VCF/chr13bed.bed')
>>> for b in bed:
>>>     print (b.fields)
['chr13', '10', '.', 'G', 'GGT', '.', 'PASS', 'DP=91;SS=1;SSC=2;GPV=3.0109E-23;SPV=5.8324E-1',
'GT:GQ:DP:RD:AD:FREQ:DP4','0/1:.:36:13:22:62.86%:2,11,1,21']
['chr13', '20', '.', 'GT', 'G', '.', 'PASS', 'DP=77;SS=1;SSC=2;GPV=2.4504E-29;SPV=6.0772E-1',
'GT:GQ:DP:RD:AD:FREQ:DP4','1/1:.:28:5:22:81.48%:0,5,1,21']
['chr13', '40', '.', 'AAAC', 'A', '.', 'PASS', 'DP=42;SS=1;SSC=12;GPV=7.3092E-18;SPV=6.278E-2',
'GT:GQ:DP:RD:AD:FREQ:DP4','0/1:.:13:4:9:69.23%:4,0,9,0']
['chr13', '50', '.', 'TC', 'T', '.', 'PASS', 'DP=41;SS=1;SSC=2;GPV=9.8874E-23;SPV=5.3659E-1',
'GT:GQ:DP:RD:AD:FREQ:DP4','1/1:.:22:1:21:95.45%:1,0,15,6']
['chr13', '60', '.', 'T', 'TTAA', '.', 'PASS', 'DP=27;SS=1;SSC=2;GPV=1.4382E-14;SPV=5.5556E-1',
'GT:GQ:DP:RD:AD:FREQ:DP4','1/1:.:12:0:12:100%:0,0,0,12']
\end{verbatim}


Similarly, we can extract structural variants corresponding to positions of chosen features from GFF file. Suppose
we have a GFF3 file of Homo sapiens, but we are interested only in variants located within pseudogenes
at 13th chromosome. \verb|fetch_gff()| method is what we need:


\begin{verbatim}
>>> vcf.fetch_gff('VCF/HS_fetch_gff.gff3', '13', 'pseudogene')
\end{verbatim}


What is more, if we want variant within pseudogenes which are within specific positions, we can add
\verb|location| argument:


\begin{verbatim}
>>> vcf.fetch_gff('VCF/HS_fetch_gff.gff3', '13', 'pseudogene', location=[1, 18270822])
Finding SV corresponding to pseudogene and chosen position
chr13	20	.	GT	G	.	PASS	DP=77;SS=1;SSC=2;GPV=2.4504E-29;SPV=6.0772E-1	GT:GQ:DP:RD:AD:FREQ:DP4
1/1:.:28:5:22:81.48%:0,5,1,21
chr13	30	.	AC	A	.	PASS	DP=22;SS=1;SSC=7;GPV=1.3117E-10;SPV=1.9481E-1	GT:GQ:DP:RD:AD:FREQ:DP4
1/1:.:10:2:8:80%:0,2,0,8
chr13	40	.	AAAC	A	.	PASS	DP=42;SS=1;SSC=12;GPV=7.3092E-18;SPV=6.278E-2	GT:GQ:DP:RD:AD:FREQ:DP4
0/1:.:13:4:9:69.23%:4,0,9,0
chr13	50	.	TC	T	.	PASS	DP=41;SS=1;SSC=2;GPV=9.8874E-23;SPV=5.3659E-1	GT:GQ:DP:RD:AD:FREQ:DP4
1/1:.:22:1:21:95.45%:1,0,15,6
chr13	60	.	T	TTAA	.	PASS	DP=27;SS=1;SSC=2;GPV=1.4382E-14;SPV=5.5556E-1	GT:GQ:DP:RD:AD:FREQ:DP4
1/1:.:12:0:12:100%:0,0,0,12
chr13	9542346	.	T	TTAA	.	PASS	DP=27;SS=1;SSC=2;GPV=1.4382E-14;SPV=5.5556E-1	GT:GQ:DP:RD:AD:FREQ:DP4
1/1:.:12:0:12:100%:0,0,0,12
\end{verbatim}


As in case of fetch depending on BED file, we can use stream of GFF file instead of local file.


\subsection{Fetching records depending on interval}


We might have a VCF file which have records referring to different chromosomes. In such case, we want to be able to
analyze only those variants that are located on chromosome of our interest. We can simply use \verb|fetch()| method
on our \verb|vcf| object, but as we want to analyze fetched records further, we should use argument specifying
new vcf.Reader object:


\begin{verbatim}
>>> v=vcf.fetch('chr13',vcf='new_vcf')
>>> v
<parser.Reader object at 0x7eff4df8f2b0>

\end{verbatim}


Now, when we have all variants located on chromosome 13, we can choose only those within specified intervals:


\begin{verbatim}
>>> v.fetch_multilocal('chr13',[[10, 30], [80, 100], [85837129, 100000000]])
chr13	20	.	GT	G	.	PASS	DP=77;SS=1;SSC=2;GPV=2.4504E-29;SPV=6.0772E-1	GT:GQ:DP:RD:AD:FREQ:DP4
1/1:.:28:5:22:81.48%:0,5,1,21
chr13	30	.	AC	A	.	PASS	DP=22;SS=1;SSC=7;GPV=1.3117E-10;SPV=1.9481E-1	GT:GQ:DP:RD:AD:FREQ:DP4
1/1:.:10:2:8:80%:0,2,0,8
chr13	85837130	.	T	TTAA	.	PASS	DP=27;SS=1;SSC=2;GPV=1.4382E-14;SPV=5.5556E-1	GT:GQ:DP:RD:AD:FREQ:DP4
1/1:.:12:0:12:100%:0,0,0,12
<BedTool(/tmp/pybedtools.dbyw5lct.tmp)>

\end{verbatim}


VCF file created after \verb|vcf.fetch| is saved to local disc and can be easily achieved with
VCF.Reader() anytime needed.