\chapter{VCF Databases}

Bio.VCF.databases is a Biopython module that supports \verb|1001 Genomes| and \verb|dbSNP| databases.
Since most of VCF files are of enormous sizes, it is convenient to use streams instead of local files.
Functionality of this package enables users to search through above-mentioned databases, and use selected streams
of VCF files as VCF.Reader objects in further analysis.


If you are not familiar with \verb|1001 Genomes| and \verb|dbSNP| databases you might be interested in
\href{http://1001genomes.org/}{1001 Genomes} and \href{https://www.ncbi.nlm.nih.gov/SNP/}{dbSNP}. However, following
sections provide brief description of methods searching through both databases as well as some information about them.


\section{1001 Genomes database}
\label{sec:object}


\verb|1001 Genomes| database was launched to discover whole-genome variation of the \verb|Arabidopsis Thaliana| strains.
\verb|Bio.VCf.databases| module supports current database's version which includes analysis of 1135 genomes achieved
during the first phase of the project.
Searching through the database can be based on strain's name, ec number, eco type, as well as on information of its
origin - such as country, longitude or latitude.


Below there is a brief description of \verb|thousandgenome()| methods:


\begin{description}
  \item[\texttt{thousandgenomes(name, ecotype, ecnumber, country, longitude, latitude)}] \

    This method enables to search through \verb|1001 Genomes| database for VCF file corresponding to selected
    Arabidopsis Thaliana strain, origin country or longitude and/or latitude where Arabidopsis Thaliana live.

    Country name must be chosen from: "USA", "FRA", "CZE", "AUT", "KGZ", "TJK", "SWE", "UK", "GER", "KAZ",
    "BEL", "CPV", "ESP", "RUS", "NED", "FIN", "SUI", "ITA", "IRL", "POR", "EST", "DEN", "IND", "LTU", "JPN", "POL", "NOR",
    "CAN", "UKR", "AZE", "GEO", "ARM", "MAR", "CRO", "BUL", "GRC", "SVK", "ROU", "UZB", "SRB", "CHN", "IRN", "LBN", "MAR",
    "AFG".
    Longitude and latitude should be provided as an interval - longitude = (int1, int2), latitude = (int1,int2).
    Strain names, eco types and ec numbers are unique, country, while longitude and latitude enables users to seek for
    multiple VCF files corresponding to selected features.
    Method returns VCF.Reader object of selected VCF file stream from 1001 Genomes Database or a list of
    such VCF.Reader objects.

\item[\texttt{download(vcf\_reader, path\_filename)}] \

    Downloads VCF file corresponding to stream on which provided VCF.Reader object is initialized.
    VCF.Reader object as well as download directory and filename are required.


\end{description}

Let's try some functionality of the module:

Suppose we want to analyse structural variants identified in "CYR" strain of Arabidopsis Thaliana. First of all,
we need to find VCF file corresponding to our selected strain. To do so, we can simply type:


\begin{verbatim}
>>> from Bio.VCF import databases
>>> vcf = databases.thousandgenomes("CYR")
>>> vcf
<Bio.VCF.parser.Reader object at 0x7f4bf49feef0>

\end{verbatim}


Now, we can continue analysis for example by running methods from VCF.Reader class.


We may also be interested in retrieving all structural variants that are identified in strains occupying a particular
territory, such as area of longitude (-73.1494, -73.1492) and latitude (40.9063, 40.9064). Module's \verb|thousandgenomes()|
method can manage this:


\begin{verbatim}
>>> vcf_list = databases.thousandgenomes(longitude=(-73.1494, -73.1492), latitude=(40.9063, 40.9064))
>>> vcf_list
[<Bio.VCF.parser.Reader object at 0x7f4bf4a0d978>, <Bio.VCF.parser.Reader object at 0x7f4bf4a1e080>]

\end{verbatim}


Method returned two VCF.Reader objects which we can analyse further or simply download to local disc:


\begin{verbatim}
>>> for v in vcf_list:
>>>     databases.download(v,'database_download.gz')
\end{verbatim}